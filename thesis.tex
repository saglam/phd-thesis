\documentclass[11pt,proquest]{uwthesis}[2016/11/22]

\usepackage[utf8]{inputenc}
\usepackage{amsmath, amsthm, amssymb}
\usepackage{lmodern}
\usepackage{microtype}
\usepackage[usenames]{xcolor}
\usepackage{hyperref}
\usepackage{aliascnt}
\usepackage{multirow}
\usepackage{mathtools}
\usepackage{bbm}
\usepackage{float}
\usepackage{tikz}
\usepackage{enumerate}
\usepackage{nicefrac}

\include{definitions/theoremlike}
\include{definitions/math}
\include{definitions/fields}
\include{definitions/probability}
\include{definitions/entropy}
\include{definitions/linearalgebra}
\include{definitions/communication}

\setcounter{tocdepth}{1}
\definecolor{Blue}{RGB}{0, 60, 60}
\definecolor{Red}{RGB}{0,35,60}
\hypersetup{
    colorlinks,
    citecolor=Red,
    filecolor=Blue,
    linkcolor=Blue,
    urlcolor=Blue
}
\DeclareMathOperator{\down}{down}
\DeclareMathOperator{\up}{up}
\DeclareMathOperator{\wt}{wt}
\DeclareMathOperator{\EE}{EE}
\DeclareMathOperator{\Match}{Match}

\newcommand{\propref}[1] {%
\hyperref[#1]{Property~(P\ref*{#1})}}


\usetikzlibrary{positioning}

\begin{document}

\prelimpages

\Title  {Settling the complexity of $k$-disjointness
         and the $k$-Hamming distance problems}
\Author {Mert Sağlam}
\Year   {2019}
\Program{Computer Science and Engineering}

\Chair{Shayan Oveis Gharan}{Title of Chair}{Computer Science and Engineering}
\Signature{Paul Beame}
\Signature{Anup Rao}
\Signature{Ioana Dumitriu, GSR}

\copyrightpage
\titlepage  
\setcounter{page}{-1}


% !TeX root = thesis.tex
\abstract{%
Suppose two parties, traditionally called Alice and Bob, are
given respectively the inputs $x\in\mathcal{X}$ and
$y\in\mathcal{Y}$ to a function
$f\colon\mathcal{X}\times\mathcal{Y}\to\mathcal{Z}$ and are
required to compute $f(x,y)$.
Since each party only has one part of the input, they can
compute $f(x,y)$ only if some communication takes place between
them. The communication complexity of a given function is the
minimum amount of communication (in bits) needed to evaluate it
on any input with high probability.

We study the communication complexity of two related problems,
the $k$-Hamming distance and $k$-disjointness and give tight
bounds to both of these problems: The $r$-round communication
complexity of the $k$-disjointness problem is
$\Theta(k\log^{(r)}k)$, whereas a tight $\Omega(k\log k)$ bound
holds for the $k$-Hamming distance problem for any number of
rounds.

The lower bound direction of our first result is obtained by
proving a {\em super-sum} result on computing the OR of $n$
equality problems, which is the first of its kind.
Using our second bound, we settle the complexity of various
property testing problems such as $k$-linearity, which was open
since 2002 or earlier. Our lower bounds are obtained via
information theoretic arguments and along the way we resolve a
question conjectured by Erdős and Simonovits in 1982, which
incidentally was studied even earlier by Blakley and Dixon in
1966.
}


\tableofcontents
\listoffigures
\listoftables
\thispagestyle{plain}
\textpages

%% !TeX root = thesis.tex
\chapter{Introduction}
\label{sec:introduction}

Communication complexity is a versatile abstraction of communication requirements of computational tasks, leaving aside the compute power required and focusing on bandwidth, latency and everything else communication related.


What is communication complexity

Who introduced it

Why is it useful

Useful on its own; an interactive generalization of Shannon's theory of communication.



In a two player communication problem the players, 
named Alice and Bob, receive separate inputs, 
respectively $x\in \mathcal{X}$ and $y\in \mathcal{Y}$, 
and they communicate in order to compute the value 
$f(x,y)$ of a function
$f\colon \mathcal{X}\times \mathcal{Y}\to\binary$ 
(known to both players). 
In an $r$-round protocol, the players can take at most 
$r$ turns alternately sending each other a message 
(that is, a bit string) and the last player to receive 
a message declares the output of the protocol.
A protocol can be {\em deterministic} or {\em randomized}; 
in the latter case the players can base their actions on 
a common random source and we measure the 
{\em error probability}: the maximum over inputs 
$(x,y)\in \mathcal{X}\times\mathcal{Y}$, of the probability 
that the output of the protocol differs from $f(x,y)$. 
The {\em communication cost} of a protocol is the maximum, 
over the inputs and the random string, of the total number 
of bits sent between the players.
For a function 
$f\colon \mathcal{X}\times\mathcal{Y}\to\binary$, 
an integer $r$ and $\delta\in[0,1]$, we denote by 
$R^r_{\delta}(f)$ the minimum over all protocols
for $f$ having $r$-rounds and error probability at most 
$\delta$, of the communication cost incurred. We define 
$R_{\delta}(f)$ similarly, but we take the maximum over 
$\delta$-error protocols with no restriction on the number 
of rounds it uses.

In the $k$-Hamming distance problem, denoted $\Ham^n_k$,
the players receive length-$n$ bit strings, respectively 
$x,y\in\cube$, and are required determine if 
$\lone{x-y}< k$ or not.
There is a well known one-round communication protocol
which accomplishes this with error probability $\delta$ 
by communicating $O(k\log\nparen{k/\delta})$ bits.

\begin{theorem}
[e.g., Huang, Shi, Zhang and Zhu \cite{HuangSZZ2006}]
\label{thm:ub}
It holds that $$R^1_\delta(\Ham^n_k) = 
O(\min\set{k\log\nparen{k/\delta}, k\log (n/k)}).$$
\end{theorem}

Highly related to the $\Ham^n_k$ is the $k$-disjointness 
problem $\Disj^n_k$, wherein the players each receive a 
$k$-subset of $[n]$ and their goal is to determine if their 
sets intersect.
Notice that $\Disj^n_k$ can be seen as a promise version of
$\Ham^n_{2k-2}$ where each player is guaranteed to have a string
with Hamming weight $k$: the sets are disjoint if and only if the
Hamming distance between the characteristic vectors of the sets
is more than $2k-2$. Therefore any upper bound for the $\Ham^n_k$ 
carries over to $\Disj^n_k$ and any lower bound for $\Disj^n_k$
carries over to $\Ham^n_k$.
Around 1993, 
Håstad and Wigderson \cite{HastadW2007} showed that
there is a more efficient protocol for $\Disj^n_k$ than that implied 
by \autoref{thm:ub}, which communicates only $O(k)$ bits, but over
$O(\log k)$ rounds.

On the lower bounds side, the result of \cite{KalyanasundaramS1992}
implies that $\Omega(k)$ bits is needed for these problems even
if one uses arbitrarily large number of round protocols.
In \cite{BuhrmanGMW2012} it was shown that
any 1-round protocol for $\Disj^n_k$ needs to communicate
at least $\Omega(k \log k)$ bits when $k^2<n$
(this result was proven later in \cite{DasguptaKS12} also).
In Theorem~3.2 of \cite{Saglam2011}, an $\Omega(k\log(1/\delta))$
bound for $1$-round complexity of $\Disj^n_k$ was shown even 
when Bob receives just one element (i.e., the indexing problem) 
for $k<\delta n$ and a 
slightly more general result was shown in \cite{JayramW2011}.
Finally Sağlam and Tardos in \cite{SaglamT2013}, 
which we review in sections \ref{sec:disj-upperbound} and 
\ref{sec:disj-lowerbound} of this document, the communication complexity
of $\Disj^n_k$ was settled:
\begin{align*}
R^r_{1/3}(\Disj^n_k)=\Theta(k\log^{(r)}k)
\end{align*}
for $1\le r \le \log^*k$ and $k^2<n$.
Our upper bound solves the disjointness problem with 
error probability at most $1/\exp k + 1/\exp^{(r)}(c\log^{(r)}k)$ 
for any $c>1$ by communicating
$O(k \log^{(r)} k)$ bits over $r$ rounds. In fact bulk of the bits
is sent in the first round and the rest of the rounds
amount to an $O(k)$ bits of communication. Taking $r=\log^* k$,
this leads to an $O(k)$ bits protocol with error probability 
that is exponentially small in $k$.
Their lower bound shows that at least one
message of size 
$\Omega(k\log^{(r)} k)$ bits needs to be sent by any 
$r$-round protocol, even if it has error probability $1/3$.
Prior to this work, this lower bound provided the strongest lower
bound for $\Ham^n_k$ also, along with the incomparable bound of 
$\Omega(k\log (1/\delta))$ due to \cite{BlaisBG2014}
which holds for any number of rounds, which we discuss shortly.

\input{hamvsdisj}

To summarize the above results, the 1-round communication 
complexity of both $\Disj^n_k$ and $\Ham^n_k$ is 
$\Theta(k\log(k/\delta))$ by 
\cite{BuhrmanGMW2012, Saglam2011, JayramW2011} and 
\cite{HuangSZZ2006}. We know that $\Disj^n_k$ can be solved 
much more efficiently if one is allowed multiple rounds:
firstly the $\log k$ factor can be removed \cite{HastadW2007} 
and secondly the error probability can be brought down to 
$\exp(-k)$, by using no more than $\log^* k$ rounds 
\cite{SaglamT2013}. It is an interesting question whether 
similar efficiency improvements can be obtained for $\Ham^n_k$ 
also, by using multiple rounds.
The first separation of $\Disj^n_k$ and $\Ham^n_k$ was proven
in \cite{BlaisBG2014}, which shows that $\Omega(k\log(1/\delta))$
lower bound holds for any protocol solving $\Ham^n_k$. 
Therefore in $\Ham^n_k$, we get no improvements in error probability
by interactive communication. It remained an open 
question whether {\em any} improvement can be made 
at all to the 1-round protocol by communicating interactively.
In \autoref{sec:ham-lowerbound} we answer this question negatively: 
\begin{theorem}
\label{thm:klogk}
Any two party $\delta$-error randomized protocol
solving the $k$-Hamming distance problem 
over length-$n$ strings communicates 
at least $\Omega(k\log (k/\delta))$ 
bits for $k^2\le \delta n$.
\end{theorem}
\chapter{Introduction}
\newpage\newpage\newpage\newpage
\section{Outline}
In this thesis we study the communication complexity of the 
$k$-disjointness and $k$-Hamming distance problems. 
In \autoref{sec:ham}, we give our $k$-disjointness protocol,
and show a matching lower bound after we take a brief detour 
into combinatorics to develop our isoperimetric inequality used in the lower bound proof.

Our lower bound for the $k$-Hamming distance, given in \autoref{sec:ham},
requires us to understand how heat behaves in discrete time. We develop this
theory in \autoref{sec:heat}. The notation we use and some formal preliminaries
are given in \autoref{sec:prelim}.

\newpage\newpage
% !TeX root = thesis.tex
\chapter{Preliminaries}
\label{sec:prelim}

In this chapter we list the notation we use, conventions we adopt
and formally and define formally and in more detail 
the computational models that we work with in this thesis. 

\section{Notation}
\label{sec:prelim:notation}

We denote by $[n]$ the set $\set{1,2,\ldots,n}$. 
Throughout this thesis, we
take $\exp$ and $\log$ functions to the base 2.
For exponentials and logarithms in other bases
such as $b$, we write $\exp_b$ and $\log_b$. 
We adopt the convention $\ln \defeq \log_e$ where $e=2.718\ldots$
is the limiting value of $(1+1/n)^n$ as $n\to \infty$.
We will also use
the iterated versions of these functions:
\begin{align*}
  \log^{(0)}x&\defeq x, 
             & 
  \exp^{(0)}x&\defeq x,\\
  \log^{(r)}x&\defeq \log\nparen{\log^{(r-1)}x}, 
             & 
  \exp^{(r)}x&\defeq \exp\nparen{\exp^{(r-1)}x}
  \quad\text{for $r\geq 1$}.
\end{align*}
Moreover we define the $\log^* x$ to be the smallest integer $r$ 
for which $\log^{(r)} x<2$. For instance, we have $\log^* 16 = 3$ and
$\log^* 2^{16} = 4$.
The function $\log^*$ is conventionally called the 
{\em iterated logarithm}, which we adopt. 
To differentiate, we call the function $\log^{(r)}$ the 
{\em $r$-iterated logarithm}.

\section{Random variables and distributions}
\label{sec:prelim:rand}
Let $\Omega$ be a countable set. 
For a function $\mu\colon\Omega\to\realspos$ and a set 
$\Psi\subseteq\Omega$, we use the shorthand
\begin{align*}
\mu(\Psi) \defeq \sum_{x\in\Psi}\mu(x).
\end{align*}
A function $\mu\colon\Omega\to\realspos$
is said to be a distribution on $\Omega$ if
$\mu(\Omega) = 1$ and a subdistribution
if $\mu(\Omega) \le 1$. For a function
$\mu$ on $\Omega$, we define
\begin{align*}
\supp(\mu)\defeq \setbuilder{x\in \Omega}{\mu(x)>0}.
\end{align*}
For two distributions $\mu\colon\Omega_1\to\realspos$
and $\nu\colon\Omega_2\to\realspos$, let us denote by
$\mu\nu$ the distribution on $\Omega_1\times\Omega_2$
given by $(\mu\nu)(x_1,x_2) = \mu(x_1)\nu(x_2)$.

For a discrete random variable $X$, we denote by
$\dist(X)$ the distribution function of $X$ and
we define $\supp(X)\defeq\supp(\dist(X))$. 
If $X$ is so that $\dist(X)\colon\Omega\to\realspos$,
then we say that $X$ has sample space $\Omega$.
Two 
random variables $X$ and $Y$ are said to be 
independent if $\dist(XY) =\dist(X)\dist(Y)$.

\begin{lemma}[Jensen \cite{Jensen1906}, Formula (5)]
\label{lem:jensen}
Let $X$ be a real-valued random variable and
$f$ be a convex function. We have
$\E\sparen{f(X)} \ge f(\E[X])$. When
$f$ is strictly convex, the inequality holds
with equality if and only if $X$ is constant
with probability 1.
\end{lemma}

\section{Facts from information theory}
\label{sec:prelim:info}
In this section we review the definitions and 
facts we use from information theory.
Let $\mu$ and $\nu$ be two 
nonnegative functions on $\Omega$.
The Kullback-Leibler divergence \cite{Wald1945, KullbackL1951}
of $\mu$ from $\nu$,
denoted $\kldiv{\mu}{\nu}$, is defined by
%
\begin{align}
\label{eq:kldiv}
  \kldiv{\mu}{\nu} \defeq \sum_{x\in \Omega}
      \mu(x)\log\frac{\mu(x)}{\nu(x)}\,.
\end{align}
%
Here, if $\mu(x)=0$ for some $x$, then its contribution
to the summation is taken as 0, even when $\nu(x)=0$.
The divergence is undefined if there is an
$x\in \Omega$ such that $\mu(x)>0$ and
$\nu(x)=0$.
It can be shown that if the related series 
converges for the right hand side of \autoref{eq:kldiv},
it converges absolutely, 
which justifies leaving the summation order unspecified.
A fundamental property of 
$\kldiv{\cdot}{\cdot}$ is that
the divergence of a distribution from a subdistribution 
is always nonnegative.
\begin{lemma}[Gibbs \cite{Gibbs1902}, Theorem VIII]
\label{lem:gibbs}
Let $\mu,\nu\colon\Omega\to\reals$ be such that $\mu$ is a
distribution and $\nu$ is a subdistribution.
We have $\kldiv{\mu}{\nu}\ge 0$ with equality if and
only if $\mu=\nu$.
\end{lemma}
\begin{lemma}[Kullback and Leibler \cite{KullbackL1951},
Lemma 3.2]
\label{lem:klcond}
Let $\mu,\nu\colon\Omega\to\realspos$ be so that
$\mu$ is a distribution on $\Omega$ and 
$\supp(\mu) = \Psi\subseteq\Omega$. We have
\begin{align*}
\kldiv{\mu}{\nu}\ge -\log \nu(\Psi)
\end{align*}
with equality if and only if $\mu(x) = \nu(x)/\nu(\Psi)$ for
$x\in\Psi$ and $\mu(x) = 0$ for $x\notin\Psi$.
\end{lemma}
\begin{proof}
By \autoref{eq:kldiv} we write
\begin{align*}
\kldiv{\mu}{\nu} &= -\sum_{x\in \Psi}
    \mu(x)\log\frac{\nu(x)}{\mu(x)}
  \ge -\log\sum_{x\in \Psi}\nu(x) =-\log \nu(\Psi)\,,
\end{align*}
where the inequality follows from
\autoref{lem:jensen} and concavity of
$z\mapsto\log z$ on $\realspos$. 
If $\mu(x) = \nu(x)/\nu(\Psi)$ for $x\in\Psi$, we have 
$\kldiv{\mu}{\nu}=-\log \nu(\Psi)$ by direct computation.
Otherwise $\kldiv{\mu}{\nu}>-\log\nu(\Psi)$ by strict 
concavity of $z\mapsto\log z$.
\end{proof}

We extend the divergence notation
$\kldiv{\cdot}{\cdot}$ to apply to random
variables as follows. Let $X,Y$ be discrete random
variables on the same sample space $\Omega$. Define
\begin{align}
\label{eq:klrv}
\kldiv{X}{Y}\defeq \kldiv{\dist(X)}{\dist(Y)}.
\end{align}
%
With this notation in hand, we are ready to define the 
conditional divergence. Let $X_1X_2$ and $Y_1Y_2$ be 
random variables defined on the sample space $\Omega_1\times \Omega_2$.
The divergence of $X_1\emid X_2$ from $Y_1\emid Y_2$ is
defined by 
\begin{align}
\kldiv{X_1\emid X_2}{Y_1\emid Y_2}
    \defeq \E_{x_2\sim X_2}
        \kldiv{X_1\emid X_2=x_2}{Y_1\emid Y_2 = x_2}.
\end{align}
Here, for each $x_2\in\supp(X_2)$, 
$X_1\emid X_2=x_2$ and $Y_1\emid Y_2 = x_2$ 
are random variables on the sample space
$\Omega_1$ obtained from, respectively
$X_1X_2$ and $Y_1Y_2$, by conditioning on the
second coordinate equaling $x_2$.
% !TeX root = thesis.tex
\begin{lemma}[e.g., \cite{CoverT2006}]
\label{lem:klchain}
Let $X_1X_2$ and $Y_1Y_2$ be random variables, both on the
sample space $\Omega_1\times \Omega_2$. We have
\begin{align*}
\kldiv{X_1X_2}{Y_1Y_2} = 
    \kldiv{X_1}{Y_1} + \kldiv{X_2\emid X_1}{Y_2\emid Y_1}.
\end{align*}
\end{lemma}

\begin{proof}
Let $\mu,\nu\colon\Omega_1\times \Omega_2\to \realspos$
be the distributions of respectively $X_1X_2$ and $Y_1Y_2$.
Using the shorthands
$\mu(\Omega_1,x_2)\defeq\sum_{x_1\in\Omega_1}\mu(x_1,x_2)$
and 
$\nu(\Omega_1,x_2)\defeq\sum_{x_1\in\Omega_1}\nu(x_1,x_2)$,
we write
\begin{align*}
  \kldiv{X_1\emid X_2}{Y_1\emid Y_2} &=
     \sum_{x_2\in\Omega_2}\mu(\Omega_1,x_2)
       \sum_{x_1\in\Omega_1}\frac{\mu(x_1,x_2)}
           {\mu(\Omega_1,x_2)}
       \log\frac{\mu(x_1,x_2)\nu(\Omega_1,x_2)}
                {\nu(x_1,x_2)\mu(\Omega_1,x_2)}\\
  &= \sum_{x_1,x_2}\mu(x_1,x_2)
       \log\frac{\mu(x_1,x_2)}{\nu(x_1,x_2)} +
     \sum_{x_2}
       \mu(\Omega_1,x_2)\log\frac{\nu(\Omega_1,x_2)}
           {\mu(\Omega_1,x_2)}
\end{align*}
by splitting the terms inside the logarithm. 
Using \autoref{eq:kldiv} together with \autoref{eq:klrv}
 we conclude
\begin{align*}
\kldiv{X_1\emid X_2}{Y_1\emid Y_2}  &= \kldiv{X_1X_2}{Y_1Y_2} - \kldiv{X_2}{Y_2}.
\end{align*}
Rearranging, we obtain the statement of the lemma.
\end{proof}

The next lemma establishes that the Kullback-Leibler divergence
is jointly convex in its parameters. This fact is also called 
the data processing inequality for Kullback-Leibler divergence. 
\begin{lemma}
\label{lem:klconvex}
Let $\mu_1,\mu_2\colon\Omega_1\to\realspos$ be distributions
supported on $\Omega_1$ and let 
$\nu_1,\nu_2\colon\Omega_2\to\realspos$ be distributions
supported on $\Omega_2$. For any $a\in[0,1]$, we have 
\begin{align*}
\kldiv{a\mu_1 + (1-a)\mu_2}{a\nu_1 + (1-a)\nu_2}
  \le a\kldiv{\mu_1}{\nu_1} + (1-a)\kldiv{\mu_2}{\nu_2}
\end{align*}
\end{lemma}

Further for two reals $p,q\in[0,1]$, we use the shorthand notation 
$\kldivb{p}{q}$ to denote the divergence of the random variable $P$
from $Q$ where $P,Q$ are Bernoulli random variables with expectation
respectively $p$ and $q$.


\subsection{Mutual information and Shannon entropy}
\label{sec:prelim:muti}
Let $X$ and $Y$ be jointly distributed random variables.
The 
mutual information of $X$ and $Y$, 
denoted $\muti{X}{Y}$, is defined as
\begin{align}
\muti{X}{Y} \defeq \kldiv{\dist(X,Y)}{\dist(X)\dist(Y)}.
\label{eq:muti-def}
\end{align}
The mutual information of a random variable 
with itself, i.e., the quantity $\muti{X}{X}$ is 
called the Shannon entropy of $X$ and denoted by $\ent{X}$.
If $X\in[t]^n$ and $L\subseteq[n]$, then the projection of $X$
to the coordinates in $L$ is denoted by $X_L$. Namely, $X_L$ is
obtained from $X=(X_1,\ldots,X_n)$ by keeping only the
coordinates $X_i$ with $i\in L$. The following lemma of Chung et
al.~\cite{ChungGFS1986} relates the entropy of a variable to the
entropy of its projections.
\begin{lemma}[Chung et al.\ \cite{ChungGFS1986}]
\label{lem:ent-subset}
Let $\supp(X)\subseteq[t]^n$. We have
$\frac{l}{n}\ent{X} \leq \E_L[\ent{X_L}]$, where the expectation is taken for
a uniform random $l$-subset $L$ of $[n]$.
\end{lemma}

\section{Concentration bounds}
Let $X_1,\ldots, X_n$ be random
variables such that $\E[X_i] = \epsilon$ for 
some $0\leq \epsilon\leq 1$. Define $X = 
X_1+\cdots + X_n$. By linearity of 
expectation we have $\E[X] = \epsilon n$.
If each $X_i$ is chosen independently,
due  to the concentration of measure phenomenon,
it is well understood that $X$ takes values 
close to its expectation with very high probability.
The classical results of Chernoff 
\cite{Chernoff1952} and Hoeffding 
\cite{Hoeffding1963} give quantitative and 
intuitive bounds on the deviation probability.
\begin{theorem}[Chernoff \cite{Chernoff1952}]
\label{thm:chernoff}
Let $X=X_1+\cdots+ X_n$, where $X_i$ for 
$i\in [n]$ are independent binary random 
variables with expectation $\epsilon$. Then 
for any $\epsilon\leq\gamma\leq 1$ we have 
\begin{align*}
  \Pr\sparen{X\geq\gamma n}
    \leq \exp\nparen{-n\kldivb{\gamma}{\epsilon}}.
\end{align*}
\end{theorem}

\begin{proof}
Let $E$ be the event that $X\ge \gamma n$. 
Let $\mu$ be the Bernoulli distribution that 
equals 1 with probability $\epsilon$. We have
\begin{align*}
-\log\Pr[E]
  &= \kldiv{\dist(X_1\ldots X_n\emid E)}{\mu^n} 
  & \text{(by \autoref{lem:klcond})}\\
%
  &\ge \sum_{i=1}^n \kldiv{\dist(X_i\emid E)}{\mu} 
  &\text{(by \autoref{lem:klchain})}\\
%
  &\geq n \kldivb{\gamma}{\epsilon}
  &\text{(as $\kldivb{\delta}{\epsilon}
    \geq\kldivb{\gamma}{\epsilon}$ for $\epsilon\leq \gamma\leq\delta$)}
\end{align*}
Hence,
$\Pr[E]
  \le \exp\nparen{-n\D_2(\gamma\dmid \epsilon)}$ as required.
\end{proof}


\section{Communication complexity and protocols}
\label{sec:prelim:comm}

\paragraph{The model.} In the two party communication complexity model we have two
parties called respectively Alice and Bob who are required to
evaluate a function
$f\colon\mathcal{X}\times\mathcal{Y}\to\mathcal{Z}$ (known to
both of them) on some input $(x,y)$ where $x$ is revealed to
Alice only and $y$ is revealed to Bob only.

In the randomized variant of this model, which is what we solely
study in this thesis, the players also have access to a shared
random source. Without loss of generality, the random source can
be taken as an infinite read-only string of bits, chosen
uniformly at random. The players, having received their inputs
$x$ and $y$ respectively, and with access to the shared random
source, engage in a dialogue by alternately sending each other
messages in rounds, in order to compute $f(x,y)$. Here, a
message is a bit string of arbitrary length.

\paragraph{Protocols.}
A {\em communication protocol} specifies, for each round, which
player's turn it is to speak and what message should be sent and
whether the protocol terminates with some answer. The protocol
specifies the message to be sent through a function mapping the
random string, the current players input and all the messages
the current player received so far to a bit string which is to be
sent to the other player. This ensures that the message sent by
the players, say Alice for illustration, 
can depend on only information known to her at the
time: her own input, the shared random source and the message
she received in the previous rounds.
Some message are marked as answers; instead of being sent to the
other player, these message are to be announced as the output of
the protocol, after which the protocol terminates. We say that a
communication protocol for a function
$f\colon\mathcal{X}\times\mathcal{Y}\to\mathcal{Z}$ has at most $\delta$-error
if for any input pair $(x,y)\in\mathcal{X}\times\mathcal{Y}$
with probability at least $1-\delta$ the output of the protocol
equals $f(x,y)$, where the probability is over the random
choices that come from the shared random source.

In an $r$-round protocol there are at most $r$ messages
(excluding the output message) sent for any input and any
configuration of the shared random source. 
To illustrate the way the number of messages is counted, consider 
the following protocol. Alice sends a single message to Bob and 
in return Bob replies with another message after which Alice
announces the answer of the protocol. This protocol has two rounds.
The complexity of a protocol is defined to be the the maximum, over all input pairs
$(x,y)\in\mathcal{X}\times\mathcal{Y}$ and all choices of the
random source, of the total number of bits sent by the two
parties.

Let $P$ be a protocol for a function 
$f\colon\mathcal{X}\times\mathcal{Y}$. We denote by $P(x,y,r)$
the output of the protocol on input $x\in\mathcal{X}$, $y\in\mathcal{Y}$
and when the shared random string is fixed to $r$. We denote by $P(x,y)$ the random
variable denoting the output of the protocol on inputs $(x,y)$.
The transcript of the protocol $P$, denoted $\Pi_P(x,y,r)$, 
is the concatenation of all the messages sent by the players 
(excluding the answer of the protocol) on inputs
$x\in\mathcal{X}$, $y\in\mathcal{Y}$ and when the shared random string is $r$.
Likewise, $\Pi_P(x,y)$ denotes the random variable entailing the communication
took place the two players when the random source is not fixed. 
We denote by $\abs{\Pi_P}$ the length of the transcript, in bits.

\paragraph{Communication complexity}
The $\delta$-error randomized communication complexity of a
function $f\colon\mathcal{X}\times\mathcal{Y}\to\mathcal{Z}$,
denoted $R_\delta(f)$, is the minimum over all $\delta$-error
protocols for $f$, of the complexity of the protocol. We also
define the $r$-round $\delta$-error randomized communication
complexity of a function $f$, denoted $R^r_\delta(f)$, wherein
the minimization this time is over all $r$-round protocols for
$f$ with at most $\delta$-error. Let us summarize the 
two key definitions of this section in notation.

For a function 
$f\colon\mathcal{X}\times\mathcal{Y}\to\mathcal{Z}$,
\begin{align}
R_\delta(f) &\defeq 
  \min_{P}\max_{x\in\mathcal{X},y\in\mathcal{Y},r} \abs{\Pi_P(x,y,r)}
  \label{def:randcomm}\\
%
R^r_\delta(f) &\defeq
  \min_{P}\max_{x\in\mathcal{X},y\in\mathcal{Y},r} \abs{\Pi_P(x,y,r)}
  \label{def:randcommr}
\end{align}
where $P$ ranges over all $\delta$-error protocols for $f$ in \autoref{def:randcomm}
and $P$ ranges over all $\delta$-error $r$-round protocols for $f$ in \autoref{def:randcommr}.

\subsection{The $k$-disjointness problem}
In the $k$-disjointness problem (also known as the sparse set
disjointness or small set disjointness), each of the players
receives a subset of $[n]$ of size at most $k$. Call the set
Alice receives $S$ and the set Bob receives $T$. The goal of the
players is to determine whether their sets $S$ and $T$ intersect.
We denote this problem by $\Disj^n_k$.

One useful way to think about $\Disj^n_k$ problem is to associate 
the sets players receive with their characteristic vectors. This way,
Alice and Bob each receive an $n$ bit string with at most $k$ ones
and their goal is to determine if there is a coordinate in which
both of their strings is one. Notice that the characteristic vectors
have a common 1 if and only if their Hamming distance is at most $2k-2$.

\subsection{The $k$-Hamming distance problem}
In the $k$-Hamming distance problem the players get $n$ bit strings,
say $x,y\in\cube$, respectively with the promise that $\Ham(x,y)\le k$.
Their goal is to determine whether $\Ham(x,y)$ under this promise.
Note that by the previous paragraph, the $k$-disjointness problem
is a special case of the $2k$-Hamming distance problem, therefore
up to constant factors, the communication complexity of
$\Disj^n_k$ is upper bounded by $\Ham^n_k$.

% k-disjointness
\include{disj-intro}
\include{disj-upperbound}
\include{disj-elementary}
\include{disj-isoperimetry}
\include{disj-lowerbound}
\input  {disj-notes}

\include{heat-intro}
\include{heat-monotonicity}
\input  {heat-logconvex}
\section{Chapter notes}
\label{sec:heat:notes}

The results of this chapter were obtained in our FOCS 2018 paper \cite{Saglam2018}.


% k-Hamming distance
\include{ham-intro}
\include{ham-lowerbound}
\include{ham-discussion}

\bibliographystyle{plain}
\bibliography{references/main}

\end{document}
