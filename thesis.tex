\documentclass[11pt,proquest]{uwthesis}[2016/11/22]

\usepackage[utf8]{inputenc}
\usepackage{amsmath, amsthm, amssymb}
\usepackage{lmodern}
\usepackage{microtype}
\usepackage[usenames]{xcolor}
\usepackage{hyperref}
\usepackage{aliascnt}
\usepackage{multirow}
\usepackage{mathtools}
\usepackage{bbm}
\usepackage{float}
\usepackage{tikz}
\usepackage{enumerate}
\usepackage{nicefrac}

\include{definitions/theoremlike}
\include{definitions/math}
\include{definitions/fields}
\include{definitions/probability}
\include{definitions/entropy}
\include{definitions/linearalgebra}
\include{definitions/communication}

\setcounter{tocdepth}{1}
\definecolor{Blue}{RGB}{0, 60, 60}
\definecolor{Red}{RGB}{0,35,60}
\hypersetup{
    colorlinks,
    citecolor=Red,
    filecolor=Blue,
    linkcolor=Blue,
    urlcolor=Blue
}
\DeclareMathOperator{\down}{down}
\DeclareMathOperator{\up}{up}
\DeclareMathOperator{\wt}{wt}
\DeclareMathOperator{\EE}{EE}
\DeclareMathOperator{\Match}{Match}

\newcommand{\propref}[1] {%
\hyperref[#1]{Property~(P\ref*{#1})}}


\usetikzlibrary{positioning}

\begin{document}

\prelimpages

\Title  {Settling the complexity of $k$-disjointness
         and the $k$-Hamming distance problems}
\Author {Mert Sağlam}
\Year   {2019}
\Program{Computer Science and Engineering}

\Chair{Shayan Oveis Gharan}{Title of Chair}{Computer Science and Engineering}
\Signature{Paul Beame}
\Signature{Anup Rao}
\Signature{Ioana Dumitriu, GSR}

\copyrightpage
\titlepage  
\setcounter{page}{-1}


% !TeX root = thesis.tex
\abstract{%
Suppose two parties, traditionally called Alice and Bob, are
given respectively the inputs $x\in\mathcal{X}$ and
$y\in\mathcal{Y}$ to a function
$f\colon\mathcal{X}\times\mathcal{Y}\to\mathcal{Z}$ and are
required to compute $f(x,y)$.
Since each party only has one part of the input, they can
compute $f(x,y)$ only if some communication takes place between
them. The communication complexity of a given function is the
minimum amount of communication (in bits) needed to evaluate it
on any input with high probability.

We study the communication complexity of two related problems,
the $k$-Hamming distance and $k$-disjointness and give tight
bounds to both of these problems: The $r$-round communication
complexity of the $k$-disjointness problem is
$\Theta(k\log^{(r)}k)$, whereas a tight $\Omega(k\log k)$ bound
holds for the $k$-Hamming distance problem for any number of
rounds.

The lower bound direction of our first result is obtained by
proving a {\em super-sum} result on computing the OR of $n$
equality problems, which is the first of its kind.
Using our second bound, we settle the complexity of various
property testing problems such as $k$-linearity, which was open
since 2002 or earlier. Our lower bounds are obtained via
information theoretic arguments and along the way we resolve a
question conjectured by Erdős and Simonovits in 1982, which
incidentally was studied even earlier by Blakley and Dixon in
1966.
}


\tableofcontents
\listoffigures
\listoftables
\thispagestyle{plain}
\textpages

%% !TeX root = thesis.tex
\chapter{Introduction}
\label{sec:introduction}

In communication complexity, one tries to understand the 
limitations of computation through communication bottlenecks. 
In computation, communication bottlenecks are everywhere; 
inside the same core, between the cores, inside the same die, package, 
between the memory and the processor, between software components and more. 
These bottlenecks are not specific to the current designs of computers; 
in fact, any physical computer has to deal with similar communication 
issues as dictated by laws of nature.
Communication bottlenecks remain relevant and challenging even higher up in the 
abstraction hierarchy, including theoretical abstractions such as data structures
and algorithms.

In 1979 to study such bottlenecks in a generic way, 
Yao introduced the following abstract model \cite{Yao1979}. 
Two players, called Alice and Bob, are required to evaluate a known 
function $f$ on inputs $(x,y)$, where Alice has only $x$ and Bob has only $y$.
The players communicate by sending messages, that is, bit strings, to each other 
in turns until one of the players is able to deduce $f(x,y)$. The communication
complexity of one such function $f$ is taken to be the minimum number of bits
of communication needed to be able to evaluate it on any input $(x,y)$
with hight probability. Note here that the inputs $x$ and $y$, for instance, 
may correspond to data stored in separate parts of a computational device and 
the communication complexity of the function would then model the data transfer 
between these parts. 

One problem that proved its significance early on in the history of 
communication complexity is set disjointness. 
In the set disjointness problem, the players receive respectively subsets $S, T$ 
of a ground set, say $\set{1,2,\ldots,n}$, and are required to figure out 
if their sets $S$ and $T$ intersect.
The first lower bound for this problem that applies even to randomized protocols
was given by Babai, Frankl and Simon in \cite{BabaiFS1986}. Their lower bound
shows that any randomized protocol for the set disjointness problem over 
the ground set $\set{1,2,\ldots, n}$
requires at least $\Omega(\sqrt{n})$ bits of communication.
On the flip side, the best known protocol 
was the trivial one round protocol wherein one players sends their entire 
input to the other, thereby communicating
$n$ bits and solving problem deterministically.
This gap between the upper bound and the lower bound was closed
in the landmark paper of Kalyanasundaram and Schnitger from 1987,
which showed a tight $\Omega(n)$ bits lower bound. This result was later 
simplified in
influential papers of Razborov \cite{Razborov1992} and Bar-Yossef, Jayram, Kumar and Sivakumar
\cite{Bar-YossefJKS2004}, which led to a whole new understanding of 
communication through the lens of
information theory.

\paragraph{The $k$-disjointness problem.}
To gather a more refined understanding of the set disjointness problem, 
in this work we study a variant of this problem where
 the set sizes are restricted to be at most $k$ for some arbitrary parameter $k\le n$. 
Needless to say, we can always take $k$ large enough to 
recover the unrestricted version of the problem.
This restriction, while may seem dull initially given the earlier remark, 
 allows us to
uncover an entirely new and surprising aspects of the set disjointness problem: 
that the complexity of the problem exhibits a rounds versus communication trade-off,
that vanishes very quickly with increasing number of rounds.

In fact, we are not the first to study this restriction 
on the size of the sets.
In 1990s Håstad and Wigderson studied this variant of the problem,
henceforth called the $k$-disjointness problem,
and gave a protocol which communicates only $O(k)$ bits over $O(\log k)$ rounds,
solving the problem with constant probability \cite{HastadW1990, ParnafesIRWA1997, HastadW2007}.
In the first part of this thesis, we improve this protocol 
to run in just $\log^*k$ rounds, while we simultaneously reduce the error probability to
exponentially small in $k$. In fact, for an arbitrary integer $r\le \log^* k$, this protocol 
can be run in $r$ rounds with communication cost $O(k\log^{(r)}k)$ bits in total,
 and error probability well below polynomial for $r\ge 2$.
More importantly, we show that this improvement is final: we prove that any $r$-round 
set disjointness protocol with even constant error probability requires at least one message of size
$\Omega(k\log^{(r)}k)$ bits. This proof works even when $r$ is a function of $k$ such as $r=\log^* k$, 
settling the complexity of the problem entirely.
 
\paragraph{The $k$-Hamming distance problem.} 
An alternative way of viewing the $k$-disjointness problem is through the characteristic 
vectors of sets $S, T$ given to the players.
We may think of each player as having received a $k$-sparse bit vector vector and the goal 
of the players is to understand whether the Hamming distance of their vectors is less than $2k$.
Notice that the Hamming distance of these vectors is less than $2k$ if and only if the 
corresponding set $S$ and $T$ intersect.
An important generalization of this problem is the {\em $k$-Hamming distance problem} 
wherein the players are given bit vectors, respectively $x$ and $y$,
within Hamming distance $k$ and the goal of the players is to compute the Hamming distance 
between $x$ and $y$ exactly. Notice that compared to the $k$-disjointness problem here 
we removed the $k$-sparsity promise while keeping the closeness promise intact.

For the $k$-disjointness and the $k$-Hamming distance problems 
there is a simple 1-round $O(k \log k)$ bits protocol.  As the work of 
Håstad and Wigderson and our work uncovered, for the $k$-disjointness problem 
the complexity goes to $k$ very quickly as we allow more and more rounds.
Interestingly, no such protocol could be found for the $k$-Hamming distance problem;
not even a single improvement over the easy 1-round protocol.
This lack of progress became all the more pressing when in 2011 
Blais Brody and Matulef \cite{BlaisBM2012} showed a connection between
the $k$-Hamming distance problem and a host of important property testing problems.

\paragraph{The property testing model.}
In the {\em property testing model} one is given black-box query access 
to an otherwise unknown function $g$, with the task of determining
whether the function is inside a certain class or differs from any function inside 
the class in at least $\eps$-fraction of the possible outputs.
A function $f\colon\field_2^n\to\field_2$ is called $k$-linear if $f(x)=\iprod{x}{w}$
for some $w\in\field_2^n$ with $\lone{w}\le k$. In other words, a $k$-linear function 
outputs the sum of at most $k$ bits of the input in mod 2. 
In \cite{Blais2009}, Blais gave a $O(\log k)$ round tester with $O(k\log k)$ 
queries for $k$-linearity. This result was improved in \cite{BuhrmanGMW2012} by Buhrman, Garcia-Soriano, Matsliah and
de Wolf, who gave a nonadaptive $O(k\log k)$ bits tester. The progress stuck here however  and the gap between the $O(k\log k)$ query upper bound and the $\Omega(k)$ lower bound remained unresolved for several years.

Multiple attempts have been made to show an $\Omega(k\log k)$ lower bound to the $k$-Hamming distance problem
to close this gap. In \cite{BuhrmanGMW2012}, \cite{DasguptaKS2012}, and \cite{Patrascu2009} an $\Omega(k\log k)$ lower bound was shown, but only for protocols
having 1 round. These lower bound do not rule however the possibility that the complexity of the $k$-Hamming distance problem
decays to, say $O(k)$, with multiple rounds, just like the related $k$-disjointness problem.
In fact, all 3 of these bounds apply to both $k$-disjointness and $k$-Hamming distance problem alike, 
therefore are incapable of separating them.
Even some other attempts have been made to prove an 
$O(k)$ upper bound to the $k$-Hamming distance problem (certainly the present author).
One daunting challenge in proving a sharp lower bound the the $k$-Hamming distance problem
lies in the fact that one of the most powerful proof techniques in communication complexity, the corruption
bound, is {\em unable} prove any bound beyond $\Omega(k\log 1/\delta)$ where $\delta$ is the error probability of the protocol.
The corruption bound fails due to the existence of large combinatorial rectangles in the communication matrix of the $k$-Hamming distance
function.

In 2014, Blais, Brody and Ghazi \cite{BlaisBG2014} showed an 
$\Omega(k\log 1/\delta)$ lower bound to the $k$-Hamming distance problem 
using an information theoretic argument. While this may seem like a 
small improvement over the $\Omega(k)$ lower bound that follows from the 
work of \cite{KalyanasundaramS1992} as one usually takes $\delta$ to be a constant, 
the importance of this result lies in that it is the first formal evidence that 
$k$-Hamming distance is harder than the $k$-disjointness problem. 
Recall that our upper bound \cite{SaglamT2013} solves the $k$-disjointness problem 
with $O(k)$ bits of communication and exponentially small error probability in $k$, 
so no bound of the form $\Omega(k\log 1/\delta)$ can be true for $k$-disjointness. 
We remark that the lower bound \cite{BlaisBG2014} does not go further than $\Omega(k\log 1/\delta)$ 
not only due to the limitation of the corruption method alluded to in the earlier paragraph, 
but actually the problem they study, the $\mathrm{OR}\circ\mathrm{1vs3}$ problem, to establish their $k$-Hamming distance lower bound
admits an actual $O(k \log 1/\delta)$ upper bound. Notice this is a much stronger statement than the existence of larger rectangles.

The usual decomposition technique employed in randomized communication complexity of 
considering a special set of possible inputs which can be interpreted as calculating a 
composed function, say $f\circ g^k$, with a simple $f$ leads to the $\mathrm{OR}\circ\mathrm{1vs3}$
function of \cite{BlaisBG2014}. As mentioned earlier, this restriction of the problem admits
and $O(k\log 1/\delta)$ bits upper bound, so is no help to prove an $\Omega(k\log k)$ lower bound.
In the same work \cite{BlaisBG2014}, the composed function 
$\mathrm{MAJ}\circ\mathrm{1vs3}$ was proposed as a candidate to obtain an $\Omega(k\log k)$ lower bound.
However working with the majority function appears to be difficult and it is unclear if the decomposition
bought us any comfort in proving the optimal lower bound at all.

If one does not go through the decomposition route, the only standard technique 
which appears to be able to provide a global analysis to the $k$-Hamming distance problem is the spectral norm bound.
While it is possible to show an $\Omega(k\log\frac{\log1/\delta}{k} + k)$ lower bound to the log-spectral-norm
of the corresponding $k$-threshold function via a duality argument,  anything beyond this bound proved difficult.
Through some experimentation with linear program solvers, we believe this lower bound is tight for the log-spectral-norm of the $k$-threshold function.
In fact, in an upcoming work with de Wolf and Hoyer, we show there are quantum communication protocols for the $k$-Hamming distance problem
with just $O(k)$ qubits of communication and exponentially small error probability. This implies that the log-spectral-norm of $k$-threshold function is $O(k)$, even though this upper bound does not match the log-spectral-norm lower bound in the $\delta$ parameter. 

Given that various powerful lower bound techniques in communication complexity are all stuck at $\Omega(k)$, 
and all upper bound attempt are stuck at $O(k\log k)$,
it was far from clear what the correct bound for the communication complexity of the $k$-Hamming distance should be.
In our recent work \cite{Saglam2018}, we managed to find a connection between the $k$-Hamming distance problem and a 
very natural statement 
about how heat behaves in time. It turns out, similar question were considered earlier in the literature.
In 1966, Blakley and Dixon \cite{BlakleyD1966} studied a special case of  our main inequality, defined and proven in \autoref{sec:heat} of this thesis.
In 1982, Erdős and Simonovits \cite{ErdosS1982}
studied a slightly more restrictive case of the inequality of \cite{BlakleyD1966} and posed it as a conjecture. 
In \autoref{sec:heat} we prove that the inequality of Blakley and Dixon holds true even in a more general form, which in turn affirmatively answers the conjecture of Erdős and Simonovits. Furthermore in the second part of \autoref{sec:heat}, we prove a substantial refinement of this inequality, which in essence argues that the probability mass contained in a certain region of Markov chain must be nearly convex
as a function of time. In \autoref{sec:ham} we prove that the aforementioned inequality about Markov chains is precisely what is needed
to prove an $\Omega(k\log k)$ lower bound to the $k$-Hamming distance problem, thereby establishing the complexity of 
$k$-Hamming distance problem and the tight bounds for the property testing of $k$-linearity.

\section{Our techniques.}
\label{sec:intro:tech}

An important challenge in proving the inequality of Blakley and Dixon \cite{BlakleyD1966} is the following.
Our generalized version of the inequality states that for vectors $u$, $v$ with nonnegative entries and a symmetric matrix $S$ with nonnegative entries,
we have $\iprod{v}{S^{k}u}^{1/k}\ge\iprod{v}{S^{t}u}^{1/t}$ for $k>t$ integers of the same parity. We remark the nonnegativity requirement on $u,v$ is crucial
and the inequality fails for general $u,v$. In fact, one way to deduce this is to take our recent $O(k)$ qubits quantum communication protocol for the $k$-Hamming distance problem and obtain $u,v$ from the log-spectral-norm relaxation of the protocol. 
The key technical challenge in proving this inequality seems to be devising a proof that is sensitive to $u,v$ being in the the positive orthant.
In \cite{BlakleyR1965} Blakley and Roy showed that the special case $\iprod{v}{S^{k}u}^{1/k} \ge \iprod{v}{Su}$ is true by taking a geometric view and working with the positive orthant directly. For $t>1$, working with the positive orthant directly appears to be very difficulty. 
Instead, we break apart from the geometric view and into a probabilistic perspective to make positivity an inherent part of the technique.
We cast this inequality as a probabilistic statement and analyze it through the lens of information theory by viewing it as a success probability of discrete time stochastic process.

\section{Future directions}
\label{sec:intro:future}

A surprising part of our  $\Omega(k\log k)$ lower bound for the $k$-Hamming distance problem is that the proof does not use at all the tensor structure of the hypercube. The only two properties of the hypercube that we use are 1) that it is an undirected graph
2) that if we perform the standard discrete time random walk from a vertex $u$, at time $t\ll\sqrt{n}$, we end up with a vertex $v$ satisfying $\Ham(u,v) = t$ with probability, say 0.9. 

We believe a similar proof the Gap Hamming Distance problem may be possible. Current lower bound proofs for the Gap Hamming Distance depend heavily on the product structure of the hypercube through concentration of measure phenomenon. We discuss this possibility in \autoref{conj:continuouslogconv}.

\section{Outline}
In this thesis we study the communication complexity of the 
$k$-disjointness and $k$-Hamming distance problems. 
In \autoref{sec:ham}, we give our $k$-disjointness protocol,
and show a matching lower bound after we take a brief detour 
into combinatorics to develop our isoperimetric inequality used in the lower bound proof.

Our lower bound for the $k$-Hamming distance, given in \autoref{sec:ham},
requires us to understand how heat behaves in discrete time. We develop this
theory in \autoref{sec:heat}. The notation we use and some formal preliminaries
are given in \autoref{sec:prelim}.
\chapter{Introduction}
\newpage\newpage\newpage\newpage
\section{Outline}
In this thesis we study the communication complexity of the 
$k$-disjointness and $k$-Hamming distance problems. 
In \autoref{sec:ham}, we give our $k$-disjointness protocol,
and show a matching lower bound after we take a brief detour 
into combinatorics to develop our isoperimetric inequality used in the lower bound proof.

Our lower bound for the $k$-Hamming distance, given in \autoref{sec:ham},
requires us to understand how heat behaves in discrete time. We develop this
theory in \autoref{sec:heat}. The notation we use and some formal preliminaries
are given in \autoref{sec:prelim}.

\newpage\newpage
% !TeX root = thesis.tex
\chapter{Preliminaries}
\label{sec:prelim}

In this chapter we list the notation we use, conventions we adopt
and formally and define formally and in more detail 
the computational models that we work with in this thesis. 

\section{Notation}
\label{sec:prelim:notation}

We denote by $[n]$ the set $\set{1,2,\ldots,n}$. 
Throughout this thesis, we
take $\exp$ and $\log$ functions to the base 2.
For exponentials and logarithms in other bases
such as $b$, we write $\exp_b$ and $\log_b$. 
We adopt the convention $\ln \defeq \log_e$ where $e=2.718\ldots$
is the limiting value of $(1+1/n)^n$ as $n\to \infty$.
We will also use
the iterated versions of these functions:
\begin{align*}
  \log^{(0)}x&\defeq x, 
             & 
  \exp^{(0)}x&\defeq x,\\
  \log^{(r)}x&\defeq \log\nparen{\log^{(r-1)}x}, 
             & 
  \exp^{(r)}x&\defeq \exp\nparen{\exp^{(r-1)}x}
  \quad\text{for $r\geq 1$}.
\end{align*}
Moreover we define the $\log^* x$ to be the smallest integer $r$ 
for which $\log^{(r)} x<2$. For instance, we have $\log^* 16 = 3$ and
$\log^* 2^{16} = 4$.
The function $\log^*$ is conventionally called the 
{\em iterated logarithm}, which we adopt. 
To differentiate, we call the function $\log^{(r)}$ the 
{\em $r$-iterated logarithm}.

\section{Random variables and distributions}
\label{sec:prelim:rand}
Let $\Omega$ be a countable set. 
For a function $\mu\colon\Omega\to\realspos$ and a set 
$\Psi\subseteq\Omega$, we use the shorthand
\begin{align*}
\mu(\Psi) \defeq \sum_{x\in\Psi}\mu(x).
\end{align*}
A function $\mu\colon\Omega\to\realspos$
is said to be a distribution on $\Omega$ if
$\mu(\Omega) = 1$ and a subdistribution
if $\mu(\Omega) \le 1$. For a function
$\mu$ on $\Omega$, we define
\begin{align*}
\supp(\mu)\defeq \setbuilder{x\in \Omega}{\mu(x)>0}.
\end{align*}
For two distributions $\mu\colon\Omega_1\to\realspos$
and $\nu\colon\Omega_2\to\realspos$, let us denote by
$\mu\nu$ the distribution on $\Omega_1\times\Omega_2$
given by $(\mu\nu)(x_1,x_2) = \mu(x_1)\nu(x_2)$.

For a discrete random variable $X$, we denote by
$\dist(X)$ the distribution function of $X$ and
we define $\supp(X)\defeq\supp(\dist(X))$. 
If $X$ is so that $\dist(X)\colon\Omega\to\realspos$,
then we say that $X$ has sample space $\Omega$.
Two 
random variables $X$ and $Y$ are said to be 
independent if $\dist(XY) =\dist(X)\dist(Y)$.

\begin{lemma}[Jensen \cite{Jensen1906}, Formula (5)]
\label{lem:jensen}
Let $X$ be a real-valued random variable and
$f$ be a convex function. We have
$\E\sparen{f(X)} \ge f(\E[X])$. When
$f$ is strictly convex, the inequality holds
with equality if and only if $X$ is constant
with probability 1.
\end{lemma}

\section{Facts from information theory}
\label{sec:prelim:info}
In this section we review the definitions and 
facts we use from information theory.
Let $\mu$ and $\nu$ be two 
nonnegative functions on $\Omega$.
The Kullback-Leibler divergence \cite{Wald1945, KullbackL1951}
of $\mu$ from $\nu$,
denoted $\kldiv{\mu}{\nu}$, is defined by
%
\begin{align}
\label{eq:kldiv}
  \kldiv{\mu}{\nu} \defeq \sum_{x\in \Omega}
      \mu(x)\log\frac{\mu(x)}{\nu(x)}\,.
\end{align}
%
Here, if $\mu(x)=0$ for some $x$, then its contribution
to the summation is taken as 0, even when $\nu(x)=0$.
The divergence is undefined if there is an
$x\in \Omega$ such that $\mu(x)>0$ and
$\nu(x)=0$.
It can be shown that if the related series 
converges for the right hand side of \autoref{eq:kldiv},
it converges absolutely, 
which justifies leaving the summation order unspecified.
A fundamental property of 
$\kldiv{\cdot}{\cdot}$ is that
the divergence of a distribution from a subdistribution 
is always nonnegative.
\begin{lemma}[Gibbs \cite{Gibbs1902}, Theorem VIII]
\label{lem:gibbs}
Let $\mu,\nu\colon\Omega\to\reals$ be such that $\mu$ is a
distribution and $\nu$ is a subdistribution.
We have $\kldiv{\mu}{\nu}\ge 0$ with equality if and
only if $\mu=\nu$.
\end{lemma}
\begin{lemma}[Kullback and Leibler \cite{KullbackL1951},
Lemma 3.2]
\label{lem:klcond}
Let $\mu,\nu\colon\Omega\to\realspos$ be so that
$\mu$ is a distribution on $\Omega$ and 
$\supp(\mu) = \Psi\subseteq\Omega$. We have
\begin{align*}
\kldiv{\mu}{\nu}\ge -\log \nu(\Psi)
\end{align*}
with equality if and only if $\mu(x) = \nu(x)/\nu(\Psi)$ for
$x\in\Psi$ and $\mu(x) = 0$ for $x\notin\Psi$.
\end{lemma}
\begin{proof}
By \autoref{eq:kldiv} we write
\begin{align*}
\kldiv{\mu}{\nu} &= -\sum_{x\in \Psi}
    \mu(x)\log\frac{\nu(x)}{\mu(x)}
  \ge -\log\sum_{x\in \Psi}\nu(x) =-\log \nu(\Psi)\,,
\end{align*}
where the inequality follows from
\autoref{lem:jensen} and concavity of
$z\mapsto\log z$ on $\realspos$. 
If $\mu(x) = \nu(x)/\nu(\Psi)$ for $x\in\Psi$, we have 
$\kldiv{\mu}{\nu}=-\log \nu(\Psi)$ by direct computation.
Otherwise $\kldiv{\mu}{\nu}>-\log\nu(\Psi)$ by strict 
concavity of $z\mapsto\log z$.
\end{proof}

We extend the divergence notation
$\kldiv{\cdot}{\cdot}$ to apply to random
variables as follows. Let $X,Y$ be discrete random
variables on the same sample space $\Omega$. Define
\begin{align}
\label{eq:klrv}
\kldiv{X}{Y}\defeq \kldiv{\dist(X)}{\dist(Y)}.
\end{align}
%
With this notation in hand, we are ready to define the 
conditional divergence. Let $X_1X_2$ and $Y_1Y_2$ be 
random variables defined on the sample space $\Omega_1\times \Omega_2$.
The divergence of $X_1\emid X_2$ from $Y_1\emid Y_2$ is
defined by 
\begin{align}
\kldiv{X_1\emid X_2}{Y_1\emid Y_2}
    \defeq \E_{x_2\sim X_2}
        \kldiv{X_1\emid X_2=x_2}{Y_1\emid Y_2 = x_2}.
\end{align}
Here, for each $x_2\in\supp(X_2)$, 
$X_1\emid X_2=x_2$ and $Y_1\emid Y_2 = x_2$ 
are random variables on the sample space
$\Omega_1$ obtained from, respectively
$X_1X_2$ and $Y_1Y_2$, by conditioning on the
second coordinate equaling $x_2$.
% !TeX root = thesis.tex
\begin{lemma}[e.g., \cite{CoverT2006}]
\label{lem:klchain}
Let $X_1X_2$ and $Y_1Y_2$ be random variables, both on the
sample space $\Omega_1\times \Omega_2$. We have
\begin{align*}
\kldiv{X_1X_2}{Y_1Y_2} = 
    \kldiv{X_1}{Y_1} + \kldiv{X_2\emid X_1}{Y_2\emid Y_1}.
\end{align*}
\end{lemma}

\begin{proof}
Let $\mu,\nu\colon\Omega_1\times \Omega_2\to \realspos$
be the distributions of respectively $X_1X_2$ and $Y_1Y_2$.
Using the shorthands
$\mu(\Omega_1,x_2)\defeq\sum_{x_1\in\Omega_1}\mu(x_1,x_2)$
and 
$\nu(\Omega_1,x_2)\defeq\sum_{x_1\in\Omega_1}\nu(x_1,x_2)$,
we write
\begin{align*}
  \kldiv{X_1\emid X_2}{Y_1\emid Y_2} &=
     \sum_{x_2\in\Omega_2}\mu(\Omega_1,x_2)
       \sum_{x_1\in\Omega_1}\frac{\mu(x_1,x_2)}
           {\mu(\Omega_1,x_2)}
       \log\frac{\mu(x_1,x_2)\nu(\Omega_1,x_2)}
                {\nu(x_1,x_2)\mu(\Omega_1,x_2)}\\
  &= \sum_{x_1,x_2}\mu(x_1,x_2)
       \log\frac{\mu(x_1,x_2)}{\nu(x_1,x_2)} +
     \sum_{x_2}
       \mu(\Omega_1,x_2)\log\frac{\nu(\Omega_1,x_2)}
           {\mu(\Omega_1,x_2)}
\end{align*}
by splitting the terms inside the logarithm. 
Using \autoref{eq:kldiv} together with \autoref{eq:klrv}
 we conclude
\begin{align*}
\kldiv{X_1\emid X_2}{Y_1\emid Y_2}  &= \kldiv{X_1X_2}{Y_1Y_2} - \kldiv{X_2}{Y_2}.
\end{align*}
Rearranging, we obtain the statement of the lemma.
\end{proof}

The next lemma establishes that the Kullback-Leibler divergence
is jointly convex in its parameters. This fact is also called 
the data processing inequality for Kullback-Leibler divergence. 
\begin{lemma}
\label{lem:klconvex}
Let $\mu_1,\mu_2\colon\Omega_1\to\realspos$ be distributions
supported on $\Omega_1$ and let 
$\nu_1,\nu_2\colon\Omega_2\to\realspos$ be distributions
supported on $\Omega_2$. For any $a\in[0,1]$, we have 
\begin{align*}
\kldiv{a\mu_1 + (1-a)\mu_2}{a\nu_1 + (1-a)\nu_2}
  \le a\kldiv{\mu_1}{\nu_1} + (1-a)\kldiv{\mu_2}{\nu_2}
\end{align*}
\end{lemma}

Further for two reals $p,q\in[0,1]$, we use the shorthand notation 
$\kldivb{p}{q}$ to denote the divergence of the random variable $P$
from $Q$ where $P,Q$ are Bernoulli random variables with expectation
respectively $p$ and $q$.


\subsection{Mutual information and Shannon entropy}
\label{sec:prelim:muti}
Let $X$ and $Y$ be jointly distributed random variables.
The 
mutual information of $X$ and $Y$, 
denoted $\muti{X}{Y}$, is defined as
\begin{align}
\muti{X}{Y} \defeq \kldiv{\dist(X,Y)}{\dist(X)\dist(Y)}.
\label{eq:muti-def}
\end{align}
The mutual information of a random variable 
with itself, i.e., the quantity $\muti{X}{X}$ is 
called the Shannon entropy of $X$ and denoted by $\ent{X}$.
If $X\in[t]^n$ and $L\subseteq[n]$, then the projection of $X$
to the coordinates in $L$ is denoted by $X_L$. Namely, $X_L$ is
obtained from $X=(X_1,\ldots,X_n)$ by keeping only the
coordinates $X_i$ with $i\in L$. The following lemma of Chung et
al.~\cite{ChungGFS1986} relates the entropy of a variable to the
entropy of its projections.
\begin{lemma}[Chung et al.\ \cite{ChungGFS1986}]
\label{lem:ent-subset}
Let $\supp(X)\subseteq[t]^n$. We have
$\frac{l}{n}\ent{X} \leq \E_L[\ent{X_L}]$, where the expectation is taken for
a uniform random $l$-subset $L$ of $[n]$.
\end{lemma}

\section{Concentration bounds}
Let $X_1,\ldots, X_n$ be random
variables such that $\E[X_i] = \epsilon$ for 
some $0\leq \epsilon\leq 1$. Define $X = 
X_1+\cdots + X_n$. By linearity of 
expectation we have $\E[X] = \epsilon n$.
If each $X_i$ is chosen independently,
due  to the concentration of measure phenomenon,
it is well understood that $X$ takes values 
close to its expectation with very high probability.
The classical results of Chernoff 
\cite{Chernoff1952} and Hoeffding 
\cite{Hoeffding1963} give quantitative and 
intuitive bounds on the deviation probability.
\begin{theorem}[Chernoff \cite{Chernoff1952}]
\label{thm:chernoff}
Let $X=X_1+\cdots+ X_n$, where $X_i$ for 
$i\in [n]$ are independent binary random 
variables with expectation $\epsilon$. Then 
for any $\epsilon\leq\gamma\leq 1$ we have 
\begin{align*}
  \Pr\sparen{X\geq\gamma n}
    \leq \exp\nparen{-n\kldivb{\gamma}{\epsilon}}.
\end{align*}
\end{theorem}

\begin{proof}
Let $E$ be the event that $X\ge \gamma n$. 
Let $\mu$ be the Bernoulli distribution that 
equals 1 with probability $\epsilon$. We have
\begin{align*}
-\log\Pr[E]
  &= \kldiv{\dist(X_1\ldots X_n\emid E)}{\mu^n} 
  & \text{(by \autoref{lem:klcond})}\\
%
  &\ge \sum_{i=1}^n \kldiv{\dist(X_i\emid E)}{\mu} 
  &\text{(by \autoref{lem:klchain})}\\
%
  &\geq n \kldivb{\gamma}{\epsilon}
  &\text{(as $\kldivb{\delta}{\epsilon}
    \geq\kldivb{\gamma}{\epsilon}$ for $\epsilon\leq \gamma\leq\delta$)}
\end{align*}
Hence,
$\Pr[E]
  \le \exp\nparen{-n\D_2(\gamma\dmid \epsilon)}$ as required.
\end{proof}


\section{Communication complexity and protocols}
\label{sec:prelim:comm}

\paragraph{The model.} In the two party communication complexity model we have two
parties called respectively Alice and Bob who are required to
evaluate a function
$f\colon\mathcal{X}\times\mathcal{Y}\to\mathcal{Z}$ (known to
both of them) on some input $(x,y)$ where $x$ is revealed to
Alice only and $y$ is revealed to Bob only.

In the randomized variant of this model, which is what we solely
study in this thesis, the players also have access to a shared
random source. Without loss of generality, the random source can
be taken as an infinite read-only string of bits, chosen
uniformly at random. The players, having received their inputs
$x$ and $y$ respectively, and with access to the shared random
source, engage in a dialogue by alternately sending each other
messages in rounds, in order to compute $f(x,y)$. Here, a
message is a bit string of arbitrary length.

\paragraph{Protocols.}
A {\em communication protocol} specifies, for each round, which
player's turn it is to speak and what message should be sent and
whether the protocol terminates with some answer. The protocol
specifies the message to be sent through a function mapping the
random string, the current players input and all the messages
the current player received so far to a bit string which is to be
sent to the other player. This ensures that the message sent by
the players, say Alice for illustration, 
can depend on only information known to her at the
time: her own input, the shared random source and the message
she received in the previous rounds.
Some message are marked as answers; instead of being sent to the
other player, these message are to be announced as the output of
the protocol, after which the protocol terminates. We say that a
communication protocol for a function
$f\colon\mathcal{X}\times\mathcal{Y}\to\mathcal{Z}$ has at most $\delta$-error
if for any input pair $(x,y)\in\mathcal{X}\times\mathcal{Y}$
with probability at least $1-\delta$ the output of the protocol
equals $f(x,y)$, where the probability is over the random
choices that come from the shared random source.

In an $r$-round protocol there are at most $r$ messages
(excluding the output message) sent for any input and any
configuration of the shared random source. 
To illustrate the way the number of messages is counted, consider 
the following protocol. Alice sends a single message to Bob and 
in return Bob replies with another message after which Alice
announces the answer of the protocol. This protocol has two rounds.
The complexity of a protocol is defined to be the the maximum, over all input pairs
$(x,y)\in\mathcal{X}\times\mathcal{Y}$ and all choices of the
random source, of the total number of bits sent by the two
parties.

Let $P$ be a protocol for a function 
$f\colon\mathcal{X}\times\mathcal{Y}$. We denote by $P(x,y,r)$
the output of the protocol on input $x\in\mathcal{X}$, $y\in\mathcal{Y}$
and when the shared random string is fixed to $r$. We denote by $P(x,y)$ the random
variable denoting the output of the protocol on inputs $(x,y)$.
The transcript of the protocol $P$, denoted $\Pi_P(x,y,r)$, 
is the concatenation of all the messages sent by the players 
(excluding the answer of the protocol) on inputs
$x\in\mathcal{X}$, $y\in\mathcal{Y}$ and when the shared random string is $r$.
Likewise, $\Pi_P(x,y)$ denotes the random variable entailing the communication
took place the two players when the random source is not fixed. 
We denote by $\abs{\Pi_P}$ the length of the transcript, in bits.

\paragraph{Communication complexity}
The $\delta$-error randomized communication complexity of a
function $f\colon\mathcal{X}\times\mathcal{Y}\to\mathcal{Z}$,
denoted $R_\delta(f)$, is the minimum over all $\delta$-error
protocols for $f$, of the complexity of the protocol. We also
define the $r$-round $\delta$-error randomized communication
complexity of a function $f$, denoted $R^r_\delta(f)$, wherein
the minimization this time is over all $r$-round protocols for
$f$ with at most $\delta$-error. Let us summarize the 
two key definitions of this section in notation.

For a function 
$f\colon\mathcal{X}\times\mathcal{Y}\to\mathcal{Z}$,
\begin{align}
R_\delta(f) &\defeq 
  \min_{P}\max_{x\in\mathcal{X},y\in\mathcal{Y},r} \abs{\Pi_P(x,y,r)}
  \label{def:randcomm}\\
%
R^r_\delta(f) &\defeq
  \min_{P}\max_{x\in\mathcal{X},y\in\mathcal{Y},r} \abs{\Pi_P(x,y,r)}
  \label{def:randcommr}
\end{align}
where $P$ ranges over all $\delta$-error protocols for $f$ in \autoref{def:randcomm}
and $P$ ranges over all $\delta$-error $r$-round protocols for $f$ in \autoref{def:randcommr}.

\subsection{The $k$-disjointness problem}
In the $k$-disjointness problem (also known as the sparse set
disjointness or small set disjointness), each of the players
receives a subset of $[n]$ of size at most $k$. Call the set
Alice receives $S$ and the set Bob receives $T$. The goal of the
players is to determine whether their sets $S$ and $T$ intersect.
We denote this problem by $\Disj^n_k$.

One useful way to think about $\Disj^n_k$ problem is to associate 
the sets players receive with their characteristic vectors. This way,
Alice and Bob each receive an $n$ bit string with at most $k$ ones
and their goal is to determine if there is a coordinate in which
both of their strings is one. Notice that the characteristic vectors
have a common 1 if and only if their Hamming distance is at most $2k-2$.

\subsection{The $k$-Hamming distance problem}
In the $k$-Hamming distance problem the players get $n$ bit strings,
say $x,y\in\cube$, respectively with the promise that $\Ham(x,y)\le k$.
Their goal is to determine whether $\Ham(x,y)$ under this promise.
Note that by the previous paragraph, the $k$-disjointness problem
is a special case of the $2k$-Hamming distance problem, therefore
up to constant factors, the communication complexity of
$\Disj^n_k$ is upper bounded by $\Ham^n_k$.

% k-disjointness
% !TeX root = thesis.tex
\chapter{The $k$-disjointness problem}
\label{sec:ham}

\label{sec:ham:intro}
In the set disjointness problem two players, traditionally
called Alice and Bob, receive a subset of
$[n]\defeq\set{1,\ldots,n}$ each and the goal of the players is
to determine whether their sets intersects or not. Since each
player knows only one of the sets, this goal is achievable only
if some communication takes place between the players. The
communication complexity of the set disjointness problem is then
defined to be the minimum amount of communication needed by the
players to determine whether their sets intersect, with high
probability, for any possible sets they may receive. 

Set disjointness is perhaps the most studied problem in
communication complexity. In the most common version
the players receive subsets of $[n]$ with no restriction
on the sizes of the sets. The primary question is
whether the players can significantly improve on the trivial
deterministic protocol, wherein the first player sends the
entire input to the other player, thereby communicating $n$
bits. The first lower bound on the randomized complexity of this
problem was given in \cite{BabaiFS1986} by Babai, Frankl and
Simon, who showed that any $\eps$-error protocol for
disjointness must communicate at least $\Omega(\sqrt{n})$ bits.
The tight bound of $\Omega(n)$-bits was first given by
Kalyanasundaram and Schnitger \cite{KalyanasundaramS1992} and
was later simplified by Razborov \cite{Razborov1992} and
Bar-Yossef, Jayram, Kumar and Sivakumar
\cite{Bar-YossefJKS2004}.

In the sparse set disjointness problem $\Disj_k^n$, the sets
given to the players are guaranteed to have at most $k$
elements. The deterministic communication complexity of this
problem is well understood. The trivial protocol, where Alice
sends her entire input to Bob solves the problem in one round
using $O(k\log(n/k) + k)$ bits. On the other hand, an
$\Omega(k\log(n/k) + k)$ bit total communication lower bound can be
shown even for protocols with an arbitrary number of rounds, say
using the rank method; see \cite{KushilevitzN1997}, page 175.

The randomized complexity of the problem is far more subtle. The
result of Kalyanasundaram and Schnitger cited above immediately 
imply an $\Omega(k)$ lower bound for this version of the problem. 
The folklore $1$-round protocol
solves the problem using $O(k\log k)$ bits, wherein Alice sends
$O(\log k)$-bit hashes for each element of her set. Håstad and
Wigderson \cite{HastadW2007} gave a protocol that matches the
$\Omega(k)$ lower bound mentioned above. Their $O(k)$-bit
randomized protocol runs in $O(\log k)$-rounds and errs with a
small constant probability. In \autoref{sec:upperbound}, we
improve this protocol to run in $\log^*k$ rounds, still with
$O(k)$ total communication, but with exponentially small error
in $k$. We also present an $r$-round protocol for any
$r<\log^*k$ with total communication $O(k\log^{(r)}k)$ and error
probability well below $1/k$; see \autoref{thm:ub}. (Here
$\log^{(r)}$ denotes the iterated logarithm function, see
\autoref{sec:prelim}.) As the exists-equal problem with parameters
$t$ and $n$ (see below) is a special case of $\DISJ_n^{tn}$, our
lower bounds for the exists-equal problem (see below) show that
complexity of this algorithm is optimal for any number
$r\le\log^*k$ of rounds, even if we allow the much larger error
probability of $1/3$. Buhrman et al.~\cite{BuhrmanGMW2012} and
Woodruff \cite{Woodruff2008} (as presented in \cite{Patrascu2009})
show an $\Omega(k\log k)$ lower bound for $1$-round complexity
of $\Disj^n_k$ by a reduction from the indexing problem (this 
reduction was also mentioned in \cite{DasguptaKS12}). We
note that these lower bounds do not apply to the exists-equal
problem, as the input distribution they use generates instances
inherently specific to the disjointness problem; furthermore
this distribution admits a $O(\log k)$ protocol in two rounds.

In the equality problem Alice and Bob receive elements $x$ and
$y$ of a universe $[t]$ and they have to decide whether $x=y$.
We define the two player communication game exists-equal with
parameters $t$ and $n$ as follows. Each player is given an
$n$-dimensional vector from $[t]^n$, namely $x$ and $y$. The
value of the game is one if there exists a coordinate $i\in[n]$
such that $x_i = y_i$, zero otherwise. Clearly, this problem is
the OR of $n$ independent instances of the equality problem.

The direct sum problem in communication complexity is the study
of whether $n$ instances of a problem can be solved using less
than $n$ times the communication required for a single instance
of the problem. This question has been studied extensively for
specific communication problems as well as some class of
problems \cite{ChakrabartiSWY2001, JainRS2003, JainRS2005,
Ben-AroyaRW2008, Gavinsky2008, JainKN2008, HarshaJMR2010, BarakBCR2010}.
The so called direct sum approach is a very powerful tool to
show lower bounds for communication games. In this approach, one
expresses the problem at hand, say as the OR of $n$ instances of
a simpler function and the lower bound is obtained by combining
a lower bound for the simpler problem with a direct sum
argument. For instance, the two-player and multi-player
disjointness bounds of \cite{Bar-YossefJKS2004}, the lopsided set
disjointness bounds \cite{Patrascu2011}, and the lower bounds for
several communication problems that arise from streaming
algorithms \cite{JayramW2009, MagniezMN2010} are a few examples of
results that follow this approach.

Exists-equal with parameters $t$ and $n$ is a special case of
$\DISJ_n^{tn}$, so our protocols in \autoref{sec:disj:upperbound}
solve exists-equal. We show that when $t=\Omega(n)$ these
protocols are optimal, namely every $r$-round randomized
protocol ($r\le\log^*n$) with at most $1/3$ error error
probability needs to send at least one message of size
$\Omega(n\log^{(r)}n)$ bits. See \autoref{thm:main}. Our result
shows that computing the OR of $n$ instances of the equality
problem requires {\em strictly more} than $n$ times the
communication required to solve a single instance of the
equality problem when the number of rounds is smaller than
$\log^* n-O(1)$. Recall that the equality problem admits an
$\epsilon$-error $\log(1/\epsilon)$-bit one-round protocol in
the common random source model.

For $r=1$, our result implies that to compute the OR of $n$
instances of the equality problem with {\em constant
probability}, no protocol can do better than solving each
instance of the equality problem with {\em high probability} so
that the union bound can be applied when taking the OR of the
computed results. The single round case of our lower bound also
generalizes the $\Omega(n\log n)$ lower bound of Molinaro et
al.\ \cite{MolinaroWY2013} for the one round communication
problem, where the players have to find all the answers of $n$
equality problems, outputting an $n$ bit string.

\section{Lower bound techniques}
\label{sec:disj:techniques}

We obtain our general lower bound via a round elimination
argument. In such an argument one assumes the existence of a
protocol $P$ that solves a communication problem, say $f$, in
$r$ rounds. By suitably modifying the inner working of $P$, one
obtains another protocol $P'$ with $r-1$ rounds, which typically
solves smaller instances of $f$ or has larger error than $P$.
Iterating this process, one obtains a protocol with zero rounds.
If the protocol we obtain solves non-trivial instances of $f$
with good probability, we conclude that we have arrived at a
contradiction, therefore the protocol we started with, $P$,
cannot exist. Although round elimination arguments have been
used for a long time, our round elimination lemma is the first
to prove a {\em super-linear} communication lower bound in the
number of primitive problems involved, obtaining which requires
new and interesting ideas.

At the heart of the general round elimination lemma is a new
isoperimetric inequality on the discrete cube $[t]^n$ equipped
with the Hamming distance. We present this result,
\autoref{thm:isoperimetry}, in \autoref{sec:disj:isoperimetry}. 
The first isoperimetric inequality on
this metric space was proven by Lindsey in \cite{Lindsey1964},
where the subsets of $[t]^n$ of a certain size with the so
called minimum induced-edge number were characterized. This
result was rediscovered in \cite{KleitmanKR1971} and
\cite{Clements1971} as well. See \cite{AzizogluO2003} for a
generalization of this inequality to universes which are
$n$-dimensional boxes with arbitrary side lengths. In
\cite{BollobasL1991}, Bollobás et al.\ study isoperimetric
inequalities on $[t]^n$ endowed with the $\ell_1$ distance. For
the purposes of our proof we need to find sets $S$ that minimize
a substantially more complicated measure. This measure also
captures how spread out $S$ is and can be described roughly as
the average over points $x\in[t]^n$ of the logarithm of the
number of points in the intersection of $S$ and a Hamming ball
around $x$.

\section{Related work}
\label{sec:ham:related}
In \cite{MiltersenNSW1998}, a round
elimination lemma was given, which applies to a class of
problems with certain self-reducibility properties. The lemma is
then is used to get lower bounds for various problems including
the greater-than and the predecessor problems. This result was
later tightened in \cite{SenV2003} to get better bounds for the
aforementioned problems. Different round elimination arguments
were also used in \cite{KarchmerW1990, HalstenbergR1988,
NisanW1993,Miltersen1994, DurisGS1987,BeameF2001} for various
communication complexity lower bounds and most recently in
\cite{BrodyC2009} and \cite{BrodyCRVW2010} for obtaining lower
bounds for the gapped Hamming distance problem.

In parallel and independent of the present form of this paper
Brody et al.\ \cite{BrodyCK2012} have also established an
$\Omega(n\log^{(r)}n)$ lower bound for the $r$-round
communication complexity of the exists-equal problem with
parameter $n$. Their result applies for protocols with a
polynomially small error probability like $1/n$. This stronger
assumption on the protocol allows for simpler proof techniques,
namely the information complexity based direct sum technique
developed in several papers including
\cite{Ablayev1996,ChakrabartiSWY2001}, but it is not enough to
create an example where solving the OR of $n$ communication
problems requires more than $n$ times the communication of
solving a single instance. Indeed, even in the shared random
source model one needs $\log n$ bits of communication
(independent of the number of rounds) to achieve $1/n$ error in
a single equality problem.

\section{Structure of the present chapter}
\label{sec:disj:map}

The general round elimination presented in
\autoref{sec:disj:lowerbound} is technically involved, but the lower bound
on the one-round protocols can also be obtained in a more
elementary way. As the one round case exhibits the most dramatic
super-linear increase in the communication cost and also
generalizes the lower bound in \cite{MolinaroWY2013}, we include
this combinatorial argument separately in
\autoref{sec:elementary}.

We start in \autoref{sec:disj:upperbound} with our protocols for the
sparse set disjointness. Note that the exists-equal problem is a
special case of sparse set disjointness, so our protocols work
also for the exists-equal problem. In the rest of the paper we
establish matching lower bounds showing that the complexity of
our protocols are within a constant factor to optimal for both
the exists-equal and the sparse set disjointness problems, and
for any number of rounds. In \autoref{sec:elementary} we give an
elementary proof for the case of single round protocols. In
\autoref{sec:disj:isoperimetry} we develop our isoperimetric
inequality and in \autoref{sec:disj:lowerbound} we use it in our
round elimination proof to get the lower bound for multiple
round protocols. Finally in \autoref{sec:disj:discussion} we point
toward possible extensions of the result of this chapter.



% !TeX root = thesis.tex
\section{The upper bound for $\Disj^n_k$}
\label{sec:disj:upperbound}

Recall that in the communication problem $\DISJ_k^m$, each of
the two players is given a subset of $[m]$ of size at most $k$
and they communicate in order to determine whether their sets
are disjoint or not. In early 1990s, Håstad and Wigderson
\cite{ParnafesIRWA1997,HastadW2007, HastadW1990} discovered a
randomized protocol that solves the $\Disj^m_k$ problem with 
$O(k)$ bits of communication and has constant one-sided error 
probability. The protocol takes $O(\log k)$ rounds. 
Let us briefly review this protocol as this
is the starting point of our protocol.

\subsection{The Håstad Wigderson protocol}
Let $S,T\subseteq[m]$ be the inputs of Alice and Bob. Observe
that if they find a set $Z$ satisfying $S\subseteq Z\subseteq
[m]$, then Bob can replace his input $T$ with $T'=T\cap Z$ as
$T'\cap S=T\cap S$. The main observation is that if $S$ and $T$
are disjoint, then a random set $Z\supseteq S$ will intersect
$T$ in a uniform random subset, so one can expect
$|T'|\approx|T|/2$. In the H\aa stad-Wigderson protocol the
players alternate in finding a random set that contains the
current input of one of them, effectively halving the other
player's input. If in this process the input of one of the
players becomes empty, they know the original inputs were
disjoint. If, however, the sizes of their inputs do not show the
expected exponential decrease in time, then they declare that
their inputs intersect. This introduces a small one sided error.
Note that one of the two outcomes happens in $O(\log k)$ rounds.
An important observation is that Alice can describe a random set
$Z\supseteq S$ to Bob using an expected $O(|S|)$ bits by making
use of the joint random source. This makes the total
communication $O(k)$.

\subsection{Our protocol}
\label{sec:disj:upperbound:our}
In our protocol proving the next theorem, we do almost the same,
but we choose the random sets $Z\supseteq S$ not uniformly, but
from a biased distribution favoring ever smaller sets. This
makes the size of the input sets of the players decrease much
more rapidly, but describing the random set $Z$ to the other
player becomes more costly. By carefully balancing the
parameters we optimize for the total communication given any
number of rounds. When the number of rounds reaches
$\log^*k-O(1)$ the communication reaches its minimum of $O(k)$
and the error becomes exponentially small.

\begin{theorem}\label{thm:ub}
For any $r\leq \log^*k$, there is an $r$-round probabilistic
protocol for $\DISJ^m_k$ with $O(k\log^{(r)}k)$ bits total
communication. There is no error for intersecting input sets,
and the probability of error for disjoint sets can be made
$O(1/\exp^{(r)}(c\log^{(r)} k)+ \exp(-\sqrt k))\ll 1/k$ for any
constant $c > 1$.

For $r=\log^*k-O(1)$ rounds this means an $O(k)$-bit protocol
with error probability $O(\exp(-\sqrt k))$.
\end{theorem}

\begin{proof}
We start with the description of the protocol. Let $S_0$ and
$S_1$ be the input sets of Alice and Bob, respectively. For
$1\le i\le r$, $i$ even Alice sends a message describing a set
$Z_i\supset S_i$ based on her ``current input'' $S_i$ and Bob
updates his ``current input'' $S_{i-1}$ to $S_{i+1}\defeq
S_{i-1}\cap Z_i$. In odd numbered rounds the same happens with
the role of Alice and Bob reversed. We depart from the H\aa
stad-Wigderson protocol in the way we choose the sets $Z_i$:
Using the shared random source the players generate $l_i$ random
subsets of $[m]$ containing each element of $[m]$ independently
and with probability $p_i$. We will set these parameters later.
The set $Z_i$ is chosen to be the first such set containing
$S_i$. Alice or Bob (depending on the parity of $i$) sends the
index of this set or ends the protocol by sending a special
error signal if none of the generated sets contain $S_i$. The
protocol ends with declaring the inputs disjoint if the error
signal is never sent and we have $S_{r+1}=\emptyset$. In all
other cases the protocol ends with declaring ``not disjoint''.

This finishes the description of the protocol except for the
setting of the parameters. Note that the error of the protocol
is one-sided: $S_0\cap S_1=S_i\cap S_{i+1}$ for $i\le r$, so
intersecting inputs cannot yield $S_{r+1}=\emptyset$.

We set the parameters (including $k_i$ used in the analysis) 
as follows:
\begin{align*}
u&=(c+1)\log^{(r)}k,\\
p_i&=\frac1{\exp^{(i)}u}&\hbox{for }1\le i\le r,\\
l_1&=k\exp(ku),\\
l_i&=k2^{k/2^{i-4}}&\hbox{for }2\le i\le r,\\
k_0&=k_1=k,\\
k_i&=\frac k{2^{i-4}\exp^{(i-1)}u}&\hbox{for }2\le i\le r,\\
k_{r+1}&=0.
\end{align*}

The message sent in round $i>1$ has length
$\lceil\log(l_i+1)\rceil<k/2^{i-4}+\log k+1$, thus the total
communication in all rounds but the first is $O(k)$. The length
of the first message is $\lceil\log(l_1+1)\rceil\le ku+\log
k+1$. The total communication is $O(ku)=O(ck\log^{(r)}k)$ as
claimed (recall that $c$ is a constant).

Let us assume the input pair is disjoint. To estimate the error
probability we call round $i$ {\em bad} if an error message is
sent or a set $S_{i+1}$ is created with $|S_{i+1}|>k_{i+1}$. If
no bad round exists we have $S_{r+1}=\emptyset$ and the protocol
makes no error. In what follows we bound the probability that
round $i$ is bad assuming the previous rounds are not bad and
therefore having $|S_j|\le k_j$ for $0\le j\le i$.

The probability that a random set constructed in round $i$
contains $S_i$ is $p_i^{-|S_i|}\ge p_i^{-k_i}$. The probability
that none of the $l_i$ sets contains $S_i$ and thus an error
message is sent is therefore at most
$(1-p_i^{k_i})^{l_i}<e^{-k}$.

If no error occurs in the first bad round $i$, then
$|S_{i+1}|>k_{i+1}$. Note that in this case $S_{i+1}=S_{i-1}\cap
Z_i$ contains each element of $S_{i-1}$ independently and with
probability $p_i$. This is because the choice of $Z_i$ was based
on it containing $S_i$, so it was independent of its
intersection with $S_{i-1}$ (recall that $S_i\cap
S_{i-1}=S_1\cap S_0=\emptyset$). For $i<r$ we use the Chernoff
bound. The expected size of $S_{i+1}$ is $|S_{i-1}|p_i\le
k_{i-1}p_i\le k_{i+1}/2$, thus the probability of
$|S_{i+1}|>k_{i+1}$ is at most $2^{-k_{i+1}/4}$. Finally for the
last round $i=r$ we use the simpler estimate $p_rk_{r-1}\le
k/\exp^{(r)}u$ for $|S_{r+1}|>k_{r+1}=0$.

Summing over all these estimates we obtain the following error
bound for our protocol:
$$\Pr[\hbox{error}]\le re^{-k}+\frac k{\exp^{(r)}u}+
\sum_{i=2}^r2^{-k_i/4}.$$
In case $k_r\ge4\sqrt k$ this error estimate proves the theorem.
In case $k_r<4\sqrt k$ we need to make a minor adjustments in
the setting of our parameters. We take $j$ to be the smallest
value with $k_j<4\sqrt k$, modify the parameters for round $j$
and stop the protocol after this round declaring ``disjoint'' if
$S_{j+1}=\emptyset$ and ``intersecting'' otherwise. The new
parameters for round $j$ are $k'_j=4\sqrt k$, $p'_j=2^{-2\sqrt
k}$, $l'_j=k2^{8k}$. This new setting of the parameters makes
the message in the last round linear in $k$, while both the
probability that round $j-1$ is bad because it makes
$|S_j|>k'_j$, or the probability that round $j$ is bad for any
reason (error message or $S_{j+1}\ne\emptyset$) is $O(2^{-\sqrt
k})$. This finishes the analysis of our protocol.
\end{proof}

% !TeX root = thesis.tex
\section{Lower bound for single round protocols}
\label{sec:elementary}
In this section we give an combinatorial proof that any single
round randomized protocol for the exists-equal problem with
parameters $n$ and $t=4n$ has complexity $\Omega(n\log n)$ if
its error probability is at most $1/3$. As pointed out in the
Introduction, to our knowledge this is the fist established case
when solving the OR of $n$ instances of a communication problem
requires strictly more than $n$ times the complexity needed to
solve a single such instance.

We start with with a simple and standard reduction from the
randomized protocol to the deterministic one and further to a
large set of inputs that makes the first (and in this case only)
message fixed. These steps are also used in the general round
elimination argument therefore we state them in general form.

Let $\epsilon>0$ be a small constant and let $P$ be an
$1/3$-error randomized protocol for the exists-equal problem
with parameters $n$ and $t=4n$. We repeat the protocol $P$ in
parallel taking the majority output, so that the number of
rounds does not change, the length of the messages is multiplied
by a constant and the error probability decreases below
$\epsilon$. Now we fix the coins of of this $\epsilon$-error
protocol in a way to make the resulting deterministic protocol
err on at most $\epsilon$ fraction of the possible inputs.
Denote the deterministic protocol we obtain by $Q$.

\begin{lemma}
\label{lem:determine-s}
Let $Q$ be a deterministic protocol for the $\EE_n$ problem that
makes at most $\epsilon$ error on the uniform distribution.
Assume Alice sends the first message of length $c$. There exists
an $S\subset [t]^n$ of size $\mu(S)=2^{-c-1}$ such that the
first message of Alice is fixed when $x\in S$ and we have
$\Pr_{y\sim \mu}[Q(x,y)\neq\EE(x,y)]\leq 2\epsilon$ for all
$x\in S$.
\end{lemma}

\begin{proof}
Note that the quantity $e(x)=\Pr_{y\sim
\mu}[Q(x,y)\neq\EE(x,y)]$, averaged over all $x$, is the error
probability of $Q$ on the uniform input, hence is at most
$\epsilon$. Therefore for at least half of $x$, we have
$e(x)\leq 2\epsilon$. The first message of Alice partitions this
half into at most $2^c$ subsets. We pick $S$ to consist of
$t^n/2^{c+1}$ vectors of the same part: at least one part must
have this many elements.
\end{proof}

We fix a set $S$ as guaranteed by the lemma. We assume we
started with a single round protocol, so $Q(x,y)=Q(x',y)$
whenever $x,x'\in S$. Indeed, Alice sends the same message by
the choice of $S$ and then the output is determined by Bob, who
has the same input in the two cases.

We call a pair $(x,y)$ {\em bad} if $x\in S$, $y\in[t]^n$ and
$Q$ errs on this input, i.e., $Q(x,y)\ne\EE(x,y)$. Let $b$ be
the number of bad pairs. By \autoref{lem:determine-s} each
$x\in|S|$ is involved in at most $2\epsilon t^n$ bad pairs, so
we have $$b\le2\epsilon|S|t^n.$$
%
We call a triple $(x,x',y)$ {\em bad} if $x,x'\in S$,
$y\in[t]^n$, $\EE(x,y)=1$ and $\EE(x',y)=0$. The proof is based
on double counting the number $z$ of bad triples.
%
Note that for a bad triple $(x,x',y)$ we have $Q(x,y)=Q(x',y)$
but $\EE(x,y)\ne\EE(x',y)$, so $Q$ must err on either $(x,y)$ or
$(x',y)$ making one of these pairs bad. Any pair (bad or not) is
involved in at most $|S|$ bad triples, so we have
\begin{align*}
z\le b|S|\le2\epsilon|S|^2t^n.
\end{align*}

Let us fix arbitrary $x,x'\in S$ with $\Match(x,x')\le n/2$. We
estimate the number of $y\in[t]^n$ that makes $(x,x',y)$ a bad
triple. Such a $y$ must have $\Match(x,y)>\Match(x',y)=0$. To
simplify the calculation we only count the vectors $y$ with
$\Match(x,y)=1$. The match between $y$ and $x$ can occur at any
position $i$ with $x_i\ne x'_i$. After fixing the coordinate
$y_i=x_i$ we can pick the remaining coordinates $y_j$ of $y$
freely as long as we avoid $x_j$ and $x'_j$. Thus we have
\begin{align*}
|\{y\emid(x,x'y)\hbox{ is bad}\}|\ge
(n-\Match(x,y))(t-2)^{n-1}\ge(n/2)(t-2)^{n-1}>t^n/14,
\end{align*}
where in the last inequality we used $t=4n$. Let $s$ be the size
of the Hamming ball
$B_{n/2}(x)=\{y\in[t]^n\emid\Match(x,y)>n/2\}$. By the Chernoff
bound we have $s<t^n/n^{n/2}$ (using $t=4n$ again). For a fixed
$x$ we have at least $|S|-s$ choices for $x'\in S$ with
$\Match(x,x')\le n/2$ when the above bound for triples apply.
Thus we have $$z\ge|S|(|S|-s)t^n/14.$$ Combining this with the
lower bound on the number of bad triples we get
$$28\epsilon|S|\ge|S|-s.$$

Therefore we conclude that we either have large error
$\epsilon>1/56$ or else we have $|S|\le2s<2t^n/n^{n/2}$. As we
have $|S|=t^n/2^{c+1}$ the latter possibility implies 
$$c\ge n\log n/2-2.$$ Summarizing we have the following.

\begin{theorem} \label{thm:singleround} A single round
randomized protocol for $\EE_n$ with error probability $1/3$
has complexity $\Omega(n\log n)$.
A single round deterministic protocol for $\EE_n$ that errs on
at most $1/56$ fraction of the inputs has complexity at least
$n\log n/2-2$.
\end{theorem}
% !TeX root = examreport.tex
\section{An isoperimetric inequality on the discrete grid}
\label{sec:disj:isoperimetry}

The isoperimetric problem on the Boolean cube $\{0,1\}^n$ proved
extremely useful in theoretical computer science. The problem is
to determine the set $S\subseteq \{0,1\}^n$ of a fixed
cardinality with the smallest ``perimeter'', or more generally,
to establish connection between the size of a set and the size
of its boundary. Here the boundary can be defined in several
ways. Considering the Boolean cube as a graph where vertices of
Hamming distance 1 are connected, the {\em edge boundary} of a
set $S$ is defined as the set of edges connecting $S$ and its
complement, while the {\em vertex boundary} consists of the
vertices outside $S$ having a neighbor in $S$.

Harper \cite{Harper1964} showed that the vertex boundary of a
Hamming ball is smallest among all sets of equal size, and the
same holds for the edge boundary of a subcube. These results can
be generalized to other cardinalities \cite{Hart1976}; see the
survey by Bezrukov \cite{Bezrukov1994}.

Consider the metric space over the set $[t]^n$ equipped 
with the Hamming distance.
Let $f$ be a concave function on the nonnegative integers 
and $1\le M<n$ be an integer. 
We consider the following
value as a generalized perimeter of a set $S\subseteq[t]^n$:
\begin{align*}
%\label{eq:nbr}
\E_{x\sim\mu}\sparen{f\nparen{\abs{B_M(x)\cap S}}},
\end{align*}
where $B_M(x)=\setbuilder{y\in[t]^n}{\Match(x,y)\ge M}$ is the radius
$n-M$ Hamming ball around $x$. Note that when $M=n-1$ and $f$ is
the counting function given as $f(0)=0$ and $f(l)=1$ for $l>0$
(which is concave), the above quantity is the size of the vertex 
boundary of $S$ up to some normalization. For other concave functions
$f$ and parameters $M$ this quantity can still be considered a
measure of how ``spread out'' the set $S$ is. We conjecture that
$n$-dimensional boxes minimize this measure in every case.

\begin{conjecture}
\label{conj:product}
Let $1\le k\le t$ and $1\le M<n$ be integers. 
Let $S$ be an arbitrary subset of $[t]^n$ of size $k^n$ and 
$P=[k]^n$. We have
\begin{align*}
\E_{x\sim\mu}[f\left(\left|B_M(x)\cap P\right|\right)]\leq
\E_{x\sim\mu}[f\left(\left|B_M(x)\cap S\right|\right)].
\end{align*}
\end{conjecture}
Even though a proof of \autoref{conj:product} remained elusive,
in \autoref{thm:isoperimetry}, we prove an approximate version
of this result, where, for technical reasons, we have to
restrict our attention to a small fraction of the coordinates.
Having this weaker result allows us to prove our communication
complexity lower bound in the next section but proving the
conjecture here would simplify this proof.

\subsection{A shifting argument}
\label{sec:ham:isoperimetry:comp}
We start the technical part of this section by introducing the
notation we will use. For $x,y\in[t]^n$ and $i\in[n]$ we write
$x\sim_iy$ if $x_j=y_j$ for $j\in[n]\setminus\{i\}$. Observe
that $\sim_i$ is an equivalence relation. A set $K\subseteq
[t]^n$ is called an {\em $i$-ideal} if $x\sim_i y$, $x_i<y_i$
and $y\in K$ implies $x\in K$. We call a set $K\subseteq[t]^n$
an {\em ideal} if it is an $i$-ideal for all $i\in[n]$.

For $i\in[n]$ and $x\in[t]^n$ we define
$\down_{i}(x)=(x_1,\ldots,x_{i-1},x_i-1,x_{i+1},\ldots,x_n)$. 
We have $\down_i(x)\in[t]^n$ whenever $x_i>1$.
%
Let $K\subseteq [t]^n$ be a set, $i\in[n]$ and $2\le a\in[t]$. 
For $x\in K$, we define $\down_{i,a}(x,K)=\down_i(x)$ if 
$x_i=a$ and $\down_i(x)\notin K$ and we set 
$\down_{i,a}(x,K)=x$ otherwise. We further define
$\down_{i,a}(K)=\{\down_{i,a}(x,K)\mid x\in K\}$.
For $K\subseteq[t]^n$ and $i\in[n]$ we define
\begin{align*}
%\label{eq:down}
\down_i(K)\defeq\setbuilder{y\in[t]^n}{y_i\le\abs{\setbuilder{z\in K}{y\sim_iz}}}.
\end{align*}
Finally for $K\subseteq[t]^n$ we define
\begin{align*}
%\label{eq:down-cascade}
\down(K)\defeq\down_1\nparen{\down_2\nparen{\ldots\down_n(K)\ldots}}.
\end{align*}
The following lemma states few simple observations about these
down operations.
\begin{lemma}
\label{lem:down}
Let $K\subseteq[t]^n$ be a set and let $i,j\in[n]$ be integers. 
The following hold.
\begin{enumerate}[(i)]
\item $\down_i(K)$ can be obtained from $K$ by applying several
  operations $\down_{i,a}$.
\item $|\down_{i,a}(K)|=|K|$ for each $2\le a\le t$, 
$|\down_i(K)|=|K|$ and $|\down(K)|=|K|$.
\item $\down_i(K)$ is an $i$-ideal and
if $K$ is a $j$-ideal, then $\down_i(K)$ is also a $j$-ideal.
\item $\down(K)$ is an ideal. For any $x\in\down(K)$ we have 
$P\defeq[x_1]\times[x_2]\times\cdots\times[x_n]
\subseteq\down(K)$ and there exists a set $T\subseteq K$ with
$P= \down(T)$.
\end{enumerate}
\end{lemma}
\begin{proof}
For statement (i) notice that as long as $K$ is not 
an $i$-ideal one of the operations $\down_{i,a}$ 
will not fix $K$ and hence will decrease $\sum_{x\in K}x_i$. 
Thus a finite sequence of these operations will transform $K$ 
into an $i$-ideal. 
It is easy to see that the operations $\down_{i,a}$ preserve the
number of elements in each equivalence class of $\sim_i$, 
thus the $i$-ideal we arrive at must indeed be $\down_i(K)$.

Statement (ii) follows directly from the definitions of each of
these $\down$ operations.
 
The first claim of statement (iii), namely that $\down_i(K)$ is
an $i$-ideal, is trivial from the definition. Now assume $j\ne
i$ and $K$ is a $j$-ideal, $y\in\down_i(K)$ and $y_j>1$. To see
that $\down_i(K)$ is a $j$-ideal it is enough to prove that
$\down_j(y)\in\down_i(K)$. Since $y\in\down_i(K)$, there are
$y_i$ distinct vectors $z\in K$ that satisfy $z\sim_i y$.
Considering the vectors $\down_j(z)\sim_i\down_j(y)$ and using
that these distinct vectors are in the $j$-ideal $K$ proves that
$\down_j(y)$ is indeed contained in $\down_i(K)$.

By statement (iii), $\down(K)$ is an $i$-ideal for each $i\in
[n]$. Therefore $\down(K)$ is an ideal and the first part of
statement (iv), that is, $P\subseteq K'$ follows. We prove the
existence of suitable $T$ by induction on the dimension $n$. The
base case $n=0$ (or even $n=1$) is trivial. For the inductive
step consider $K'=\down_2(\down_3(\ldots\down_n(K)\ldots))$. As
$x\in\down(K)=\down_1(K')$, we have distinct vectors $x^{(k)}\in
K'$ for $k=1,\ldots, x_1$, satisfying $x^{(k)}\sim_1x$. Notice
that the construction of $K'$ from $K$ is performed
independently on each of the $(n-1)$-dimensional ``hyperplanes''
$S^l=\{y\in[t]^n\mid y_1=l\}$ as none of the operations
$\down_2,\ldots,\down_n$ change the first coordinate of the
vectors. We apply the inductive hypothesis to obtain the sets
$T^{(k)}\subseteq S^{x^{(k)}_1}\cap K$ such that
$\down_2(\ldots\down_n(T^{(k)})\ldots)=\{x^{(k)}_1\}
\times[x_2]\times\cdots\times[x_n]$. Using again that these sets
are in distinct hyperplanes and the operations
$\down_2,\ldots,\down_n$ act separately on the hyperplanes
$S^l$, we get for $T:=\cup_{k=1}^{x_1}T^{(k)}$ that
$$\down_2(\dots\down_n(T)\dots)=\{x^{(k)}_1\mid
k\in[x_1]\}\times[x_2]\times\cdots\times[x_n].$$ Applying
$\down_1$ on both sides finishes the proof of this last part of
the lemma.
\end{proof}

For sets $x\in[t]^n$, $I\subseteq[n]$, and integer $M\in[n]$ we
define $B_{I,M}(x)=\{y\in[t]^n\mid\Match(x_I,y_I)\ge M\}$. The
projection of $B_{I,M}$ to the coordinates in $I$ is the Hamming
ball of radius $|I|-M$ around the projection of $x$.
\begin{lemma}
\label{lem:list}
Let $I\subseteq[n]$, $M\in[n]$ and let $f$ be a concave function
on the nonnegative integers. For arbitrary $K\subseteq [t]^n$ we
have
$$\E_{x\sim\mu}[f(|B_{I,M}(x)\cap\down(K)|)]\le
\E_{x\sim\mu}[f(|B_{I,M}(x)\cap K|)].$$
\end{lemma}

\begin{proof}
By \autoref{lem:down}(i), the set $\down(K)$ can be obtained
from $K$ by a series of operations $\down_{i,a}$ with various
$i\in[n]$ and $2\le a\le t$. Therefore, it is enough to prove
that the expectation in the lemma does not increase in any one
step. Let us fix $i\in[n]$ and $2\le a\le t$. We write
$N_x=B_{I,M}(x)\cap K$ and $N'_x=B_{I,M}(x)\cap\down_{i,a}(K)$
for $x\in[t]^n$. We need to prove that
\begin{align*}
\E_{x\sim\mu}[f(|N_x|)]\ge\E_{x\sim\mu}[f(|N'_x|)].
\end{align*} 
Note that $|N_x|=|N'_x|$ whenever $i\notin I$ or
$x_i\notin\{a,a-1\}$. Thus, we can assume $i\in I$ and
concentrate on $x\in[t]^n$ with $x_i\in\{a,a-1\}$. It is enough
to prove $f(|N_x|)+f(|N_y|)\ge f(|N'_x|)+f(|N'_y|)$ for any pair
of vectors $x,y\in[t]^n$, satisfying $x_i=a$, and
$y=\down_i(x)$.

Let us fix such a pair $x,y$ and set $C=\{z\in
K\setminus\down_{i,a}(K)\emid\Match(x_I,z_I)=M\}$. Observe that
$N_x = N'_x \cup C$ and $N'_x\cap C=\emptyset$. Similarly,
observe that $N'_y = N_y \cup \down_{i,a}(C)$ and $N_y \cap
\down_{i,a}(C)=\emptyset$. Thus we have $|N'_x|=|N_x|-|C|$ and
$|N'_y|=|N_y|+|\down_{i,a}(C)|=|N_y|+|C|$.

The inequality $f(|N_x|)+f(|N_y|)\ge f(|N'_x|)+f(|N'_y|)$
follows now from the concavity of $f$, the inequalities
$|N'_x|\le|N_y|\le|N'_y|$ and the equality
$|N_x|+|N_y|=|N'_x|+|N'_y|$. Here the first inequality follows
from $\down_{i,a}(N'_x)\subseteq\down_{i,a}(N_y)$, the second
inequality and the equality comes from the observations of the
previous paragraph.
\end{proof}

\subsection{Projecting to a subset of the coordinates}
\label{sec:ham:isoperimetry:proj}
\begin{lemma}
\label{lem:find-prod}
Let $K\subseteq [t]^n$ be arbitrary. 
There exists a vector $x\in K$ having at least $n/5$ 
coordinates that are greater than 
$k\defeq\frac{t}{2}\mu(K)^{5/(4n)}$.
\end{lemma}
\begin{proof}
The number of vectors that have at most $n/5$ coordinates 
greater than $k$ can be upper bounded as 
\begin{align*}
{n\choose n/5} t^{n/5} k^{4n/5} 
= t^n {n\choose n/5} (k/t)^{4n/5} 
= |K|\frac{{n \choose n/5}}{2^{4n /5}},
%\label{eq:quant}
\end{align*}
where in the last step we have substituted
$\frac{k}{t}=\frac{1}{2}\mu(K)^{5/(4n)}$ and 
$\mu(K) = |K| / t^n$.
Estimating ${n\choose n/5}\le 2^{n\entb{1/5}}$, 
we obtain that the above quantity is less than $|K|$.
Therefore, there must exists an $x\in K$ that has at least 
$n/5$ coordinates greater than $k$.
\end{proof}

\begin{theorem}
\label{thm:isoperimetry}
Let $S$ be an arbitrary subset of $[t]^n$. Let
$k=\frac{t}{2}\mu(S)^{5/(4n)}$ and $M = nk/(20t)$. There exists
a subset $T\subset S$ of size $k^{n/5}$ and $I\subset [n]$ of
size $n/5$ such that, defining $N_x=\{x'\in
T\mid\Match(x_I,x'_I)\ge M\}$, we have
\begin{enumerate}[(i)]
\item $\Pr_{x\sim\mu}[N_x=\emptyset] \le 5^{-M}$ and
\item $\E_{x\sim \mu}[\log|N_x|]\geq (n/5-M)\log k - 
n\log k /5^M$, where we take $\log 0 = -1$ to make the above 
expectation exist.
\end{enumerate}
\end{theorem}
\begin{proof}
By \autoref{lem:down}(ii), we have $|\down(S)|=|S|$. By
\autoref{lem:find-prod}, there exists an $x\in\down(S)$ having
at least $n/5$ coordinates that are greater than $k$. Let
$I\subset[n]$ be a set of $n/5$ coordinates such that $x_i\geq
k$ for a fixed $x\in\down(S)$. By \autoref{lem:down}(iv),
$\down(S)$ is an ideal and thus it contains the set
$P=\prod_iP_i$, where $P_i=[k]$ for $i\in I$ and $P_i=\{1\}$ for
$i\notin I$. Also by \autoref{lem:down}(iv), there exists a
$T\subseteq S$ such that $P = \down(T)$. We fix such a set $T$.
Clearly, $|T|=k^{n/5}$.

For a vector $x\in[t]^n$, let $h(x)$ be the number of
coordinates $i\in I$ such that $x_i\in [k]$. Note that
$\E_{x\sim \mu}[h(x)] = 4M$ and $h(x)$ has a binomial
distribution. By the Chernoff bound we have $\Pr_{x\sim
\mu}[h(x)<M] < 5^{-M}$. For $x$ with $h(x)\ge M$ we have
$|B_{I,M}(x)\cap P|\ge k^{n/5-M}$, but for $h(x)<M$ we have
$B_{I,M}(x)\cap P=\emptyset$. With the unusual convention
$\log0=-1$ we have
\begin{align*}
\E_{x\sim \mu} [\log|B_{I,M}(x)\cap P|]
&\ge\Pr[h(x)\ge M](n/5-M)\log k-\Pr[h(x)<M]\\
&>(n/5-M)\log k-n\log k/5^M
\end{align*}

We have $\down(T)=P$ and our unusual $\log$ is concave on the
nonnegative integers, so \autoref{lem:list} applies and proves
statement (ii):
\begin{align*}
\E_{x\sim \mu}[\log |N_x|] &\ge\E_{x\sim \mu} 
[\log|B_{I,M}(x)\cap P|]\\
&\ge(n/5-M)\log k - n\log k /5^M.
\end{align*}

To show statement (i), we apply \autoref{lem:list} with the
concave function $f$ defined as $f(0)=-1$ and $f(l)=0$ for all
$l>0$. We obtain that
\begin{align*}
\Pr_{x\sim\mu}[N_x=\emptyset]
&=-\E_{x\sim\mu}[f(|N_x|)]\\
&\le-\E_{x\sim\mu}[f(|B_{I,M}(x)\cap P|)]\\
&=\Pr_{x\sim\mu}[B_{I,M}(x)\cap P=\emptyset]\\
&<5^{-M}.
\end{align*}
This completes the proof.
\end{proof}

% !TeX root = thesis.tex
\section{Lower bound for $\Disj^n_k$}
\label{sec:disj:lowerbound}

%Let us start with an informal overview of the proof.

%Recall that by $\mu$ we denoted the uniform 
%distribution on $[t]^n$.

%\subsection{Inspecting the protocol for the large instance}
%\label{sec:inspect}
In this section we prove our main lower bound result for the 
exists-equal problem, which implies a corresponding lower bound
for the $k$-disjointness problem.
\begin{theorem}
\label{thm:main}
For any $r\leq\log^*n$, an $r$-round probabilistic protocol for
$\EE_n$ with error probability at most $1/3$ sends at least one
message of size $\Omega(n\log^{(r)}n)$.
\end{theorem}

Note that the $r=1$ round case of this theorem was proved as
\autoref{thm:singleround} in \autoref{sec:elementary}. The other
extreme, which immediately follows from \autoref{thm:main}, is
the following.

\begin{corollary}
Any randomized protocol for $\EE_n$ with maximum message size
$O(n)$ and error $1/3$ has at least $\log^* n - O(1)$ rounds.
\end{corollary}

\autoref{thm:main} is a direct consequence of the corresponding
statement on deterministic protocols with small distributional
error on uniform distribution; see \autoref{thm:main2} at the
end of this section. Indeed, we can decrease the error of a
randomized protocol below any constant $\epsilon>0$ for the
price of increasing the message length by a constant factor,
then we can fix the coins of this low error protocol in a way
that makes the resulting deterministic protocol $Q$ err in at
most $\epsilon$ fraction of the possible inputs. Applying
\autoref{thm:main2} to the protocol $Q$ proves
\autoref{thm:main}.

In the rest of this section we use a round-elimination argument
to prove \autoref{thm:main2}, that is, we will use $Q$ to solve
smaller instances of the exists-equal problem in a way that the
first message is always the same, and hence can be eliminated.

Suppose Alice sends the first message of $c$ bits in protocol $Q$. 
By \autoref{lem:determine-s}, there exists a $S\subset [t]^n$ of
size $\mu(S)=2^{-c-1}$ such that the first message of Alice is
fixed when $x\in S$ and we have $\Pr_{y\sim
\mu}[Q(x,y)\neq\EE(x,y)]\leq 2\epsilon$ for all $x\in S$. Fix
such a set $S$ and let $k\defeq t/2^{\frac{5(c+1)}{4n} + 1}$ and
$M \defeq nk/(20t)$. By \autoref{thm:isoperimetry}, there exists
a $T\subset S$ of size $k^{n/5}$ and $I\subset[n]$ of size $n/5$
such that defining
\begin{align*}
N_x=\{y\in T\mid\Match(x_I,y_I)\ge M\}
\end{align*}
we have $\Pr_{x\sim\mu}[N_x=\emptyset] \le 5^{-M}$ and
$\E_{x\sim \mu}[\log|N_x|]\geq (n/5-M)\log k - n\log k /5^M$.
Let us fix such sets $T$ and $I$. Note also that
\autoref{thm:isoperimetry} guarantees that $T$ is a strict
subset of $S$. Designate an arbitrary element of $S\setminus T$
as $x'_e$.

\subsection{Embedding the smaller problem}
\label{sec:embed}
Let $n'\defeq M/10$ and $t'\defeq4n'$. Suppose Alice and Bob
are given an instance $(u,v)$ of the $\EE_{n'}$ problem,
where $u,v\in[t']^{n'}$. To compute $\EE(u,v)$, 
through a random process, Alice and Bob will map $(u,v)$ to
random vectors $(X', Y)$, where $X'$ and $Y$ are supported on 
$\in[t]^n$, and then run the protocol on $\EE(X',Y)$.
The players embed a smaller instance $u,v\in[t']^{n'}$ of the
exists-equal problem in $\EE_n$ concentrating on the coordinates
$I$ determined above. We set $n'\defeq M/10$ and $t'\defeq4n'$.
Optimally, the same embedding should guarantee low error
probability for all pairs of inputs, but for technical reasons
we need to know the number of coordinate agreements
$\Match(u,v)$ for the input pairs $(u,v)$ in the smaller problem
having $\EE_{n'}(u,v)=1$. Let $R\ge1$ be this number, so we are
interested in inputs $u,v\in[t']^{n'}$ with $\Match(u,v)=0$ or
$R$. We need this extra parameter so that we can eliminate a
non-constant number of rounds and still keep the error bound a
constant. For results on constant round protocols one can
concentrate on the $R=1$ case.

In order to solve the exist-equal problem with parameters $t'$
and $n'$ Alice and Bob use the shared random source to turn their
input $u,v\in[t']^{n'}$ into longer random vectors
$X',Y\in[t]^n$, respectively, and apply the protocol $Q$ above
to solve this exists-equal problem for these larger inputs. Here
we informally list the main requirements on the process
generating $X'$ and $Y$. We require these properties for the
random vectors $X',Y\in[t]^n$ generated from a fixed pair
$u,v\in[t']^{n'}$ satisfying $\Match(u,v)=0$ or $R$.

\begin{enumerate}[(P1)]
\item $\EE(X',Y)=\EE(u,v)$ with large probability,
%\item if $\Match(u,v)=r \geq 1$ and $R\leq r$, then 
%$\Pr[\EE(X', Y)=1] \ge 0.80$;
\label{prop:2}
\item $\supp(X')=T\cup \{x'_e\}$ and
\label{prop:3}
\item $Y\emid X'$ is distributed almost uniformly
\label{prop:4}
\end{enumerate}

Combining these properties with the fact that $\Pr_{y\sim
\mu}[Q(x,y)\neq\EE(x,y)]\leq 2\epsilon$ for each $x\in S$, we
will argue that for the considered pairs of inputs $Q(X',Y)$
equals $\EE(u,v)$ with large probability, thus the combined
protocol solves the small exists-equal instance with small
error, at least for input pairs with $\Match(u,v)=0$ or $R$.
Furthermore, by \propref{prop:3} the first message of Alice will
be fixed and hence does not need to be sent, making the combined
protocol one round shorter.

The random variables $X'$ and $Y$ are constructed as follows.
Let $m\defeq 2n/(MR)$ be an integer. Each player repeats his or
her input ($u$ and $v$, respectively) $m$ times, obtaining a
vector of size $n/(5R)$. Then using the shared randomness, the
players pick $n/(5R)$ uniform random maps $m_i\colon[t']\to[t]$
independently and apply $m_i$ to $i$\/th coordinate.
Furthermore, the players pick a uniform random 1-1 mapping
$\pi\colon[n/(5R)]\to I$ and use it to embed the coordinates of the
vectors they constructed among the coordinates of the vectors
$X$ and $Y$ of length $n$. The remaining $n-n/(5R)$ coordinates
of $X$ is picked uniformly at random by Alice and similarly, the
remaining $n-n/(5R)$ coordinates of $Y$ is picked uniformly at
random by Bob. Note that the marginal distribution of both $X$
and $Y$ are uniform on $[t]^n$. If $\Match(u,v)=0$ the vectors
$X$ and $Y$ are independent, while if $\Match(u,v)=R$, then $Y$
can be obtained by selecting a random subset of $I$ of
cardinality $mR$, copying the corresponding coordinates of $X$
and filling the rest of $Y$ uniformly at random.

This completes the description of the random process for Bob.
However Alice generates one more random variable $X'$ as
follows. Recall that $N_x=\{z\in T\mid\Match(z_I,x_I)\ge M\}$.
The random variable $X'$ is obtained by drawing $x\sim X$ first
and then choosing a uniform random element of $N_x$. In the
(unlikely) case that $N_x=\emptyset$, Alice chooses $X'=x'_e$.

Note that $X'$ either equals $x'_e$ or takes values from $T$,
hence \propref{prop:3} holds. In the next lemma we quantify and
prove \propref{prop:2} as well.

\begin{lemma}
\label{lem:error}
Assume $n\ge3$, $M\ge2$ and $u,v\in[t']^{n'}$. We have
\begin{enumerate}[(i)]
\item 
if $\Match(u,v)=0$ then $\Pr[\EE(X',Y)=0] > 0.77$;
\item 
if $\Match(u,v)=R$, then
$\Pr[\EE(X', Y)=1] \ge 0.80$.
\end{enumerate}
\end{lemma}

\begin{proof}
For the first claim, note that when $\Match(u,v) = 0$, the
random variables $X$ and $Y$ are independent and uniformly
distributed. We construct $X'$ based on $X$, so its value is
also independent of $Y$. Hence $\Pr[\EE(X',Y)=0]=(1-1/t)^n$.
This quantity goes to $e^{-1/4}$ since $t=4n$ and is larger than
$0.77$ when $n\geq 3$. This establishes the first claim.

For the second claim let $J = \{i\in I\mid X_i=Y_i\}$ and
$K=\{i\in I\mid X'_i = X_i\}$. By construction,
$|J|=\Match(X_{I},Y_{I})\ge mR$ and $|K|=\Match(X'_{I}, X_I)
\geq M$ unless $N_X=\emptyset$. By our construction, each
$J\subset I$ of the same size is equally likely by symmetry,
even when we condition on a fix value of $X$ and $X'$. Thus we
have $\E[|J\cap K|\emid N_X\ne\emptyset]\ge mRM/|I|=10$ and
$\Pr[J\cap K=\emptyset\emid N_X\ne\emptyset]<e^{-10}$. Note that
$X$ is distributed uniformly over $[t]^n$, therefore by
Theorem~\ref{thm:isoperimetry}(i) the probability that
$N_X=\emptyset$ is at most $5^{-M}$. Note that
$\Match(X',Y)\ge|J\cap K|$ and thus
$\Pr[\EE(X',Y)=0]\le\Pr[J\cap K=\emptyset]\le\Pr[J\cap
K=\emptyset\emid N_X\ne\emptyset] +\Pr[N_X=\emptyset]\le
e^{-10}+5^{-M}$. This completes the proof.
\end{proof}

We quantify the correlation of $X'$ and $Y$ stated in
\propref{prop:4} by their mutual information. This mutual
information argument is postponed to the next subsection; here
we show how such a bound to the mutual information implies that
the error introduced by $Q$ is small.

\begin{lemma}
\label{lem:kl-err}
Let $\gamma = \Pr[Q(X',Y)\neq \EE(X',Y)]$. 
If $\gamma \ge 2\eps$, then we have $\kldivb{\gamma}{2\eps}\le \muti{X'}{Y}$.
\end{lemma}
\begin{proof}
For all fixings $x\in\supp(X')$ and a distribution $\nu$
on $[t]^n$, define 
\begin{align*}
e_{x}(\nu) \defeq \Pr_{y\sim \nu}[Q(x, y)\neq\EE(x, y)].
\end{align*}
By the definition of mutual information and the conditional divergence, 
%We prove the contrapositive of the
%statement of the lemma, that is assuming $\Pr_{y\sim Y|X'=x'}[Q(x',y) 
%\neq \EE(x',y)]>\gamma$ we prove $\Ent(Y\emid X'=x')< n\log t - 
%\BD(\gamma \dmid 2\epsilon)$:
\begin{align*}
\muti{X'}{Y}
    &=   \kldiv{Y\emid X'}{Y}\\
    &=   \E_{x\sim X'}\kldiv{Y\emid X'=x}{Y}\\
    &\ge \E_{x\sim X'}
         \kldivb{e_x\nparen{\dist(Y\emid X'=x)}}
                         {e_x\nparen{\mu}}\\
    &\ge \kldivb{\E_{x\sim X'}e_x\nparen{\dist(Y\emid X'=x)}}
                         {\E_{x\sim X'}e_x\nparen{\mu}}\\
    &= \kldivb{\gamma}{\E_{x\sim X'}e_x\nparen{\mu}}\\
    &\ge \kldivb{\gamma}{2\eps}
\end{align*}
where the first inequality is the data processing inequality,
the second inequality follows from the convexity of $\kldiv{\cdot}{\cdot}$
and the last inequality follows from 
the guarantee $e_x(\mu)\le 2\eps$ 
for all $x$ that is provided by \autoref{lem:determine-s}
and the assumption of the present lemma that $\gamma\ge 2\eps$.
\end{proof}

\subsection{Establishing the low correlation property}

We quantify \propref{prop:4} using the mutual information. 
If $\Match(u,v)=R$ our process generates $X$
and $Y$ with the expected number $\E[\Match(X_I,Y_I)]$ of
matches only slightly more than the minimum $mR$. We lose most
of these matches with $Y$ when we replace $X$ by $X'$ and only
an expected constant number remains. A constant number of forced
matches with $X'$ within $I$ restricts the number of possible
vectors $Y$ but it only decreases the entropy by $O(1)$. The
calculations in this subsection make this intuitive argument
precise.

Recall that $X$ and $Y$ are correlated due to the random process
with which Alice and Bob generate them and $X'$ is obtained from
$X$. To understand $\muti{X'}{Y}$
\begin{lemma}
\label{lem:x-uniform}
For any $u,v\in[t']^{n'}$ it holds that
$\muti{X}{X'}\le M\log k  + n\log k / 5^M$.
\end{lemma}
\begin{proof}
We have
\begin{align*}
\muti{X}{X'} = \ent{X'} - \ent{X'\emid X}
\end{align*}
and 
$\ent{X'}\le \log |\supp(X')|
    = \log \nparen{|T|+1}\le \frac{n}{5}\log k + 1$.

Observe that $\ent{X'\emid X} = \E_{x\sim
\mu}[\log|N_x|]$, where $\log 0$ is now taken to be $0$. From
\autoref{thm:isoperimetry}(ii) we get $\ent{X'\emid X}\geq
\frac{n}{5}\log k - M\log k - n\log k/5^M$.
Plugging in, the $\frac{n}{5}\log k$ terms cancel and we get
the statement of the lemma.
\end{proof}


\begin{lemma}
\label{lem:y-uniform}
Let $X',Y$ be as constructed above. The following hold.
\begin{enumerate}[(i)]
\item If $\Match(u,v)=0$ we have
$\muti{X'}{Y} = 0$
\item If $M>100\log n$ and $\Match(u,v)=R$ we have
$\muti{X'}{Y} = O(1).$
\end{enumerate}
\end{lemma}
\begin{proof}
Part (i) holds as $Y$ is independent of $X'$
whenever $\EE(u,v)=0$ by construction.

For part (ii) recall that if $\Match(u,v)=R$ one can construct
$X$ and $Y$ by uniformly selecting a size $mR$ set $L\subseteq
I$ and selecting $X$ and $Y$ uniformly among all pairs
satisfying $X_L=Y_L$. Recall that $L$ is the set of coordinates
the $mR$ matches between $u^m$ and $v^m$ were mapped. These are
the ``intentional matches'' between $X_I$ and $Y_I$. Note that
there may be also ``unintended matches'' between $X_I$ and
$Y_I$, but not too many: their expected number is
$(n/5-mR)/t<1/20$. As given any fixed $L$, the marginal
distribution of both $X$ and $Y$ are still uniform, so in
particular $X$ is independent of $L$ and so is $X'$ constructed
from $X$. Let us expand $\muti{X'L}{Y}$ using the chain rule in
two different ways obtaining
\begin{align}
\muti{X'L}{Y}
    &= \muti{X'}{Y} + \muti{L}{Y\emid X'}\nonumber\\
    &= \muti{L}{Y} + \muti{X'}{Y\emid L}\label{teq:mutexpand}.
\end{align}
Since the first term of \eqref{teq:mutexpand} is zero 
by independence of $Y$ and $L$, we conclude
\begin{align}
\muti{X'}{Y}
    &= \muti{X'}{Y\emid L} - \muti{L}{Y\emid X'}\nonumber\\
    &= \muti{X'}{Y\emid L} - \muti{L}{X'Y}\label{teq:twoterms},
\end{align}
where the second inequality follows again by the chain rule 
and the fact that $X'$ and $L$ are independent.
Let us understand the terms of \eqref{teq:twoterms} one by one.
First we  expand the first term by the chain rule, obtaining
\begin{align*}
\muti{X'}{Y\emid L} = \muti{X'}{Y_L\emid L} + \muti{X'}{Y_{[n]\setminus L}\emid LY_L}
\end{align*}
however since $Y_{[n]\setminus L}$ is uniformly distributed for any fixed
$L$, $X'$ and $Y_L$, the second term on the right hand side is zero. 
By construction we have $X_L=Y_L$, thus
\begin{align*}
\muti{Y_L}{X'\emid L} 
    &= \muti{X_L}{X'\emid L}\\
    &\le \frac{mR}{n/5}\muti{X_I}{X'}\\
    &\le 10\log k + \frac{MR}{5^{M-1}}\log k,
\end{align*}
where the first inequality follows by \autoref{lem:ent-subset}
as $L$ is a uniform and independent of $X$ and $X'$ and the
second inequality follows from \autoref{lem:x-uniform} that we
will prove shortly and the formula defining $m$.

Here, when condition on $L$, the correlation of $X'$ and $Y$
is roughly $10\log k$ bits, which is significantly more than the
constant bound we seek. Next we will see that all but a constant bits
of this correlation comes from having observed what $L$ is.
The next term, $\ent{L}$ is easy to compute as $L$ is a uniform
subset of $I$ of size $mR$:
\begin{align*}
\ent{L}=\log{n/5\choose mR}
\end{align*}

It remains to bound the term $\ent{L\emid Y, X'}$. Let
$Z=\{i\mid i\in I \text{ and } X'_i=Y_i\}$. Note that $Z$ can be
derived from $X',Y$ (as $I$ is fixed) hence 
$\ent{L\emid Y,X'}\leq \ent{L\emid Z}$. 
Further, let $C=|Z\setminus L|$. We obtain
\begin{align*}
\ent{L\emid Y,X'}&\le\ent{L\emid Z}\leq \ent{L\emid Z, C} + \ent{C}\\
&< \E_{Z,C}\left[\log{n/5-|Z|+C \choose mR - |Z|+C}\right]
+ \E_{Z,C}\left[\log {|Z| \choose C}\right]+2
\end{align*}
where we used $\ent{C}<2$. Note that for any fixed $x'\in T$ and
$x\in \supp(X\emid X'=x')$, we have $$\E[|Z|-C\emid X=x,
X'=x']=\Match(x_I,x_I') mR /(n/5) \geq 10$$ as $\Match(x_I,
x_I')\geq M$ by definition.
Hence we have $$\log{n/5\choose mR}-\log{n/5-|Z|+|C|\choose
mR-|Z|+|C|}\ge10\log\frac n{5m}-O(1),$$ $$\E_{Z,C}\left[\log
{|Z| \choose C}\right] \le \E[|Z|] < 20.$$ Summing the estimates
above for the various parts of $\ent{Y\emid X'}$ the statement
of the lemma follows.
\end{proof}

It remains to prove the following simple lemma that ``reverses''
the conditional entropy bound in \autoref{thm:isoperimetry}(ii):

\subsection{The round elimination lemma}

Let $\nu_n$ be the uniform distribution on $[t]^n\times [t]^n$,
where we set $t=4n$. The following lemma gives the base case of
the round elimination argument.
\begin{lemma}\label{lem:terminal} 
Any 0-round deterministic protocol for $\EE_n$ has at least 0.22
distributional error on $\nu_n$, when $n\geq 1$.
\end{lemma}
\begin{proof}
The output of the protocol is decided by a single player, say
Bob. For any given input $y\in[t]^n$ we have $3/4
\leq\Pr_{x\sim\mu}[\EE(x,y)=0] < e^{-1/4} < 0.78$. Therefore the
distributional error is at least $0.22$ for any given $y$
regardless of the output Bob chooses, thus the overall error is
also at least $0.22$.
\end{proof}
Now we give our full round elimination lemma.
\begin{lemma}\label{lem:roundel} 
Let $r>0, c ,n$ be an integers such that $c < \frac{4}{5}n\log n$.
There is a constant $0<\epsilon_0<1/200$ such that if there is
an $r$-round deterministic protocol with $c$-bit messages for
$\EE_n$ that has $\epsilon_0$ error on $\nu_n$, then there is an
$(r-1)$-round deterministic protocol with $O(c)$-bit messages
for $\EE_{n'}$ that has $\epsilon_0$ error on $\nu_{n'}$, where
$n' = \Omega(n/2^\frac{5c}{4n})$.
\end{lemma}
\begin{proof}
We start with an intuitive description of our reduction. Let us
be given the deterministic protocol $Q$ for $\EE_n$ that errs on
an $\epsilon_0$ fraction of the inputs. To solve an instance
$(u,v)$ of the smaller $\EE_{n'}$ problem the players perform
the embedding procedure described in previous subsection $k_0$
times independently for each parameter $R\in[R_0]$. Here $k_0$
and $R_0$ are constants we set later. They perform the protocol
$Q$ in parallel for each of the $k_0R_0$ pairs of inputs they
generated. Then they take the majority of the $k_0$ outputs for
a fixed parameter $R$. We show that this result gives the
correct value of $\EE(u,v)$ with large probability provided that
$\Match(u,v)=0$ or $R$. Finally they take the OR of these
results for the $R_0$ possible values of $R$. By the union bound
this gives the correct value $\EE(u,v)$ with large probability
provided $\Match(u,v)\le R_0$. Fixing the random choices of the
reduction we obtain a deterministic protocol. The probability of
error for the uniform random input can only grow by the small
probability that $\Match(u,v)>R_0$ and we make sure it remains
below $\epsilon_0$. The rest of the proof makes this argument
precise.

For random variables $X'$ and $Y$ constructed in
\autoref{sec:embed}, \autoref{lem:y-uniform} guarantees that
$\ent{Y\emid X'}\ge n\log t - \alpha_0$ for some constant
$\alpha_0$, as long as $M>100\log n$ and $\Match(u,v)=R$. Let
$\epsilon_0$ be a constant such that $\kldivb{1/10}{2\epsilon_0}
> 200(\alpha_0 + 1)$. Note that such $\epsilon_0$ can be found
as $\kldivb{1/10}{\epsilon}$ tends to infinity as $\epsilon$
goes to 0. We can bound $\Pr_{(x,y)\sim\nu_m}[\Match(x,y) \ge l]
\le 1/(4^l l!)$ for all $m\ge1$. We set $R_0$ such that
$\Pr_{(x,y)\sim\nu_m}[\Match(x,y) \ge R_0 ] \le \epsilon_0 / 2$
for all $m\ge1$.

Let $Q$ be a deterministic protocol for $\EE_n$ that sends $c <
(n\log n)/2$ in each round and that has $\epsilon_0$ error on
$\nu_n$. Let $S$ be as constructed in \autoref{lem:determine-s}
and let $M$ be as defined in \autoref{thm:isoperimetry}. We have
$M=\frac{n}{40}2^{\frac{-5(c+1)}{4n}}$ as $t=4n$ and
$\mu(S)=2^{-(c+1)}$ by \autoref{lem:determine-s}. Note that by
our choice of $c$, we have $M>100\log n$, hence the hypotheses
of \autoref{lem:y-uniform} are satisfied.

Let $n' = M/10 = \frac{n}{400}2^{\frac{-5(c+1)}{4n}}$. Now we
give a randomized protocol $Q'$ for $\EE_{n'}$. Suppose the
players are given an instance of $\EE_{n'}$, namely the vectors
$(u,v)\in[4n']^{n'}\times[4n']^{n'}$. Let $k_0 = 10\log (R_0 +
1/\epsilon_0)$. For $R\in[R_0]$ and $k\in [k_0]$, the players
construct the vectors $X'_{R,k}$ and $Y_{R,k}$ as described in
\autoref{sec:embed} with parameter $R$ and with fresh randomness
for each of the $R_0k_0$ procedures. The players run $R_0 k_0$
instances of protocol $Q$ in parallel, on inputs $X'_{R,k},
Y_{R,k}$ for $R\in[R_0]$ and $k\in[k_0]$. Note that the first
message of the first player, Alice, is fixed for all instances
of $Q$ by \propref{prop:3} and \autoref{lem:determine-s}.
Therefore, the second player, Bob, can start the protocol
assuming Alice has sent the fixed first message. After the
protocols finish, for each $R\in[R_0]$, the last player who
received a message computes $b_R$ as the majority of
$Q(X_{R,k}',Y_{R,k})$ for $k\in [k_0]$. Finally, this player
outputs $0$ if $b_R=0$ for all $R\in[R_0]$ and outputs $1$
otherwise.

Suppose now that $\EE(u,v) = 0$. By \autoref{lem:error}(i), we
have $\Pr[\EE(X'_{R,k},Y_{R,k}) = 0] \ge 0.77$ for each $R$ and
$k$. Recall that that $Y_{R,k}$ is distributed uniformly for
each $R$ and $k$ and since $\EE(u,v)=0$, it is independent of
$X'_{R,k}$. Therefore, by $X'_{R,k}\in S$ (\propref{prop:3}) and
the fact that $\Pr_{y\sim \mu}[Q(x,y)\neq\EE(x,y)]\leq
2\epsilon_0$ for all $x\in S$ as per \autoref{lem:determine-s},
we obtain $\Pr[Q(X'_{R,k},Y_{R,k}) = 0] \ge 0.77 - 2\epsilon_0 >
0.76$. By the Chernoff bound we have $\Pr[b_R = 1] <
\epsilon_0/(2R_0)$, and by the union bound $\Pr[Q'\hbox{ outputs
}0]\ge1-\epsilon_0 /2$.

Let us now consider the case $\Match(u,v) = R$ for some
$R\in[R_0]$. Fix any $k\in[k_o]$ and set $X'=X'_{R,k}$,
$Y=Y_{R,k}$. By \autoref{lem:error}(ii), $\Pr[\EE(X',Y) = 1]\ge
0.80$. By \autoref{lem:y-uniform}, $\muti{X'}{Y}\leq \alpha_0$ 
and that $Y$ is distributed uniformly at random.
By \autoref{lem:kl-err} and our choice of $\epsilon_0$, we
have $\Pr[\EE(X', Y)\neq Q(X', Y)] < 1/10$. Furthermore, by
\autoref{lem:error}(ii), $\Pr[\EE(u,v) \neq \EE(X', Y)]< 0.20$
hence with probability at least $0.70$ we have $\EE(u,v) = Q(X',
Y)$. This happens independently for all the values of
$k\in[k_0]$, so by the Chernoff bound and our choice of $k_0$,
we have $\Pr[Q'\hbox{ outputs }0]\le\Pr[b_R = 0] < \epsilon_0 /
2$.

Finally, $\Pr_{(u,v)\sim \nu_{n'}}[\Match(u,v) \ge R_0] \le
\epsilon_0 /2$ by our choice of $R_0$. Note that the protocol
$Q'$ uses a shared random bit string, say $W$, in the
construction of the vectors $X'_{R,k}$ and $Y_{R,k}$. Hence,
overall, we have
\begin{align*}
  \Pr_{W, (u,v)\sim\nu_{n'}}[\EE(u,v) = Q'(u,v)]
      \ge 1 - \eps_0
\end{align*}
Since we measure the error of the protocol under a
distribution, we can fix $W$ to a value without increasing the
error under the aforementioned distribution by the so called
easy direction of Yao's lemma. Namely, there exists a $w\in
\supp(W)$ such that
\begin{align*}
  \Pr_{(u,v)\sim\nu_{n'}}[\EE(u,v)  = Q'(u,v)\emid W=w] 
      \ge 1 - \eps_0
\end{align*}
Fix such $w$. Observe that $Q'$ is a $(r-1)$-round protocol for
$\EE_{n'}$ where
$n'=\frac{n}{400}2^\frac{-5(c+1)}{4n}=\Omega(n/2^\frac{5c}{4n})$
and it sends at most $R_0k_0c = O(c)$ bits in each message.
Furthermore, $Q'$ is deterministic and has at most $\eps_0$
error on $\nu_{n'}$ as desired.
\end{proof}

\begin{theorem}
\label{thm:main2}
There exists a constant $\epsilon_0$ such that for any $r\leq\log^*n$,
an $r$-round deterministic protocol for $\EE_n$ which has
$\epsilon_0$ error on $\nu_n$ sends at least one message of size 
$\Omega(n\log^{(r)}n)$.
\end{theorem}
\begin{proof}
Suppose we have an $r$-round protocol with $c$-bit messages for
$\EE_n$ that has $\epsilon_0$ error on $\nu_n$, where $c=\gamma
n\log^{(r)}n$ for some $\gamma<4/5-o(1)$. By
\autoref{lem:roundel}, this protocol can be converted to an
$r-1$ round protocol with $\alpha c$-bit messages for $\EE_{n'}$
that has $\epsilon_0$-error on $\nu_{n'}$, where $n'=\beta
n/2^{5c/4n}$ for some $\alpha, \beta >0$. We only need to verify
that $\alpha c \leq \gamma n'\log^{(r-1)}n'$. We have
\begin{align*}
\gamma n'\log^{(r-1)} n' &= \gamma\beta n/2^{5c/4n}\log^{(r-1)}
(\beta n/2^{5c/4n})\\
&= \gamma\beta n/2^{\frac{5\gamma}{4}\log^{(r)}n}\log^{(r-1)}
(\beta n/2^{5c/4n})\\
&\geq \gamma\beta n \left(\log^{(r-1)}n\right)^{1-\frac{5\gamma}{4}-o(1)}\\
&\geq \gamma\alpha n\log^{(r)}n
\end{align*}
for $\gamma< 4/5 - o(1)$ and large enough $n$. Therefore, by
iteratively applying \autoref{lem:roundel} we obtain a $0$-round
protocol for $\EE_{\bar n}$ that makes $\epsilon_0$ error on
$\nu_{\bar n}$ for some $\bar n$ satisfying $\gamma {\bar n}^2 =
\gamma \bar n \log^{(0)} \bar n\geq c \alpha^r$. Therefore $\bar
n \geq 1$ and since $\epsilon_0< 0.22$, the protocol we obtain
contradicts \autoref{lem:terminal}, showing that the protocol we
started with cannot exists.
\end{proof}
\begin{remark}
We note that in the proof of \autoref{thm:main}, to show that a
protocol with small communication does not exist, we start with
the given protocol and apply the round elimination lemma (i.e.,
\autoref{lem:roundel}) $r$ times to obtain a $0$-round protocol
with small error probability, which is shown to be impossible by
\autoref{lem:terminal}. Alternatively, one can apply the round
elimination $r-1$ times to obtain a $1$-round protocol with
$o(n\log n)$ communication for $\EE_{n}$, which is ruled out by
\autoref{thm:singleround}.
\end{remark}

% !TeX root = thesis.tex
\section{Discussion}
\label{sec:disj:discussion}
The $r$-round protocol we gave in \autoref{sec:disj:upperbound}
solves the sparse set disjointness problem in $O(k\log^{(r)}k)$
total communication. As we proved in \autoref{sec:disj:lowerbound}
this is optimal. The same, however, cannot be said of the error
probability. With the same protocol, but with more careful
setting of the parameters the exponentially small error
$O(2^{-\sqrt k})$ of the $\log^*k$-round protocol can be further
decreased to $2^{-k^{1-o(1)}}$.

For small (say, constant) values of $r$ this protocol cannot
achieve exponentially small error error without the increase in
the complexity if the universe size $m$ is unbounded. But if $m$
is polynomial in $k$ (or even slightly larger,
$m=\exp^{(r)}(O(\log^{(r)}k))$), we can replace the last round
of the protocol by one player deterministically sending his or
her entire ``current set'' $S_r$. With careful setting of the
parameters in other rounds, this modified protocol has the same
$O(k\log^{(r)}k)$ complexity but the error is now exponentially
small: $O(2^{-k/\log k})$. Note that in our lower bound on the
$r$-round complexity of the sparse set disjointness we we use
the exists-equal problem with parameters $n=k$ and $t=4k$. This
corresponds to the universe size $m=tn=4k^2$. In this case any
protocol solving the exists-equal problem with $1/3$ error can
be strengthened to exponentially small error using the same
number of rounds and only a constant factor more communication.

Our lower and upper bounds match for the exists-equal problem
with parameters $n$ and $t=\Omega(n)$, since the upper bounds
were established without any regard of the universe size, while
the lower bounds worked for $t=4n$. Extensions of the techniques
presented in this paper give matching bounds also in the case
$3\le t<n$, where the $r$-round complexity is
$\Theta(n\log^{(r)}t)$ for $r\le\log^*t$. Note, however, that in
this case one needs to consider significantly more complicated
input distributions and a more refined isoperimetric inequality,
that does not permit arbitrary mismatches. The $\Omega(n)$ lower
bound applies for the exists-equal problem of parameters $n$ and
$t\ge3$ regardless of the number of rounds, as the disjointness
problem on a universe of size $n$ is a sub-problem. For $t=2$
the situation is drastically different, the exists-equal problem
with $t=2$ is equivalent to a single equality problem.

Finally a remark on using the joint random source model of
randomized protocols throughout the paper. By a result of Newman
\cite{Newman1991} our protocols of \autoref{sec:disj:upperbound} can be
made to work in private coin model (or even if one of the
players is forced to behave deterministically) by increasing the
first message length by $O(\log\log(N)+\log(1/\epsilon))$ bits,
where $N= {m \choose k}$ is the number of possible inputs. In
our case this means adding the term $O(\log\log m)+o(k)$ to our
bound of $\smash{O(k\log^{(r)}k)}$, since our protocols make at least
$\exp(-k/\log k)$ error. This additional cost is insignificant
for reasonably small values of $m$, but it is necessary for
large values as the equality problem, which is an instance of
disjointness, requires $\Omega(\log \log m)$-bits in the private
coin model.

Note also that we achieve a super-linear increase in the
communication for OR of $n$ instances of equality even in the
private coin model for $r=1$. For $r\geq 2$, no such increase
happens in the private coin model as communication complexity of
$\EE^t_n$ is at most $O(n\log\log t)$ however a single equality
problem requires $\Omega(\log \log t)$ bits.

\section{Chapter notes}
\label{sec:disj:notes}
The results presented in this chapter are obtained 
with Gábor Tardos and published in our joint paper
\cite{SaglamT2013} in FOCS 2013.

% !TeX root = thesis.tex
\chapter{The Blakley-Dixon-Erdos-Simonovits Conjecture}
\label{sec:heat}

Suppose that some initial heat configuration 
$u\colon\Omega\to\realspos$ is given over a finite 
space $\Omega$ and the configuration evolves 
according to the map $w\mapsto Sw$ in each time step 
$t=0,1,\ldots$, for some symmetric stochastic matrix 
$S\colon \Omega\times\Omega\to\realspos\,$.
Assume that we are interested in the amount of heat 
contained in a certain region $R\subseteq \Omega$ 
and how this quantity changes over time. In notation, 
assuming $\lnorm{u}{2}=1$ for normalization purposes 
and $v(x) \defeq \indicate{x\in R}/|R|^{1/2}$ for 
$x\in\Omega$, we would like to understand how
\begin{align*}
m_t \defeq \iprod{v}{S^tu}
\end{align*}
changes as a function of $t$. In this paper we 
derive local bounds that $\set{m_t}_{t=0}^\infty$ 
must obey for any $S,u$ and $v$ satisfying the 
symmetry, magnitude and positivity constraints above 
(in fact our bounds work for any countable
$\Omega$, arbitrary non-negative unit vector $v$ 
and symmetric non-negative $S$).
Our first bound $m_{t+2}\ge \smash{m_t^{1+2/t}}$ 
answers a question of Blakley and Dixon from 1966
(see \autoref{conj:blakley-dixon} below) which was later 
conjectured independently by 
Erdős and Simonovits in 1982 also 
(see \autoref{conj:erdos-simonovits} in 
\autoref{sec:relatedwork}).

Moreover we establish a tight connection between such 
bounds and the well-studied $k$-Hamming distance problem
\cite{PangG1986, Yao2003, CormodePS2000, BarYossefJKK2004,
GavinskyKW2004, HuangSZZ2006,
BlaisBM2012, BuhrmanGMW2012, BlaisBG2014, AmbainisGSU2015} 
and the $k$-Hamming weight problem
\cite{AdaFH2012, BlaisK2012, BuhrmanGMW2012}
and obtain the first tight bounds for respectively
the communication complexity and parity decision tree 
complexity of them.
Our tight $\Omega(k \log (k/\delta))$ lower bound 
for the $\delta$-error communication complexity of 
the $k$-Hamming distance problem (that applies 
whenever $k^2< \delta n$) answers affirmatively a 
conjecture stated in \cite{BlaisBG2014} 
(Conjecture 1.4).
Prior to our work, the best impossibility results 
for this problem were an $\Omega(k\log^{(r)}k)$ bits 
lower bound ($\smash{\log^{(r)}z}$ being the $r$ nested 
applications of logarithm) that applies to any randomized $r$-round 
communication protocol \cite{SaglamT2013}, 
and an $\Omega(k\log (1/\delta))$ lower bound that 
applies to any $\delta$-error randomized protocol for 
$k <\delta n$ \cite{BlaisBG2014}.

Our parity decision tree lower bound shows that any 
$\delta$-error parity decision tree solving the 
$k$-Hamming weight problem has size 
$\exp\Omega\nparen{k\log (k/\delta)}$, 
which directly implies an $\Omega(k\log (k/\delta))$ 
bound on the depth of any such decision tree. 
Previously no nontrivial lower bound was known for 
the parity decision tree size of this problem and 
an $\Omega(k\log (1/\delta))$ bound on the parity 
decision tree depth followed from the communication 
complexity bound of \cite{BlaisBG2014}. Prior to 
\cite{BlaisBG2014}, the best bound on the parity 
decision tree depth was $\Omega(k)$, derived in 
\cite{BlaisBM2012} and \cite{BlaisK2012}.

Either by combining our communication complexity lower 
bound with the reduction technique developed in 
\cite{BlaisBM2012} or by combining our parity decision 
tree lower bound with a reduction given in 
\cite{BhrushundiCK2014}, one obtains an 
$\Omega(k\log (k/\delta))$ bound for any (potentially adaptive) 
property tester for the $\delta$-error probability 
$k$-linearity testing problem. This establishes the 
correct bound for this problem which was studied extensively 
\cite{FischerLNRRS2002, Goldreich2010, BlaisK2012, 
BhrushundiCK2014,BuhrmanGMW2012, BlaisBM2012}
since \cite{FischerLNRRS2002} or earlier.


\subsection{Motivating our bounds on $m_t$}
We would like to provide some intuition as to why one should expect
\begin{align}
m_{t+2}    &\ge m_t^{1+2/t}, \text{ and}\label{eq:bd66}\\
m_{t+2}    &\ge m_t^{1+2/t}\cdot \min\set{t^{1-\eps}, 
            \delta\frac{m_t^{1-2/t}}{m_{t-2}}}
\end{align}
to hold for appropriate $\eps,\delta$.
Recalling that $S$ is a symmetric matrix with maximum 
eigenvalue 1, we may write $S=QDQ^\transpose$
for an orthonormal matrix $Q$ having columns $q_x$, $x\in\Omega$
and a diagonal matrix $D$ with entries $\lambda_x \le 1$, 
$x\in \Omega$. Plugging this into 
$m_t=\iprod{v}{S^tu}$, we get
\begin{align}
\label{eq:spectralsum}
m_t = \sum_{x\in\Omega}\lambda_x^t\iprod{u}{q_x}\iprod{v}{q_x}.
\end{align}
For sake of analogy let us drop our assumption that $S, u, v$
are coordinate-wise nonnegative for a moment but instead assume
that each summand in the right hand side of
\autoref{eq:spectralsum}
is nonnegative by some coincidence. In this case we can
consider $\set{m_t}_{t=0}^\infty$ as the moment sequence of a 
random variable supported on $[0,1]$ that takes the
value $|\lambda_x|$ with probability 
$|\iprod{u}{q_x}\iprod{v}{q_x}|$ and the value 0 with
probability
$1-\sum_x|\iprod{u}{q_x}\iprod{v}{q_x}|$ (which is nonnegative
by Cauchy-Schwarz inequality). 
This would imply that $\set{m_t}_{t=0}^\infty$ is 
{\em completely monotone} by Hausdorff's characterization 
\cite{Hausdorff1921} and therefore log-convex 
(e.g., \cite{NiculescuP2005}, Section 2.1, Example 6).

One particular implication of the log-convexity of 
$\set{m_t}_{t=0}^\infty$, that 
$\frac{1}{t}\log m_t + \frac{t-1}{t}\log m_0\ge \log m_1$,
when combined with the fact $0\le m_0\le 1$ (that follows from our
assumption on the terms of \autoref{eq:spectralsum}), leads to
$m_t\ge m_1^t$.
In 1958, Mandel and Hughes showed that if $u=v$, rather
surprisingly, one can trade the assumption that the summands of
\autoref{eq:spectralsum} are nonnegative with the assumption
that $S$ and $u=v$ are coordinate-wise nonnegative and still
obtain the conclusion $m_t\ge m_1^t$:
\begin{theorem}[Mandel and Hughes \cite{MandelH1958}]
\label{thm:blakley-roy}
Let $u$ be a nonnegative unit vector and 
$S$ be a symmetric matrix with nonnegative entries. For an integer%
\footnote{Since $u=v$ here, the summands inside 
\autoref{eq:spectralsum} are nonnegative when $t$ is even  
so this theorem is most interesting for $t$ odd.}
$t\ge1$ we have $\iprod{u}{S^tu}\ge \iprod{u}{Su}^t$.
\end{theorem}


A more general implication of the log-convexity of 
$\set{m_t}_{t=0}^\infty$ and that $m_0\le 1$ is that for 
$k\ge t$, $\frac{t}{k}\log m_k + \frac{k-t}{k}\log m_0\ge \log m_t$,
therefore $m_k^t\ge m_t^k$. In 1966,
Blakley and Dixon \cite{BlakleyD1966} investigated 
whether $m_k^t\ge m_t^k$ holds in the case $u=v$ when the nonnegativity
assumption on the summands of \autoref{eq:spectralsum} is
replaced by the coordinate-wise nonnegativity of $S$, $u=v$.
They note that the inequality $m_k^t\ge m_t^k$ 
fails when $k$ and $t$ have different 
parity and otherwise holds true under the restriction 
$m_t\ge e^{-4t}$.
While the following is not explicitly stated as a conjecture 
in \cite{BlakleyD1966}, they write
\begin{quote}
if $t> 1$, [...] we cannot show that the inequality
\autoref{eq:bd66} holds for each nonnegative $|\Omega|$-vector
$u$ if $S$ is nonnegative.
\end{quote}
so with the earlier caveat we attribute the following to Blakley and
Dixon \cite{BlakleyD1966}:

\begin{conjecture}[Blakley and Dixon \cite{BlakleyD1966}]
\label{conj:blakley-dixon}
Let $S\colon\Omega\times\Omega\to\realspos$ be a symmetric matrix
with nonnegative entries and let $u\colon\Omega\to\realspos$ be a
nonnegative unit vector. For positive integers 
$k\ge t$ of the same parity, we have
  \begin{align*}
    \iprod{u}{S^ku}^t\ge \iprod{u}{S^tu}^k.
  \end{align*}
\end{conjecture}

In \autoref{sec:heat:monotonicity} we prove the following theorem which
shows that a generalization of \autoref{conj:blakley-dixon}
holds true.
\begin{theorem}
\label{thm:blakley-dixon}
Let  $S\colon\Omega\times\Omega\to\realspos$ be a symmetric matrix
with nonnegative entries and 
$u,v\colon\Omega\to\realspos$ be
nonnegative unit vectors. For positive integers 
$k\ge t$ of the same
parity, we have
  \begin{align*}
    \iprod{v}{S^ku}^t\ge \iprod{v}{S^tu}^k.
  \end{align*}
\end{theorem}
It goes without saying that \autoref{eq:bd66} is 
equivalent to \autoref{thm:blakley-dixon} as we can rearrange 
\autoref{eq:bd66} to $\smash{m_{t+2}^{1/(t+2)}\ge m_t^{1/t}}$ and apply
it iteratively to obtain 
$\smash{m_k^{1/k}\ge\cdots\ge m_{t+2}^{1/(t+2)}\ge m_t^{1/t}}$ whenever 
$k\ge t$ and $k,t$ have the same parity. Moreover, while defining 
\autoref{eq:bd66} we assumed $S$ to be substochastic only to illustrate our
interpretation of the inequality: indeed any nonnegative $S$ can be scaled 
to be substochastic as both sides of \autoref{eq:bd66} are 
$(t+2)$-homogeneous in $S$.

In \autoref{thm:blakley-roy} and \autoref{thm:blakley-dixon} 
we observed that increasingly more general implications 
of the log-convexity of $\set{m_t}_{t=0}^\infty$ can be 
derived by only assuming the coordinate-wise nonnegativity 
of $S,u$ and $v$. One may naturally wonder if the 
coordinate-wise nonnegativity of $S,u$ and $v$ implies
the log-convexity of $\set{m_t}_{t=0}^\infty$ in its entirety. 
Unfortunately the following example shows that this is 
far from the truth.
\begin{figure}[H]
\centering
\begin {tikzpicture}[]
\tikzstyle{state}=[circle, draw, inner sep=0pt, minimum size=27pt]
\node[state] (A){$0$};
\node[state] (B) [right =of A] {$1$};
\node[state] (C) [right =of B] {$2$};
\node        (E) [right =of C] {$\cdots$};
\node[state] (J) [right =of E] {$t-1$};
\node[state] (K) [right =of J] {$t$};
\path (A) edge node[below] {$\eps$} (B);
\path (B) edge node[below] {$\eps$} (C);
\path (C) edge node[below] {$\eps$} (E);
\path (E) edge node[below] {$\eps$} (J);
\path (J) edge node[below] {$\eps$} (K);
\end{tikzpicture}
\caption[An example Markov chain where our inequality is tight]{$\Omega=\set{0,1,\ldots, t}$, $S(i,i+1)=S(i+1,i)=\eps$ for $i=0,\ldots,t-1$.}
\label{fig:ex1}
\end{figure}
Consider the transition matrix $S$ on 
$\Omega=\set{0,1,\ldots,t}$ such that 
$S(i,i+1)=S(i+1,i)=\eps$ for 
$i=0,1,\ldots,t-1$ and $S(i,j)=0$ elsewhere.
Let $u$ and $v$ be the point masses respectively on 
states $0$ and $t$; namely 
$u=[1,0,\ldots,0]^\transpose$ and 
$v=[0,0,\ldots,1]^\transpose$.
We have $m_{t-2}=0$, $m_t=\eps^t$ and $m_{t+2}=t\eps^{t+2}$. 
Therefore $m_{t-2}m_{t+2} = 0 \not\ge \eps^{2t} = m_t^2$.
In this example the log-convexity breaks 
(in the strongest possible way) because the states $0$ 
and $t$ are separated by $t$ hops according to $S$ and 
the point mass at state $0$ cannot reach state $t$ 
before the $t$th time step. 

Our next theorem shows that such reachability issues 
are essentially the only way the log-convexity property 
can fail to hold:
\begin{theorem}
\label{thm:main}
For every $\eps>0$ there is a $\delta>0$ such that 
for any symmetric matrix 
$S\colon\Omega\times\Omega\to\realspos$ and unit 
vectors $u,v\colon\Omega\to\realspos$ with nonnegative 
entires, defining $m_t$ as before, we have
\begin{align}
\frac{m_{t+2}}{m_{t}^{1+2/t}}\ge
\min\set{t^{1-\eps}, \delta\frac{m_t^{1-2/t}}
{m_{t-2}}},\quad\quad\forall t\ge 2.
\label{eq:insidemain}
\end{align}
\end{theorem}
In other words, \autoref{thm:main} shows that one can recover a
truncated version of the log-convexity of
$\set{m_t}_{t=0}^\infty$ from just the coordinate-wise
nonnegativity assumption of $S,u$ and $v$.
We stress that \autoref{thm:main} is tight 
up to the appearance of $\eps$ and the choice of 
$\delta=\delta(\eps)$. A direct calculation
on \autoref{fig:ex1} for time steps $t, t+2, t+4$ 
shows that \autoref{eq:insidemain} cannot be improved to
\begin{align*}
\frac{m_{t+2}}{m_{t}^{1+2/t}}\ge
\min\set{t^{1-2/t}, 
\nparen{\frac{1+\eta}{2}}\frac{m_t^{1-2/t}}{m_{t-2}}}
\end{align*} for $\eta>0$.

\subsection{Related work on $m_t$}
\label{sec:relatedwork}
Almost simultaneously with the work of Mandel and Hughes
\cite{MandelH1958}, Mulholland and Smith also prove
\autoref{thm:blakley-roy} in \cite{MulhollandS1959} and 
moreover they characterize the equality conditions of 
the inequality. Independently, in 1965, Blakley and Roy 
\cite{BlakleyR1965} prove the same inequality and
characterize the equality conditions and 
\cite{London1966} provides an alternative proof to that 
of \cite{MulhollandS1959} in 1966.
We remark that \autoref{thm:blakley-roy} is 
most commonly referred to as the Blakley-Roy bound
or ``Sidorenko's conjecture for walks''.
Note these results show that \autoref{conj:blakley-dixon} 
is true whenever $t$ divides $k$. 
Finally in 2012, Pate shows that $m_t\ge m_1^t$ 
without the restriction $u=v$:

\begin{theorem}[Pate \cite{Pate2012}]
Let $S\colon\Omega\times\Omega\to\realspos$ be a 
symmetric matrix
with nonnegative entries and let 
$u,v\colon\Omega\to\realspos$ be
nonnegative unit vectors.
It holds that
\begin{align*}
\iprod{v}{S^{2t+1}u} \ge \iprod{v}{Su}^{2t+1},
\end{align*}
with equality if and only if 
$\iprod{v}{S^{2t+1}u} =0$ or 
$Su=\lambda v$ and 
$Sv=\lambda u$ 
for some $\lambda\in\realspos$.
\end{theorem}
This result already shows that 
\autoref{thm:blakley-dixon} is true when $t$ divides $k$ 
but such a bound does not have any implications for 
our applications in complexity theory. In \cite{ErdosS1982},
Erdős and Simonovits conjecture the following.
\begin{conjecture}
[Erdős and Simonovits \cite{ErdosS1982}, Conjecture 6]
\label{conj:erdos-simonovits}
For a graph $G=(V,E)$, let $w_k(G)$ be the number length 
$k$ walks in $G$ divided by $|V|$.
For an undirected graph $G$, we have 
$w_k(G)^t\ge w_t(G)^k$ for $k>t$ of the same parity.
\end{conjecture}
This conjecture was recalled in a recet book
\cite{Taubig2017} by Täubig as Conjecture~4.1.
Note that \autoref{conj:erdos-simonovits} is a 
specialization of \autoref{conj:blakley-dixon} to
$S$ having 0-1 entries and  
$u=\mathbf{1}/\sqrt{|V|}$ therefore our 
\autoref{thm:blakley-dixon} verifies 
\autoref{conj:erdos-simonovits} as well.


\subsection{Our results in complexity theory}
\label{sec:results-cc}
Here we list our results in complexity theory; see
\autoref{sec:applications} for the definition
of the models and the problems.
The following theorem 
(which was already known \cite{BlaisBG2014})
is a consequence of \autoref{thm:blakley-dixon} 
and uses the standard corruption technique in 
communication complexity.
\begin{theorem}
\label{thm:klogdelta}
Any two party $\delta$-error randomized protocol
solving the $k$-Hamming distance problem 
over length-$n$ strings communicates 
at least $\Omega(k\log (1/\delta))$ 
bits for $k^2\le \delta n$.
\end{theorem}
The next is our main result for the communication 
complexity of the $k$-Hamming distance problem
and is a consequence of \autoref{thm:main}. 
This result cannot be obtained by the standard 
corruption technique and
requires a suitable modification similar to 
\cite{Sherstov2012}.
\begin{theorem}
\label{thm:klogk}
Any two party $\delta$-error randomized protocol
solving the $k$-Hamming distance problem 
over length-$n$ strings communicates 
at least $\Omega(k\log (k/\delta))$ 
bits for $k^2\le \delta n$.
\end{theorem}
\begin{theorem}
\label{thm:paritysize}
Any $\delta$-error parity decision tree 
deciding the $k$-Hamming weight predicate 
over length-$n$ strings
has size $\exp \Omega\nparen{k\log (k/\delta)}$
for $k^2< \delta n$.
\end{theorem}
\begin{corollary}
\label{cor:propertytest}
Any $\delta$-error probability 
property tester for 
$k$-linearity requires 
$\Omega(k\log (k/\delta))$ queries.
\end{corollary}

Note the bound $m_k^t\ge m_t^k$ obtained in 
\cite{BlakleyD1966} under the condition $m_t\ge e^{-4t}$ 
does not have any implications for the communication 
complexity of the $k$-Hamming distance problem as our 
reduction crucially uses the fact that $u$ and $v$ are 
arbitrary, however it does lead to an $\exp\Omega(k)$ 
lower bound for the parity decision tree size of the 
$k$-Hamming weight problem when combined with our 
reduction.

\begin{remark}
Note that in \autoref{thm:blakley-dixon}, when $u=v$ and either
both $k,t$ are even or $S$ is positive semidefinite, the
summands of \autoref{eq:spectralsum} become nonnegative and the
inequality holds trivially. For our application in communication
complexity we crucially use the fact that $u$ and $v$ are
arbitrary and for our application in parity decision trees, one
can do away with $u=v$ but only at the expense of having to
choose $k,t$ odd. 
In both results the $S$ we choose has eigenvalues $1$ and $-1$ 
with equal multiplicities and therefore far away from being positive 
semidefinite. In either case, the implications of
\autoref{thm:blakley-dixon} in complexity theory follow 
from the interesting cases of this theorem.
\end{remark}

% !TeX root = heatdiscrete.tex
\section{Monotonicity of 
$t\mapsto m_{2t}^{1/(2t)}$ and 
$t\mapsto m_{2t+1}^{1/(2t+1)}$}
\label{sec:heat:monotonicity}

In this section we prove \autoref{thm:blakley-dixon}
which we restate here 
(with additional equality conditions) 
for the convenience of the reader.
Recall this theorem confirms 
\autoref{conj:blakley-dixon} 
and \autoref{conj:erdos-simonovits}.

\begingroup
\def\thetheorem{\ref{thm:blakley-dixon}}
\begin{theorem}[restated]
Let $S\colon\Omega\times\Omega\to\realspos$ be a 
symmetric matrix with nonnegative entries and 
$u,v\colon\Omega\to\realspos$ be
nonnegative unit vectors. For positive integers 
$k\ge t$ of the same
parity, we have
  \begin{align}
    \iprod{v}{S^ku}^t\ge \iprod{v}{S^tu}^k,
    \label{eq:bd-inthm}
  \end{align}
with equality if and only if $\iprod{v}{S^ku} = 0$ or
$Su =\lambda v$ and $Sv=\lambda u$ for some 
$\lambda\in\realspos$ when $t$ is odd and 
$u=v$ is an eigenvector of $S^2$ when $t$ is even.
\end{theorem}
\addtocounter{theorem}{-1}
\endgroup

We prove \autoref{thm:blakley-dixon} by an 
information theoretic argument.
Define the distributions $\mu \defeq u/\lone{u}$ and 
$\nu \defeq v/\lone{v}$.
Since either side of 
\autoref{eq:bd-inthm} is $kt$-homogeneous in $S$, we may
assume that $S$ is substochastic by scaling as needed. 
Having fixed this normalization, we view 
\autoref{eq:bd-inthm} as a statement about random walks
on $\Omega$ that start from a state sampled according to 
$\mu$ or $\nu$ and evolve according to the transition 
matrix $S$.

\subsection{Reference random walks}
\label{sec:refwalk}
\def\OC{\Omega_\circ}
Let $\OC = \Omega\cup\set{r}$ for some state $r\notin\Omega$ and 
$t$ be a positive integer. Recall that 
$\mu = u/\lone{u}$ and $\nu = v/\lone{v}$.
We start by defining random walks $F^t, B^t$ on $\OC$
that evolve in discrete time steps $-1, 0,1, \ldots, t, t+1$.

The random walk $F^t$ starts at $r$ and transitions to 
a state $x\in\Omega$ with probability $\mu(x)$ at time step $-1$.
In steps $0,1,\ldots, t-1$, the random walk proceeds according to
the transition matrix $S$. At the time step $t$,
each state $x\in\Omega$ transitions to $r$ 
with probability $\nu(x)$ and
transitions to an arbitrary state in 
$\Omega$ with probability
$1-\nu(x)$ 
(say, all of them to the same arbitrary state).
We view $F^t$ as a joint random variable
$F^t = (F^t_{-1},F^t_0,\ldots,F^t_{t+1})$,
where $F^t_i$ is the location of the walk in time step $i$.

The random walk $B^t$ proceeds backwards in time. At time 
step $t+1$ the walk $B^t$ starts at $r$ and transitions
to a state $x\in\Omega$ with probability $\nu(x)$.
In time
steps $t, t-1,\ldots, 1$, the random walk proceeds as prescribed
by $S$.
At time step $0$, each state $x\in\Omega$ transitions
to $r$ with probability $\mu(x)$
and to an arbitrary state in
$\Omega$ with probability $1-\mu(x)$.
Similarly, $B^t$ denotes
the joint random variable 
$B^t = (B^t_{-1},B^t_0,\ldots,B^t_{t+1})$,
where $B^t_i$ is the
location of the walk at time step $i$.

The following facts about $F^t$ and $B^t$ are immediate.
The random variables $F^t_{-1}$ and $B^t_{t+1}$ are fixed
to a single value $r$. The random variables $F^t$, $B^t$ are Markovian,
namely,
$\dist(F^t_i \emid F^t_{i-1},\ldots,F^t_{-1}) 
    = \dist(F^t_i \emid F^t_{i-1})$ and 
$\dist(B^t_{i-1} \emid B^t_{i},\ldots B^t_{t+1})
    = \dist(B^t_{i-1} \emid B^t_{i})$ for 
$i\in \set{0,\ldots, t+1}$.


\subsection{Random walks returning to the origin}
Assume that $\Pr[F^t_{t+1}=r]>0$.
Let $X$ be the walk $F^t$ conditioned on
$F^t_{t+1} = r$.
Note that $X$ is a random variable on the sample space
$\OC^{t+3}$. The next two lemmas explicitly calculate 
the distribution of $X$.

For a matrix $M\colon\Omega\times \Omega\to\realspos$,
functions $f,g\colon\Omega\to\realspos$, and $x,y\in \Omega$
we use the shorthands
\begin{align*}
M(f,y)&\defeq \sum_{x\in\Omega} f(x)M(x,y) = (M^\transpose f)(y)\\
M(x,g)&\defeq \sum_{y\in\Omega} M(x,y)g(y) = (Mg)(x)\\
M(f,g)&\defeq \sum_{x,y\in\Omega} f(x)M(x,y)g(y) 
    = f^\transpose M g,
\end{align*}
where the last expression in each line is understood as
a matrix vector multiplication.
\begin{lemma}
\label{lem:xdist}
Under our assumption $S^t(\mu, \nu) > 0$,
\begin{enumerate}[(i)]
\item we have
  $\Pr\sparen{X_i = x} 
      = \frac{S^i(\mu,x)S^{t-i}(x,\nu)}{S^t(\mu,\nu)}$, and
%
\item if $S^{t-i}(x,\nu)>0$, 
    we have $\Pr[X_{i+1}=y \emid X_{\le i} = x_{\le i}]
    =\frac{S(x_i,y)S^{t-i-1}(y,\nu)}{S^{t-i}(x_i,\nu)}$.
\end{enumerate}
\end{lemma}
\begin{proof}
From the definition of $F^t$ (cf.\ \autoref{sec:refwalk}), 
we have 
  \begin{align}
    \Pr[F^t_i =x] &= S^i(\mu, x)\label{teq:xda}\\
    \Pr[F^t_{t+1}=r\emid F^t_{i}=x] 
        &= S^{t-i}(x,\nu)\label{teq:xdb}\\
    \Pr[F^t_{t+1}=r] &= S^t(\mu,\nu)\label{teq:xdc}\\
    \Pr[F^{t}_{i+1}=y\text{ and }F^t_{t+1}=r\emid 
        F^{t}_i =  x] &= S(x,y)S^{t-i-1}(y,\nu)\label{teq:xde}.
  \end{align}
%TODO(saglam): some more elegant citing mechanism
Using Bayes' rule with \autoref{teq:xda},
\eqref{teq:xdb} and \eqref{teq:xdc} gives (i).
Combining \autoref{teq:xdb}, \eqref{teq:xde}
and the observation that $F^t$ is Markovian
gives (ii).
\end{proof}

With \autoref{lem:xdist} we confirm that the random variable $X=(X_{-1},X_0,\ldots,X_{t+1})$ is Markovian; 
in particular a time inhomogeneous random walk on $\OC$.
Next we observe that the random variable $B^t$ conditioned
on $B^t_{-1}=r$ is precisely $X$ also.

\begin{lemma}
\label{lem:xbackwards}
Under our assumption $S^t(\mu,\nu)>0$,
\begin{enumerate}[(i)]
  \item we have $\dist(X) = \dist(B^t\emid B^t_{-1}=r)$, and
  \item if $S^i(\mu, x)>0$, we have 
      $\Pr[X_{i-1}=y\emid X_{\ge i} = x_{\ge i}] = 
          \frac{S(x_i,y)S^{i-1}(y,\mu)}{S^i(x_i,\mu)}$.
\end{enumerate}
\end{lemma}
\begin{proof}
For any $x\in\OC^{t+3}$ with $x_{t+1} = r$,
  \begin{align*}
    \Pr[X = x] &= \frac{\mu(x_0)\prod_{i=1}^t S(x_{i-1},x_i)\nu(x_t)}
    {S^t(\mu,\nu)}\\
    &=\frac{\nu(x_t)\prod_{i=1}^t S(x_i, x_{i-1}) \mu(x_0)}
        {S^t(\mu,\nu)}
            &&\text{(as $S$ is symmetric)}\\
    &= \frac{\Pr[B^t = x]}{\Pr[B^t_{t+1} = r]}
     = \Pr[B^t = x \emid B^t_{t+1} = r]
        &&\text{(by Bayes' rule).}
  \end{align*}
This proves (i). Given (i), the proof of (ii) is 
the same as \autoref{lem:xdist}(ii).
\end{proof}


\begin{lemma}
\label{lem:xvsf}
We have $\kldiv{X}{F^t} = \kldiv{X}{B^t} = -\log S^t(\mu,\nu)$.
\end{lemma}
\begin{proof}
Recall that $\Pr[F^t_{t+1}=r] = S^t(\mu,\nu)$. 
Since $X$ is obtained from $F^t$ by
conditioning on $F^t_{t+1}=r$, the equality
criteria of \autoref{lem:klcond} are
fulfilled and thus $\kldiv{X}{F^t}=-\log S^t(\mu,\nu)$.
The derivation of $\kldiv{X}{B^t}$ is
identical as per \autoref{lem:xbackwards}(i).
\end{proof}

\subsection{Longer random walks}
\label{sec:zdef}
Let $J$ be an integer valued random variable taking
the values $\set{1,2,\ldots,t}$, each with equal probability. For each fixing
$j$ of $J$ we perform a random walk $Z\emid J=j$ on 
$\OC$ that evolves
in time steps $-1,0,1,\ldots,t,t+1,t+2,t+3$ as follows.

The random walk starts at $r$ and for each time step 
$-1\le i <j$, proceeds according to the
transition kernel $\dist(X_{i+1}\emid X_{i})$. At time step
$j$, the random walk proceeds according to 
$\dist(X_{j-1}\emid X_{j})$ and in time steps 
$j< i \le t+3$ proceeds according to the transition kernel 
$\dist(X_{i-1}\emid X_{i-2})$. We view $Z$ as a joint random
variable $Z=(Z_{-1},Z_0,\ldots,Z_{t+3})$, where $Z_i$ denotes
the location of the random walk at time step $i$.

\begin{lemma}
\label{lem:ydist}
For $-1\le i\le j$, we have 
$\dist(Z_i\emid J=j) = \dist(X_i)$ and for 
$j< i \le t + 3$,
$\dist(Z_i\emid J=j) = \dist(X_{i-2})$.
\end{lemma}
\begin{proof}
This follows from the fact that 
\begin{align*}
\dist(F^t \emid F^t_{t+1} = r) = \dist(X)
    = \dist(B^t \emid B^t_{-1} = r)
\end{align*}
and that $X$ is an actual random walk (i.e., Markovian)
on $\OC$.

To be more explicit, we have $\dist(X_i)=\dist(Z_i)$ for 
$i\le j$ since both $X$ and $Z$ start at $r$ in time step 
$-1$ and evolve according to the transition kernel
$\dist(X_{i+1}\emid X_i)$ for $i=-1,\ldots, j-1$. Since
$\dist(X_j) = \dist(Z_j)$ and $Z$ proceeds according to
$\dist(X_{i-1}\emid X_j)$ at time step $j$, by 
\autoref{lem:xbackwards}, $\dist(X_{j-1}) = \dist(Z_{j+1})$.
%
Finally in time steps $i>j$, we have $\dist(Z_i)=\dist(X_{i-2})$
since $\dist(X_{j-1}) = \dist(Z_{j+1})$ and $Z$ proceeds according
to $\dist(X_{i-1}\emid X_{i-2})$.
\end{proof}

From this we can deduce that $Z$ always
ends up in $r$ at time step $t+3$. We next
argue that if $X$ does not diverge too much from
the reference random walk $F^t$, then $Z$ does 
not diverge too much from $F^{t+2}$.

\begin{lemma}
\label{lem:zdiv}
We have
\begin{align*}
\kldiv{Z\emid J}{F^{t+2}} = \frac{t+2}{t}\kldiv{X}{F^t} 
  &- \frac{1}{t}\nparen{
    \kldiv{X_0}{F^t_0} + 
    \kldiv{X_{t+1}\emid X_{t}}{F^t_{t+1}\emid F^t_t}
  }\\
  &- \frac{1}{t}\nparen{
    \kldiv{X_t}{B^t_t} + 
    \kldiv{X_{-1}\emid X_{0}}{B^t_{-1}\emid B^t_0}
  }.
\end{align*}
\end{lemma}
\begin{proof}
For a fixing $j$ of $J$, we have 
\begin{align*}
\kldiv{Z\emid J=j}{F^{t+2}} 
    &= \sum_{i=-1}^{j-1}\kldiv{X_{i+1} \emid X_{i}}
        {F^{t+2}_{i+1} \emid F^{t+2}_{i}}
    + \kldiv{X_{j-1}\emid X_j}{F^{t+2}_{j+1}
                    \emid F^{t+2}_{j}}\\
    &\quad\quad
        + \sum_{i=j+1}^{t+2}\kldiv{X_{i-1}\emid X_{i-2}}
      {F^{t+2}_{i+1}\emid F^{t+2}_{i}},
\end{align*}
where we have used the chain rule for divergence (cf.\ 
\autoref{lem:klchain}), the fact that $Z\emid J=j$ and 
$F^{t+2}$ are Markovian and \autoref{lem:ydist}. 
Recalling that $F^t$ and $B^t$ evolve according 
to $S$ in time steps $0,1,\ldots,t-1$, and 
$\dist(F^t_{t+1}\emid F^t_t)
    =\dist(F^{t+2}_{t+3}\emid F^{t+2}_{t+2})$,
we write
\begin{align*}
\kldiv{Z\emid J=j}{F^{t+2}}
  &= \sum_{i=-1}^t \kldiv{X_{i+1}\emid X_i}{F^t_{i+1}\emid F^t_i}\\
     &\qquad\qquad
     + \kldiv{X_{j-1}\emid X_j}{B^t_{j-1}\emid B^t_{j}}
     + \kldiv{X_j\emid X_{j-1}}{F^t_{j}\emid F^t_{j-1}}\\
  &= \kldiv{X}{F^t} 
    + \kldiv{X_{j-1}\emid X_j}{B^t_{j-1}\emid B^t_{j}}
    + \kldiv{X_j\emid X_{j-1}}{F^t_{j}\emid F^t_{j-1}}
\end{align*}
again by the chain rule for divergence 
(\autoref{lem:klchain}) and the
fact that $X$ and $F^t$ are Markovian. Now taking 
an expectation over all $j\in \supp(J)$, we have
\begin{align*}
\kldiv{Z\emid J}{F^{t+2}} 
  &= \frac{1}{t}\sum_j\kldiv{Z\emid J=j}{F^{t+2}}\\
  &= \kldiv{X}{F^t} 
    + \frac{1}{t}\sum\nparen{\kldiv{X_{j-1}\emid X_j}
                            {B^t_{j-1}\emid B^t_{j}}
    + \kldiv{X_j\emid X_{j-1}}{F^t_{j}\emid F^t_{j-1}}}\\
  &= \kldiv{X}{F^t}
  + \frac{\kldiv{X}{B^t} 
      - \kldiv{X_t}{B^t_t}
      - \kldiv{X_{-1}\emid X_0}{B^t_{-1}\emid B^t_0}
    }{t}\\
  &\qquad\qquad\qquad\;+ \frac{\kldiv{X}{F^t} 
      - \kldiv{X_0}{F^t_0}
      - \kldiv{X_{t+1}\emid X_t}{F^t_{t+1}\emid F^t_t}
    }{t}.
\end{align*}
Since $\kldiv{X}{B^t} = \kldiv{X}{F^t}$ by 
\autoref{lem:xvsf}, collecting the $\kldiv{X}{F^t}$
terms we finish the proof.
\end{proof}

Finally, we lower bound the negative terms in the statement
of \autoref{lem:zdiv}.
\begin{lemma}
\label{lem:negterms}
We have
\begin{align}
\kldiv{X_0}{F^t_0}
    + \kldiv{X_{-1}\emid X_{0}}{B^t_{-1}\emid B^t_0}
    &\ge \ryent{2}{\mu} \defeq -\log \lnorm{\mu}{2}^2,
        \text{ and} \label{teq:negterm1}\\
\kldiv{X_t}{B^t_t}
    + \kldiv{X_{t+1}\emid X_{t}}{F^t_{t+1}\emid F^t_t}
    &\ge \ryent{2}{\nu} \defeq -\log \lnorm{\nu}{2}^2,
        \label{teq:negterm2}
\end{align}
where $\ryent{2}{\cdot}$ denotes the 
second order Rényi entropy. 
%If the inequalities hold
%with equality simultaneously, then $\mu = \nu$ and 
%$\mu$ is an eigenvector of $S$.
\end{lemma}
\begin{proof}
We only prove the first inequality as the
second one is symmetric. By \autoref{lem:xdist},
we have
\begin{align*}
\kldiv{X_0}{F^t_0} &= 
  \sum_{x\in \Omega}
      \frac{\mu(x)S^{t}(x,\nu)}{S^t(\mu,\nu)}
      \log \frac{S^{t}(x,\nu)}{S^t(\mu,\nu)}
          \quad\text{ and}\\
\kldiv{X_{-1}\emid X_{0}}{B^t_{-1}\emid B^t_0} &=
  \sum_{x\in \Omega}
      \frac{\mu(x)S^{t}(x,\nu)}{S^t(\mu,\nu)}
      \log\frac{1}{\mu(x)}.
\end{align*}
Let $\Psi=\supp(X_0)$. By adding the two terms we get
\begin{align}
\kldiv{X_0}{F^t_0}+
\kldiv{X_{-1}\emid X_{0}}{B^t_{-1}\emid B^t_0} &=
  -\sum_{x\in\Psi}
      \frac{\mu(x)S^{t}(x,\nu)}{S^t(\mu,\nu)}
      \log\frac{\mu(x)S^t(\mu,\nu)}{S^{t}(x,\nu)}
          \nonumber\\
  &\ge -\log \sum_{x\in\Psi}\frac%
    {\mu(x)^2 S^{t}(x,\nu)S^t(\mu,\nu)}
      {S^t(\mu,\nu)S^{t}(x,\nu)}
          \label{eq:norm-u-jens}\\
  &=   -\log \sum_{x\in\Psi} \mu(x)^2
      \nonumber\\
  &\ge -\log \sum_{x\in\Omega} \mu(x)^2\;,
      \label{eq:psivsomega}
\end{align}
where the first inequality is by concavity of 
$z\mapsto\log z$
and the second inequality is true as the 
summands are nonnegative.
\end{proof}

\subsection{Combining the inequalities}
\label{sec:bd-final}
\begin{proof}[Proof of \autoref{thm:blakley-dixon}]
Note that $Z_{-1}$ is fixed to $r$ by definition 
(cf.\ \autoref{sec:zdef}) and $Z_{t+3}$ is fixed to $r$
by \autoref{lem:ydist}. Therefore by 
\autoref{lem:klcond} we have
\begin{align}
-\log S^{t+2}(\mu,\nu)
  &\le \kldiv{Z}{F^{t+2}} \label{eq:kvk2} \\
  &= \kldiv{Z\emid J}{F^{t+2}} - \muti{J}{Z}
      \label{eq:ztozcondj}\\
  &\le \kldiv{Z\emid J}{F^{t+2}}
      \label{eq:mutrelax}\\
  &\le \frac{t+2}{t}\kldiv{X}{F^t} 
      + \frac{\log\lnorm{\mu}{2}^2 + \log\lnorm{\nu}{2}^2}
            {t}.
        \nonumber
\intertext{%
Here \autoref{eq:ztozcondj} follows from the chain rule 
for the divergence (\autoref{lem:klchain}) 
and the definition of mutual information 
(cf.\ \autoref{eq:muti-def}),
\autoref{eq:mutrelax} follows from
the nonnegativity of mutual information and
the last line follows from 
\autoref{lem:zdiv} and \autoref{lem:negterms}.
Plugging in $\kldiv{X}{F^t}=-\log S^t(\mu,\nu)$, provided by 
\autoref{lem:xvsf}, we obtain}
  -\log S^{t+2}(\mu,\nu)
    &\le -\frac{t+2}{t}\log S^t(\mu,\nu) 
         + \frac{\log\lnorm{\mu}{2}^2 + \log\lnorm{\nu}{2}^2}
            {t}.
        \nonumber
\end{align}
Arranging, we get
\begin{align*}
\lnorm{\mu}{2}^2 \lnorm{\nu}{2}^2\iprod{\nu}{S^{t+2}\mu}^t
  \ge
    \iprod{\nu}{S^t\mu}^{t+2}
\end{align*}
and substituting $\mu = u/\lone{u}$, 
$\nu = v/\lone{v}$, and recalling that $u,v$ 
are unit vectors, we obtain
\begin{align}
\iprod{v}{S^{t+2}u}^t 
&\ge \iprod{v}{S^t u}^{t+2}, \text{ i.e.,} \nonumber\\
m_{t+2} &\ge m_t^{1+2/t}\label{eq:absclaim1}.
\end{align}
By applying this inequality iteratively, we get
$\iprod{v}{S^{k}u}^t\ge\iprod{v}{S^t u}^{k}$ or
written differently
$m_k^{1/k}\ge m_t^{1/t}$
as long as $k>t$ and $k,t$ have the same parity.

Next we characterize the equality conditions of 
\autoref{eq:absclaim1}. Let us verify the
`if' direction of the statement. 
Clearly if $\iprod{v}{S^ku}=0$ then we have $\iprod{v}{S^ku} = \iprod{v}{S^tu}$ by the
first part of the theorem and the fact that 
$\iprod{v}{S^tu}\ge 0$. If $S^2u = \lambda^2 u$, 
then $m_{2t} = \lambda^{2t}$ and if $Su=\lambda v$ and $Sv=\lambda u$, then $m_{2t+1}=\lambda^{2t+1}$, therefore
in both $t$ even and $t$ odd cases 
the inequality holds with equality.

Conversely, if 
$0\neq \iprod{v}{S^{k+2}u}=\iprod{v}{S^ku}$, then the
inequalities
\eqref{eq:norm-u-jens}, 
\eqref{eq:psivsomega},
\eqref{eq:kvk2},
 and
\eqref{eq:ztozcondj} must hold with equality.
Combining the assumption that \autoref{eq:psivsomega}
and \eqref{eq:norm-u-jens} 
hold with equality with the strict concavity of
$z\mapsto \log z$ and Jensen's lemma, we get that 
$S^t\nu = \lambda_1\mu + \sigma_1$ for some $\lambda_1\le 1$ and $\sigma_1\in\reals^{\Omega}_{+}$ satisfying 
$\supp(\sigma_1)\cap\supp(\mu)=\emptyset$. This also means that $\Pr[X_0 = x] = \mu(x)^2/\lnorm{\mu}{2}^2$ for $x\in \Omega$ and a similar and symmetrical argument shows that
$\Pr[X_t = x] = \nu(x)^2/\lnorm{\nu}{2}^2$ for 
$x\in\Omega$.
Assuming \autoref{eq:ztozcondj} holds 
with equality leads to $\muti{Z}{J}=0$, which in turn
shows that $\dist(X_i)=\dist(X_{i+2})$ for 
$i=0,\ldots,t-2$. 
Let 
$X^{k+2} \defeq \nparen{F^{t+2}\emid F^{t+2}_{t+3}=r}$.
From our assumption $0\neq \iprod{v}{S^{k+2}u}=\iprod{v}{S^ku}$ we conclude that $\dist(Z)=\dist(X^{t+2})$
as $X^{t+2}$ is the minimally divergent distribution from $F^{t+2}$ among distributions on walks ending at state $r$. Since 
$\dist(X^{t+2}_2)=\dist(X^{t+2}_0)=\dist(X_0)$, and $S^t\nu=\lambda_1\mu+\sigma_1$ we get that 
$S^2\mu = \lambda_2 \mu + \sigma_2$  for some $\lambda_2\le 1$ and $\sigma_2\in\reals^{\Omega}_{+}$ satisfying 
$\supp(\sigma_2)\cap\supp(\mu)=\emptyset$. Now we will show that it must be that $\sigma_2=0$. Suppose for sake of contradiction that $\sigma_2(z)>0$ for some $z\notin \supp(\mu)$. There exists $x,y\in\Omega$ so that $\mu(x)S(x,y)S(y,z)>0$. If $y\in\supp(X_1)$ then adding to a walk $w\in \supp(X)$ with $w_1=y$ the loop $(y,z)(z,y)$
we obtain a length $t+2$ walk which is not in the support of $X^{t+2}$ as $\dist(X^{t+2}_2)=\dist(X_2)=\dist(X_0)$,
which contradicts the fact that 
$X^{t+2}$ is defined as $F^{t+2}\emid F^{t+2}_{t+3}=r$.
If on the other hand $y\notin\supp(X_1)$, 
adding the loop $(x,y)(y,x)$ to a walk with $w_0=x$ 
leads to a walk which is not in the support of $X^{t+2}$, 
which is a contradiction. Having established $S^2\mu = \lambda \mu$, we complete the proof for even $t$
by recalling that $\dist(X_0)=\dist(X_t)$ therefore $\mu=\nu$. For $t$ odd, the last argument shows that
 $S\mu = \lambda_3\nu + \sigma_3$ for some $\lambda_3\le 1$ and $\sigma_3\in\reals^{\Omega}_{+}$ satisfying 
$\supp(\sigma_3)\cap\supp(\mu)=\emptyset$. 
It remains to show that $\sigma_3=0$ 
by using the assumption that $\dist(Z)=\dist(X^{t+2})$. Suppose $\sigma_3(y)>0$ for some $y\notin\supp(\nu)$. There exists $x\in\Omega$ such that $\mu(x)S(x,y)>0$. Adding the loop $(x,y)(y,x)$ to
a walk $w$ with $w_{t-1}=x$ leads to a length $t+2$ walk
which is not in the support of $X^{t+2}$. A symmetrical
argument shows that $\lambda'\mu = S\nu$. This completes the proof.
\end{proof}

% !TeX root = heatdiscrete.tex
\section{Near log-convexity of 
         $t\mapsto m_{2t}$ and $t\mapsto m_{2t+1}$}
\label{sec:heat:logconvex}

In this section we would like to prove the following
improvement to \autoref{eq:absclaim1}: for all $\eps>0$ 
there exists a $\delta>0$ such that
\begin{align}
m_{t+2}
\ge m_{t}^{1+2/t}\cdot
\min\set{t^{1-\eps}, \ceil{\delta\frac{m_t^{1-2/t}}{m_{t-2}}}},
\quad \forall t \ge 2.
\label{eq:logconv}
\end{align}
%
Recall that in proving \autoref{eq:absclaim1}, 
in line \ref{eq:mutrelax}, we used the relaxation 
$\muti{J}{Z}\ge 0$.
Note that $J$ is uniformly distributed on $[t]$
therefore has $\log t$ bits of entropy and provided
that it is possible to infer $J$ from $Z$ 
(i.e., it is possibly to locate the time reversal we 
have inserted in $Z$)
the $\muti{J}{K}$ term appears to be large enough to 
recover the factor
\begin{align*}
\min\set{t^{1-\eps}, 
  \ceil{\delta\frac{m_t^{1-2/t}}{m_{t-2}}}}.
\end{align*}
Note moreover that intuitively we are able to 
infer $J$ from $Z$ better when 
$\frac{m_t^{1-2/t}}{m_{t-2}}$ is high, 
as in such cases on average for a time step $i\in[t]$ 
and a typical $x\sim X_i$,
the distributions $\dist(X_{i-1}\emid X_i=x)$ and
$\dist(X_{i+1}\emid X_i=x)$ should be far from each other, 
as otherwise we can argue that there should be many $t-2$ 
walks as follows. If $\dist(X_{i-1}\emid X_i=x)$ and
$\dist(X_{i+1}\emid X_i=x)$ are close to each other, 
there should be many $p\in\Omega$ which has high probability
in both
these distributions. Sample such a $p$, and 
attach to it a walk sampled from $X_{-1}X_1\ldots X_{i-1}\emid X_{i-1}=p$ 
and another walk sampled from $X_{i+1}X_{i+2}\ldots X_{t+1}\emid X_{i+1}=p$,
which leads to a length $t-2$ walk returning to the origin.
However if $m_{t-2}$ is low, this should not happen and therefore
$\dist(X_{i-1}\emid X_i=x)$ and
$\dist(X_{i+1}\emid X_i=x)$ on average should be far apart, 
which means that we can notice when we take a step backwards in time
and therefore infer $J$. In particular, \autoref{fig:ex1} 
gives such an example where $m_{t-2}=0$ and we can always
recover $J$ with certainty from a sample from $Z$: whenever we take
a step to the left, it must be that we are at time step $J$.

Given this discussion, a direct approach to proving 
\autoref{eq:logconv} appears to bound 
\begin{align}
\muti{Z}{J}\ge \log \min\set{t^{1-\eps}, 
\ceil{\delta\frac{m_t^{1-2/t}}{m_{t-2}}}}.
\label{eq:hope}
\end{align}
Unfortunately, this approach does not seem to 
work as we demonstrate with an example in the full version of this paper.
The problem here appears to be that we 
fix a single distribution $Z$ to explore the
two cases of \autoref{eq:logconv}. 
In our final approach, we pick different 
distributions depending on the case we would like 
to prove. Namely, if 
$\muti{J}{Z}\ge (1-\eps) \log t$, then carrying 
out the calculations in \autoref{eq:kvk2} through 
\ref{eq:absclaim1} with the assumption 
$\muti{J}{Z}\ge (1-\eps) \log t$, we prove 
the first case, namely $m_{t+2}\ge t^{1-\eps}m_t^{1+2/t}$ 
using the distribution given by $Z$.
If $\muti{J}{Z}< (1-\eps) \log t$ on the other hand, 
we demonstrate two new random variables $W,Y$ which are 
distributed respectively on length $t+2$ and length $t-2$ 
walks so that 
  $\kldiv{W}{F^{t+2}} + \kldiv{Y}{F^{t-2}}\le 
    -2\log S^t(\mu, \nu) - \log \delta$,
which implies that $m_{t-2}m_{t+2}\ge \delta m_t^2$.
While $W$ and $Y$ are constructed by modifying $X$ in 
suitable ways, which is how $Z$ was constructed also,
we do so with the hindsight of 
having inspected what causes $\muti{J}{Z}$ to be smaller
than $(1-\eps)\log t$. It is precisely this adaptivity
which enables this approach to overcome the difficulties
encountered by the one suggested in 
\autoref{eq:hope}.

If $\muti{J}{Z}\ge (1-\eps) \log t$, by plugging this into 
\autoref{eq:mutrelax} and carrying out the 
following calculations, 
we get $m_{t+2} \ge t^{1-\eps}\cdot m_t^{1+2/t}$.
Therefore it remains to show there exists a $\delta>0$
such that assuming $\muti{J}{Z} < (1-\eps) \log t$,
we have
$m_{t+2}m_{t-2}\ge \delta m_t^2$.
To do so, we will demonstrate random variables 
$W$ and $Y$
supported on walks that start from $r\in \OC$
and return to $r$ 
after spending respectively $t+2$ and $t-2$ time 
steps in 
$\Omega$ such that $\kldiv{W}{F^{t+2}} + 
\kldiv{Y}{F^{t-2}}\le -2\log S^t(\mu, \nu) - 
\log \delta$. 
Notice that by \autoref{lem:jensen} this indeed implies
that $m_{t+2}m_{t-2}\ge \delta m_t^2$.
The random variables $W$ and $Y$ will be mixtures of 
$\Theta(t)$ random walks, in particular, they are not 
Markovian in general.

For brevity let us set $\mu_i^x \defeq \dist(X_{i-1}\emid X_i = x)$ and 
$\nu_i^x \defeq \dist(X_{i+1}\emid X_i=x)$.
Let $U$ be the unary encoding of $J$: a length $t$ bit
vector of which only the $J$th coordinate is set.
First we would like to understand 
the contribution of each bit of $U$ to 
$\muti{Z}{J}=\muti{Z}{U}$.
Using the chain rule, we write
\begin{align}
(1-\eps)\log t&>   \muti{U}{Z} \nonumber\\
              &=   \sum_{i=1}^t \muti{U_i}{Z\emid U_{<i}}
                    \label{eq:t1-chain}\\
              &=   \sum_{i=1}^t \frac{t-i+1}{t}
                   \muti{U_i}{Z\emid U_{<i}=0}
                   \label{eq:t1-condition}\\
           &\ge \sum_{i=1}^t \frac{t-i+1}{t}
                \muti{U_i}{Z_iZ_{i+1}\emid U_{<i}=0}
                \label{eq:t1-data}\\
           &=   \sum_{i=1}^t\frac{1}{t}
                \E_{x\sim X_i}\kldiv{\mu_i^x}
                {\lambda_i \mu_i^x + (1-\lambda_i)\nu^x_i}
                \nonumber\\
           &\quad\quad\quad\quad\quad
                + \sum_{i=1}^t\frac{t-i}{t}
                \E_{x\sim X_i}\kldiv{\nu_i^x} 
                {\lambda_i\mu_i^x + (1-\lambda_i)\nu^x_i}
                \label{eq:t1-split}
\end{align}
where we set $\lambda_i \defeq 1/(t-i+1)$, which is
the probability that $U_i=1\emid U_{<i}=0$.
Here, \autoref{eq:t1-chain} follows from the 
chain rule, \autoref{eq:t1-condition} is 
true because if $U_{<i}\neq 0$ then $U_i=0$ 
(as $U$ has a single coordinate that is one) 
and consequently the mutual information is
zero, and \autoref{eq:t1-data} is the data 
processing inequality. Next we lower bound 
\autoref{eq:t1-split} by its first term 
(which is valid since $\mu^x_i, \nu^x_i$ 
are distributions hence the second term of 
\autoref{eq:t1-split} is nonnegative), obtaining
\begin{align}
       (1-\eps) \log t &> \E_{i\sim J}\E_{x\sim X_i}
                \kldiv{\mu_i^x}
                      {\lambda_i \mu_i^x 
                      + (1-\lambda_i)\nu^x_i}\,.
                \label{eq:mut-inf}
\end{align}

To simplify the presentation, here we only 
provide the proof of \autoref{thm:main} for 
$\eps >7/8$ which demonstrates the ideas in 
their simplest form. This bound already implies all 
our results in complexity theory, with a constant
factor loss of no more than 8.
The proof for any $\eps>0$ can be found in the 
full version of this paper.

\subsection{The bound for $\eps> 7/8$}
If we condition on the event $i\in\set{1,\ldots,\lceil t/2 \rceil}$, this expectation
increases by a factor of at most 2; namely
\begin{align*}
\E_{i\sim \sparen{t/2}}
  \E_{x\sim X_i}
      \kldiv{\mu_i^x}
      {\lambda_i \mu_i^x 
          + (1-\lambda_i)\nu^x_i} < 2(1-\eps)\log t.
\end{align*}
By Markov's inequality
\begin{align*}
\Pr_{i\sim\sparen{t/2},x\sim X_i}
  \sparen{\kldiv{\mu_i^x}
      {\lambda_i \mu_i^x 
          + (1-\lambda_i)\nu^x_i} \ge 8(1-\eps)\log t} < 1/4,
\end{align*}
so it follows that there is a set 
$T\subseteq\sparen{\lceil t/2\rceil}$ of size
at least $\floor{t/4}$ such that if $i\in T$ we have
\begin{align}
\Pr_{x\sim X_i}
  \sparen{\kldiv{\mu_i^x}
      {\lambda_i \mu_i^x 
          + (1-\lambda_i)\nu_i^x} \ge 8(1-\eps)\log t} < 1/2.
          \label{eq:halfp}
\end{align}
For each $i\in T$ let $X'_i$ be the random variable 
obtained from $X_i$ by conditioning on those 
$x\in \supp(X_i)$ satisfying $\kldiv{\mu_i^x}
{\lambda_i \mu_i^x + (1-\lambda_i)\nu^x_i}<8(1-\eps)\log t$.
Furthermore, for each $i\in T$ and $x\in\supp(X'_i)$,
we construct distributions
$\pi_i^x\colon \Omega\to\realspos$ to be specified later.
Let $P_i$ be sampled by $x\sim X'_i$ first and then picking
$p\sim \pi_i^x$. 

\subsection{The distributions $W$ and $Y$}
Let $K$ be an integer sampled uniformly at
random from the set $T$ (constructed in 
the previous section).
For each fixing $k$ of $K$, the random variables $W\emid K=k$
and $Y\emid K=k$ are random walks 
(i.e., they are Markovian) 
constructed as follows. We first pick $x,p\sim X'_kP_k$.
The walk $Y\emid K=k$ is generated by concatenating a sample
from $X_{-1}X_0\ldots X_{k-1}\emid X_{k-1} = p$ and 
an independent sample from $X_{k+1}\ldots X_{t+1}\emid X_{k+1} = p$.
The walk $W$ is generated by concatenating a sample from 
$X_{-1}X_0\ldots X_{k}\emid X_{k} = x$, the path
$(x, p)$ and $(p,x')$ for an independent sample 
$x'\sim\nparen{X'_k\emid P_k = p}$ and 
an independent sample from $X_{k}\ldots X_{t+1}\emid X_{k} = x'$.

For $k\in T$ we define another random walk 
$\check{X}^k = 
(\check{X}^k_{-1},\ldots,\check{X}^k_{t+1})$, 
only to be used in the analysis of
$W$ and $Y$. We sample $x\sim X'_k$
and set $\check{X}^k_k = x$. We pick the rest of the coordinates of $\check{X}^k$ according to the distribution $X\emid X_k = x$. Note that for any $k\in T$, we have
\begin{align*}
\kldivs{\check{X}^k}{X} = \kldiv{X'_k}{X_k}\le 1
\end{align*} by \autoref{eq:halfp}
and \autoref{lem:klcond} and the fact that both $X$ and 
$\check{X}^k$ are Markovian.

\begin{lemma}
\label{lem:wy}
We have
\begin{align*}
\kldiv{W\emid K=k}{F^{t+2}} +& \kldiv{Y\emid K=k}{F^{t-2}}\\
&\le -2\log S^t(\mu,\nu) + 2 + \E_{x\sim X'_k}\kldiv{\pi_k^x}{\mu_k^x} + 
        \E_{x\sim X'_k}\kldiv{\pi_k^x}{\nu_k^x}.
\end{align*}
\end{lemma}
\begin{proof}
We have
\begin{align*}
\kldiv{W\emid K = k}{F^{t+2}} &= \kldivs{\check{X}^k}{F^t} + 
  \kldiv{P_k\emid X'_k}{F_{k+1}\emid F_{k}}
  +\kldiv{X'_k\emid P_k}{F_{k+1}\emid F_{k}}
\end{align*}
and further
\begin{align*}
\kldiv{Y\emid K = k}{F^{t-2}} +& 
  \kldiv{P_k\emid X'_k}{F_{k+1}\emid F_{k}}
  +\kldiv{X'_k\emid P_k}{F_{k+1}\emid F_{k}}
  \\&= \kldivs{\check{X}^k}{F^t}+\E_{x\sim X'_k}
      \kldiv{\pi_k^x}{\mu_k^x} + 
      \E_{x\sim X'_k}\kldiv{\pi_k^x}{\nu_k^x}.
\end{align*}
Summing up the two inequalities and substituting 
$\kldivs{\check{X}^k}{X}\le 1$ we get the result.
\end{proof}
At this point, in light of \autoref{lem:wy},
we could pick each $\pi_k^x$ so that it minimizes
$\kldiv{\pi_k^x}{\mu_k^x} + \kldiv{\pi_k^x}{\nu_k^x}$:
the unique minimizer is given by 
$\pi_k^x = \sqrt{\mu_k^x\nu_k^x} / 
\iprod{\sqrt{\mu_k^x}}{\sqrt{\nu_k^x}}$. 
However doing so leads to $W,Y$ which 
diverge from the $F$ walk by more than a constant,
and therefore is not good enough for our needs.
To obtain better random variables $W$ and $Y$, 
we crucially use the fact that 
$W$ is a mixture of $\Theta(t)$ random walks. Namely, if
we consider the entropy coming from the $\muti{W}{K}$ term
also, a better strategy for picking the distributions 
$\pi_k^x$ becomes available.
By contrast, we do not use the fact that $Y$ is a mixture
and, in fact, it can be replaced by $Y\emid K=k_0$ where
$k_0 = \arg\min_k\kldiv{Y\emid K = k}{F^{t-2}}$, however
the averaged quantity $\kldiv{Y\emid K}{F^{t-2}}$ is
far more convenient to work with.

\subsection{The contribution of $\muti{K}{W}$}
Similar to \autoref{eq:mut-inf}, we would like to understand
the contribution of each time step $t\in T$ to 
$\muti{K}{W}$. Let $V$ be the unary encoding of $K$:
a length $t$ bit vector of which only the $V$th 
coordinate is set. Using the chain rule for 
mutual information
\begin{align*}
\muti{W}{V} &=   \sum_{i\in T}\muti{V_i}{W\emid V_{<i}}\\
            &\ge \E_{k\sim K} \E_{x\sim X'_k} 
                 \kldiv{\pi_k^x}
           {\eta_i \pi_k^x + (1-\eta_i)\widetilde{v}_k^x},
\end{align*}
where $\eta_k = 1/\rank_{T}(k)$ and 
$\widetilde{v}_k^x 
    \defeq \E_{j > k : j\in T} 
    \dist(\check{X}^j_{k+1}\emid \check{X}^j_k = x)$.
Here $\rank_T(i)$ denotes the position of 
$i\in T$ when the elements of $T$ are sorted in decreasing order.
By \autoref{eq:halfp}, and the definition of
$X'_k$, we have
$\widetilde{v}_k^x(y)\le 2\nu_k^x(y)$ for all $y\in \Omega$.
Therefore we conclude that 
\begin{align}
\muti{W}{K} \ge \E_{k\sim K} \E_{x\sim X'_k} 
                 \kldiv{\pi_k^x}
           {\eta_k \pi_k^x + 2(1-\eta_k)\nu_k^x}.\label{eq:muti-nu}
\end{align}
Note in the above divergence expression the 
reference measure is not a probability distribution, 
which our definition permits (cf.\ \autoref{eq:kldiv}).

Recall our goal in this section is to upper bound 
$\kldiv{W}{F^{t+2}} + \kldiv{Y}{F^{t-2}} 
+ 2\log S^t(\mu, \nu)$ by $\log 1/\delta$. 
Let us write
\begin{align}
\kldiv{W}{F^{t+2}} +& \kldiv{Y}{F^{t-2}} 
+ 2\log S^t(\mu, \nu) \nonumber \\
&\le \kldiv{W\emid K}{F^{t+2}} 
  + \kldiv{Y\emid K}{F^{t-2}} 
  - \muti{K}{W} + 2\log S^t(\mu, \nu)
    \nonumber\\
&\le 2 + \E_{k\sim K, x\sim X'_k}
  \kldiv{\pi^k_x}{\mu^k_x} + 
  \kldiv{\pi^k_x}{\nu^k_x} - \muti{K}{W}
        \nonumber\\
&\le 2 + \E_{k\sim K, x\sim X'_k} \E_{y\sim \pi_k^x} \log
\frac{\eta_k \pi_k^x(y)^2 + 2(1-\eta_k)\nu_k^x(y)\pi_k^x(y)}
{\mu_k^x(y)\nu_k^x(y)}\label{eq:div-min},
\end{align}
where the second inequality follows from 
\autoref{lem:wy} and the last inequality is 
obtained by plugging in \autoref{eq:muti-nu}.
Note that the function $z\mapsto z \log(az^2 + bz)$ is 
strictly convex
in $\realspos$ whenever $ab > 0$, therefore 
for each $k,x$ there is a unique minimizer $(\pi_k^x)^*$
of \autoref{eq:div-min},
which can be calculated, say, using Lagrange multipliers. 
However, instead of the minimizer, we work with a simple approximation of it.
For each $k\in T$ and $x\in \supp(X'_k)$, we let
\begin{align*}
\Psi_k^x \defeq \setbuilder{y\in\Omega}
{\nu_k^x(y)\ge \frac{\lambda_k}{1-\lambda_k}\mu_k^x(y)}.
\end{align*}
By definition of $X'_k$, we have
$\kldiv{\mu_i^x}
      {\lambda_i \mu_i^x 
          + (1-\lambda_i)\nu^x_i} < 8(1-\eps)\log t$.
Let $\gamma = 1- 8(1-\eps)$, which is positive by our assumption
$\eps>7/8$.
By Markov's inequality, and the fact that 
$\lambda_k \le 2/t$,
we get
\begin{align*}
\mu_k^x(\Psi_k^x)\ge \gamma
\end{align*}
for large enough $t$.
Let $\pi_k^x$ be $\mu_k^x \emid \Psi_k^x$, namely we have
$\pi_k^x(y) = \mu_k^x(y) / \mu_k^x(\Psi_k^x)$ if 
$y\in\Psi_k^x$, and $\pi_k^x(y) = 0$ otherwise.
Continuing from \autoref{eq:div-min}, we have
\begin{align}
 &\le 2 +  \E_{k\sim K, x\sim X'_k} \E_{y\sim \pi_k^x} \log
\frac{\eta_k \pi_k^x(y)^2 + 2(1-\eta_k)\nu_k^x(y)\pi_k^x(y)}
{\mu_k^x(y)\nu_k^x(y)}\nonumber\\
 &\le 2+ 
 \E_{k\sim K} \log
\nparen{\frac{\eta_k(1-\lambda_k)}{\lambda_k \gamma^2}+\frac{2}{\gamma}},
\label{eq:setsh}
\end{align}
where the second inequality is true by definition of
$\Psi^x_k$ and $\pi^x_k$. 
Now we argue that the expectation term in \autoref{eq:setsh} is maximized
when $T$ is the set containing the smallest $|T|$ elements of 
$[\ceil{t/2}]$. To see this suppose there is an $i\notin T$
which is smaller than the maximum element of $T$.
Let $j$ be the smallest item in $T$ which is greater than $i$. 
We see that the expectation term
increases if we replace $T$ by $T\setminus\set{j}\cup\set{i}$
as $\log\nparen{\frac{C(1-\lambda_k)}
{\lambda_k \gamma^2}+\frac{2}{\gamma}}$ is decreasing in $k$ 
and the ranks
do not change after swapping $j$ with $i$.
Therefore, 
\begin{align}
\kldiv{W}{F^{t+2}} + &\kldiv{Y}{F^{t-2}} + 2\log S^t(\mu, \nu)
\nonumber \\
&\le 2+
\log\nparen{\prod_{i=1}^{|T|}\frac{t/2 + 3i}{i \gamma^2}}^{1/|T|}
\nonumber \\
&=\log\frac{12}{\gamma^2} + 
\log\nparen{\prod_{i=1}^{|T|}\frac{t/6 + i}{i}}^{1/|T|}
\nonumber \\
&\le \log\frac{12}{\gamma^2} + \log\binom{2|T|}{|T|}^{1/|T|}
\label{eq:binomub}\\
&\le \log\frac{48}{\gamma^2},
\end{align}
where the $\binom{2|T|}{|T|}$ is the middle binomial coefficient,
in the second inequality we use the fact $|T|>t/6$,
and the last inequality is true as $\smash{\binom{2n}{n}< 2^{2n}}$.
Therefore it is enough to choose $\eps > 7/8$ and 
$\delta\le \frac{\nparen{1-8(1-\eps)}^2}{48} = \frac{4}{3}(\eps- \sfrac{7}{8})^2$. We have established
the following.

\begingroup
\def\thetheorem{\ref{thm:main}}
\begin{theorem}[restated]
For any $\eps>7/8$ there is a $\delta>0$ such that
$m_{t+2}\ge t^{1-\eps} m_t^{1+2/t}$ unless 
$m_{t+2}m_{t-2}\ge \delta m_t^2$.
\end{theorem}
\addtocounter{theorem}{-1}
\endgroup

\section{Chapter notes}
\label{sec:heat:notes}

The results of this chapter were obtained in our FOCS 2018 paper \cite{Saglam2018}.


% k-Hamming distance
% !TeX root = thesis.tex
\chapter{The $k$-Hamming distance problem}
\label{sec:ham:intro}

In this section we study the $k$-Hamming distance
problem in communication complexity and
it's incarnations in related computation models.

\section{Communication complexity}
In a two player communication problem the players, 
named Alice and Bob, receive separate inputs, 
respectively $x\in \mathcal{X}$ and $y\in \mathcal{Y}$, 
and they communicate in order to compute the value 
$f(x,y)$ of a function
$f\colon \mathcal{X}\times \mathcal{Y}\to\binary$ 
(known to both players). 
In an $r$-round protocol, the players can take at most 
$r$ turns alternately sending each other a message 
(that is, a bit string) and the last player to receive 
a message declares the output of the protocol.
A protocol can be {\em deterministic} or {\em randomized}; 
in the latter case the players can base their actions on 
a common random source and we measure the 
{\em error probability}: the maximum over inputs 
$(x,y)\in \mathcal{X}\times\mathcal{Y}$, of the probability 
that the output of the protocol differs from $f(x,y)$. 
The {\em communication cost} of a protocol is the maximum, 
over the inputs and the random string, of the total number 
of bits sent between the players.
For a function 
$f\colon \mathcal{X}\times\mathcal{Y}\to\binary$, 
an integer $r$ and $\delta\in[0,1]$, we denote by 
$R^r_{\delta}(f)$ the minimum over all protocols
for $f$ having $r$-rounds and error probability at most 
$\delta$, of the communication cost incurred. We define 
$R_{\delta}(f)$ similarly, but we take the maximum over 
$\delta$-error protocols with no restriction on the number 
of rounds it uses.

In the $k$-Hamming distance problem, denoted $\Ham^n_k$,
the players receive length-$n$ bit strings, respectively 
$x,y\in\cube$, and are required determine if 
$\lone{x-y}\le k$ or not.
There is a well known one-round communication protocol
which accomplishes this with error probability $\delta$ 
by communicating $O(k\log\nparen{k/\delta})$ bits.

\begin{theorem}
[e.g., Huang, Shi, Zhang and Zhu \cite{HuangSZZ2006}]
\label{thm:ub}
It holds that $$R^1_\delta(\Ham^n_k) = 
O(\min\set{k\log\nparen{k/\delta}, k\log (n/k)}).$$
\end{theorem}

Highly related to the $\Ham^n_k$ is the $k$-disjointness 
problem $\Disj^n_k$, wherein the players each receive a 
$k$-subset of $[n]$ and their goal is to determine if their 
sets intersect.
Notice that $\Disj^n_k$ can be seen as a promise version of
$\Ham^n_{2k-2}$ where each player is guaranteed to have a string
with Hamming weight $k$: the sets are disjoint if and only if the
Hamming distance between the characteristic vectors of the sets
is more than $2k-2$. Therefore any upper bound for the $\Ham^n_k$ 
carries over to $\Disj^n_k$ and any lower bound for $\Disj^n_k$
carries over to $\Ham^n_k$.
Around 1993, 
Håstad and Wigderson \cite{HastadW2007} showed that
there is a more efficient protocol for $\Disj^n_k$ than that implied 
by \autoref{thm:ub}, which communicates only $O(k)$ bits, but over
$O(\log k)$ rounds.

On the lower bounds side, the result of \cite{KalyanasundaramS1992}
implies that $\Omega(k)$ bits is needed for these problems even
if one uses arbitrarily large number of round protocols.
In \cite{BuhrmanGMW2012} it was shown that
any 1-round protocol for $\Disj^n_k$ needs to communicate
at least $\Omega(k \log k)$ bits when $k^2<n$
(this result was proven later in \cite{DasguptaKS12} also).
In Theorem~3.2 of \cite{Saglam2011}, an $\Omega(k\log(1/\delta))$
bound for $1$-round complexity of $\Disj^n_k$ was shown even 
when Bob receives just one element (i.e., the indexing problem) 
for $k<\delta n$ and a 
slightly more general result was shown in \cite{JayramW2011}.
Finally, in \cite{SaglamT2013} the communication complexity
of $\Disj^n_k$ was settled as presented in \autoref{sec:ham} 
of this thesis.
\begin{align*}
R^r_{1/3}(\Disj^n_k)=\Theta(k\log^{(r)}k)
\end{align*}
for $1\le r \le \log^*k$ and $k^2<n$.
Their upper bound solves the disjointness problem with 
error probability at most $1/\exp k + 1/\exp^{(r)}(c\log^{(r)}k)$ 
for any $c>1$ by communicating
$O(k \log^{(r)} k)$ bits over $r$ rounds. In fact bulk of the bits
is sent in the first round and the rest of the rounds
amount to an $O(k)$ bits of communication. Taking $r=\log^* k$,
this leads to an $O(k)$ bits protocol with error probability 
that is exponentially small in $k$.
Their lower bound shows that at least one
message of size 
$\Omega(k\log^{(r)} k)$ bits needs to be sent by any 
$r$-round protocol, even if it has error probability $1/3$.
Prior to this work, this lower bound provided the strongest lower
bound for $\Ham^n_k$ also, along with the incomparable bound of 
$\Omega(k\log (1/\delta))$ due to \cite{BlaisBG2014}
which holds for any number of rounds, which we discuss shortly.

% !TeX root = thesis.tex
\begin{table}[]
\def\arraystretch{1.2}
\makeatletter
\newcommand{\mleftarrow}{\rotatebox[origin=c]{90}{appl.}\bBigg@{2}\downarrow}
\newcommand{\mrightarro}{\rotatebox[origin=c]{90}{applies}\bBigg@{4}\uparrow}
\makeatother
\begin{tabular}{lllllr}
{\bf Problem}			& {\bf Upper bound}		& {\bf Rounds}			& {\bf Error}    			& {\bf Lower bound}		& {\bf Reference} \\\hline
\multirow{3}{*}{$\Ham^n_k$}	& $O(k\log (k/\delta))$		& $1$				& $\delta$				&				& Folklore, \cite{HuangSZZ2006}\\
				& \multirow{2}{*}{$\mleftarrow$}& any				& $\delta$				& $\Omega(k\log(1/\delta))$	& \cite{BlaisBG2014}\\
				&				& any				& $\delta$				& $\Omega(k\log(k/\delta))$ 	& This work\\
\hline
\multirow{8}{*}{$\Disj^n_k$}	& $O(k\log (k/\delta))$		& $1$				& $\delta$				& \multirow{4}{*}{$\mrightarro$}& Folklore          \\
				& $O(k)$			& $O(\log k)$			& $1/3$					&				& \cite{HastadW2007}\\
                              	& $O(k\log^{(r)}k)$		& $r$				& $1/\exp^{(r)}\nparen{c\log^{(r)} k}$	&				& \cite{SaglamT2013}\\
			      	& $O(k)$			& $\log^* k$			& $1/\exp k$				&				& \cite{SaglamT2013}\\
				&				& r				& $1/3$					& $\Omega(k \log^{(r)} k)$	& \cite{SaglamT2013}\\
				&				& 1				& $1/3$					& $\Omega(k\log k)$ 		& \cite{BuhrmanGMW2012, DasguptaKS12}\\
				&				& 1				& $\delta$				& $\Omega(k \log(1/\delta))$	& \cite{Saglam2011,JayramW2011}\\
				&				& any				& $1/3$					& $\Omega(k)$			& \cite{KalyanasundaramS1992}\\
\hline
\end{tabular}
\caption[Comparison of $\Ham^n_k$ and $\Disj^n_k$ bounds]{Known bounds for $\Ham^n_k$ and $\Disj^n_k$. 
Each upper bound for $\Ham^{n}_k$ applies to $\Disj^n_k$ 
and each lower bound for $\Disj^n_k$ applies to $\Ham^n_k$.}
\label{table:hamvsdisj}
\end{table}


To summarize the above results, the 1-round communication 
complexity of both $\Disj^n_k$ and $\Ham^n_k$ is 
$\Theta(k\log(k/\delta))$ by 
\cite{BuhrmanGMW2012, Saglam2011, JayramW2011} and 
\cite{HuangSZZ2006}. We know that $\Disj^n_k$ can be solved 
much more efficiently if one is allowed multiple rounds:
firstly the $\log k$ factor can be removed \cite{HastadW2007} 
and secondly the error probability can be brought down to 
$\exp(-k)$, by using no more than $\log^* k$ rounds 
\cite{SaglamT2013}. It is an interesting question whether 
similar efficiency improvements can be obtained for $\Ham^n_k$ 
also, by using multiple rounds.
The first separation of $\Disj^n_k$ and $\Ham^n_k$ was proven
in \cite{BlaisBG2014}, which shows that $\Omega(k\log(1/\delta))$
lower bound holds for any protocol solving $\Ham^n_k$. 
Therefore in $\Ham^n_k$, we get no improvements in error probability
by interactive communication. It remained an open 
question whether {\em any} improvement can be made 
at all to the 1-round protocol by communicating interactively.
In this work we answer this question negatively: 

\begingroup
\def\thetheorem{\ref{thm:klogk}}
\begin{theorem}[restated]
For $k^2<\delta n$ we have 
$R_{\delta}(\Ham^n_k) = \Omega(k\log (k/\delta))$.
The bound applies even to protocols that may
output an arbitrary answer when $\lone{x-y} \notin \set{k-2, k,k+2}$.
\end{theorem}
\addtocounter{theorem}{-1}
\endgroup
Before we proceed with proving \autoref{thm:klogk}, let us
first warm up by showing that \autoref{thm:blakley-dixon} 
implies an  $\Omega(k\log(1/\delta))$ lower bound on 
$R_\delta(\Ham^n_k)$.
To do so, let us review the so called {\em corruption bound}
method. Let  $f\colon\mathcal{X}\times\mathcal{Y}\to\binary$ 
be the function the players would like to compute with 
Alice having received $x\in \mathcal{X}$ and Bob $y\in \mathcal{Y}$. 
For a protocol $P$ for $f$, define the matrix 
$A_P\colon\mathcal{X}\times\mathcal{Y}\to[0,1]$ 
such that $A_P(x,y)$ is the probability
that the protocol outputs 1 on input $(x,y)$.
It is well known and not difficult to see that
if $P$ has communication cost $c$, then $A_P$ is the
average of matrices each of which is the sum of at most
$2^c$ rank 1 matrices $u v^\transpose$ with 
$u\in \binary^\mathcal{X}$ and $v\in\binary^\mathcal{Y}$.
Therefore to show the communication cost of a protocol
$P$ is more than $c$, it suffices to argue $A_P$ lies
outside $2^c$ times the polytope
\begin{align*}
\mathcal{T}\defeq 
\conv\set{uv^\transpose\mid 
u\in\binary^\mathcal{X}, v\in\binary^\mathcal{Y}},
\end{align*}
where $\conv$ denotes the convex hull. By convexity, 
$A_p$ lies outside of $2^c\mathcal{T}$ if and only if there 
is a hyperplane (with normal $H$) separating the two; 
namely that $\iprod{A_P}{H}>2^c\iprod{R}{H}$ for all 
vertices $R$ of the polytope $\mathcal{T}$.

Let $\mu_k\colon\cube\times\cube\to\realspos$ be the distribution
on pairs $(x,y)$ obtained as follows. Sample $x$ uniformly at random
and obtain $y$ by flipping $k$ coordinates of $x$ chosen uniformly 
at random and with replacement (here if a coordinate gets flipped 
twice it reverts back to its initial value).
\begingroup
\def\thetheorem{\ref{thm:klogdelta}}
\begin{theorem}[restated]
For $k^2<\delta n$ we have 
$R_{\delta}(\Ham^n_k) = \Omega(k\log (1/\delta))$.
The bound applies even to protocols that may
output an arbitrary answer when $\lone{x-y} \notin \set{k,k+2}$.
\end{theorem}
\addtocounter{theorem}{-1}
\endgroup
\begin{proof}
Suppose we have a randomized protocol for $\Ham^n_k$ with 
error probability $\delta$. Form the matrix $A$, where
$A(x,y)$ is the probability that the protocol reports
$\lone{x-y}\le k$ on input $(x,y)$.

Set $H = \mu_k - \mu_{k+2}/(3\delta)$. Let us first argue that
$\iprod{A}{H}\ge 1/3$. Note that when we sample $k$ elements 
from $[n]$ uniformly at random, by a union bound, the probability
that there is a collision is at most $\binom{k}{2}/n$, therefore 
$\mu_k$ chooses a pair $(x,y)$ at distance $k$ with probability 
at least $1-\binom{k}{2}/n$.
Hence,
$\iprod{A}{\mu_k} 
   \ge (1-\delta)(1-\binom{k}{2}/n)
     > 1-3\delta/2$,
where in the last step we used $k^2< \delta n$.
Similarly, we have $\iprod{A}{\mu_{k+2}} \le 
                    \delta + \binom{k+2}{2}/n \le 3\delta/2$,
thus it follows that $\iprod{A}{H}\ge 1/3$ for $\delta \le 1/9$.

Next we argue that $\iprod{R}{H}< (3\delta)^{k/2}$ for any 
$R=uv^\transpose$ with $u,v\in\cube$. If 
$\iprod{R}{\mu_k}<(3\delta)^{k/2}$, we are done as 
$\iprod{R}{\mu_{k+2}}\ge 0$ and is a negative term in 
$\iprod{R}{H}$. If $\iprod{R}{\mu_k}\ge (3\delta)^{k/2}$
on the other hand,
observing $\iprod{R}{\mu_k}=\iprod{v}{W^k u}/2^n$,
where $W$ is the normalized adjacency matrix of the Hamming cube,
we have by \autoref{thm:blakley-dixon}
\begin{align*}
\nparen{\frac{\lnorm{u}{2}\lnorm{v}{2}}{2^n}}^{2/k}
\iprod{v}{W^{k+2}u} \ge 
\iprod{v}{W^{k}u}^{1+2/k}.
\end{align*}
Note $\lnorm{u}{2}\lnorm{v}{2}\le 2^n$ since $u,v$ are 0-1 vectors,
therefore 
$\iprod{R}{\mu_{k+2}}\ge 3\delta\iprod{R}{\mu_{k}}$
and hence $\iprod{R}{H}\le 0$. In either case we have shown
that $\iprod{R}{H} < (3\delta)^{k/2}$. This implies an 
$\log((3\delta)^{-k/2} / 3) = \Omega(k\log(1/\delta))$ bits 
lower bound on $R_\delta(\Ham^n_k)$.
\end{proof}
Interestingly, \autoref{thm:klogk} cannot be proved by a direct 
application of the corruption method described above. 
If we assume that the protocol
is supposed to output 1 on inputs $\lone{x-y}\le k$, then there are 
vertices of the polytope $\mathcal{T}$ for which 
the $\Omega(k\log(1/\delta))$ bound of 
\autoref{thm:klogdelta} is tight. If we assume that the protocol
is supposed to output 1 on inputs $\lone{x-y}> k$ on the other
hand, no bound above $\Omega(k)$ can be obtained, as there are
vertices for which this is tight.
If we insist however that the protocol outputs 1 for $\lone{x-y}=k$
and 0 for $\lone{x-y}\in\set{k-2,k+2}$ then a protocol with cost
smaller than $O(k\log (k/\delta))$
would be in violation of the near log-convexity principle we 
established in \autoref{thm:main} as we argue next.
Of course, if we had a $\delta$-error randomized protocol 
$P$ for 
$\Ham^{n+2}_k$ outputting 1 when $\lone{x-y}\in\set{k-2,k}$ 
and 0 if $\lone{x-y}=k+2$ 
(but without any guarantees for other types of inputs), then 
given inputs $a,b\in\cube$ Alice and Bob can run $P$
(say, in parallel)
on instances $(00a, 00b)$ and $(00a, 11b)$ and declare 
$\lone{a-b}=k$ if $P$ returns 1 on $(00a, 00b)$ and 
0 on $(00a, 11b)$.
This would lead to a protocol with twice the error probability
and communication cost of $P$, deciding between $\lone{a-b}=k$,
$\lone{a-b}=k-2$ and $\lone{a-b}=k+2$. The table below shows that
$P$ outputting 1 on $(00a, 00b)$ and 0 on $(00a, 11b)$ implies
$\lone{a-b}=k$ or at least one invocation of $P$ erred.

\begin{center}
\def\arraystretch{1.2}
\begin{tabular}{rccc}
Input       & $k-2$ & $k$  & $k+2$ \\\hline
$(00a,00b)$ &   1   &  1   &  0  \\
$(00a,11b)$ &   1   &  0   &  ? 
\end{tabular}
\end{center}

\begin{proof}[Proof of \autoref{thm:klogk}]
Suppose we have a $\delta$-error randomized protocol that 
outputs 1 when $\lone{x-y} = k$ and 0 when 
$\lone{x-y} = k-2$ or $\lone{x-y} = k+2$.
 
Form the matrix $A$, where $A(x,y)$ is the probability 
that the protocol reports that $\lone{x-y}= k$ on 
input $(x,y)$. Let $\alpha_1,\alpha_2> 0$ be some reals so that
\autoref{thm:main} implies
$m_{t+2}\ge t^{\alpha_1}m^{1+2/t}$ or 
$m_{t+2}m_{t-2}\ge \alpha_2{m_t}^2$ for $m_t$ 
defined in statement of this theorem. 

Set $H = \mu_{k} - (\mu_{k-2} + \mu_{k+2})/(6\delta)$.
Let us argue first that $\iprod{A}{H}\ge 1/3$.
One can verify that 
$\iprod{A}{\mu_k}\ge 
  (1-\delta)(1-\binom{k}{2}/n)> 1-3\delta/2$
and $\iprod{A}{\mu_{k-2}} + \iprod{A}{\mu_{k-2}} < 3\delta$.
Hence $\iprod{A}{H}\ge 1/3$ for $\delta \le 1/9$.

We upper bound $\iprod{R}{H}$ for some rank-1
matrix $R=uv^\transpose$ with 0-1 values. Let $W$ be
the normalized adjacency matrix of the Hamming cube graph.
Observe that $\iprod{R}{\mu_k} = \iprod{v}{W^ku}/2^n$.
By \autoref{thm:main}, either 
$\iprod{R}{\mu_{k+2}}\iprod{R}{\mu_{k-2}}
\ge \alpha_2 \iprod{R}{\mu_k}^2$ or 
\begin{align*}
\nparen{\frac{\lnorm{u}{2}\lnorm{v}{2}}{2^n}}^{2/k}
\iprod{R}{\mu_{k+2}}\ge k^{\alpha_1}\iprod{R}{\mu_{k}}^{1+2/k}.
\end{align*}
In the former case,
\begin{align*}
\frac{\iprod{R}{\mu_{k+2}}+\iprod{R}{\mu_{k-2}}}{2}\ge 
\sqrt{\iprod{R}{\mu_{k+2}}\iprod{R}{\mu_{k-2}}}\ge
\sqrt{\alpha_2}\iprod{R}{\mu_k},
\end{align*}
which implies that $\iprod{R}{H}<0$ whenever 
$\delta<2\sqrt{\alpha_2}/6$ (recall $\alpha_2$ is a constant).
In the latter case we get $\iprod{R}{H}<0$ unless
$6\delta\iprod{R}{\mu_{k+2}} \le \iprod{R}{\mu_{k}}$, 
which implies, 
recalling $\lnorm{v}{2}\lnorm{u}{2}\le 2^n$,
that $k^{\alpha_1}\iprod{R}{\mu_k}^{2/k}<6\delta$.
From this we get
\begin{align*}
  \iprod{R}{H}\le \iprod{R}{\mu_k}\le
  \nparen{\frac{6\delta}{k^{\alpha_1}}}^{k/2},
\end{align*}
and hence $\iprod{R}{H} < \nparen{\frac{6\delta}{k^{\alpha_1}}}^{k/2}$ 
in every case and 
$R_\delta(\Ham^n_k)=\Omega(k\log(k/\delta))$
whenever $k^2<\delta n$.
\end{proof}

For a protocol $P$, denote by $\Pi=\Pi(x,y)$ the random variable 
entailing all the messages communicated between the players 
on input $(x,y)$.
So far we have considered the communication cost of a protocol
which is the maximum length of $\Pi$ over all inputs and the 
configurations of the random source (these together determine 
the value of $\Pi$). When a distribution $\mu$
on the inputs is available, we may speak of
a more refined notion of cost, {\em the internal information cost}, 
for a protocol $P$ which is defined as
\begin{align*}
\mathsf{IC}_\mu(P)\defeq \muti{\Pi}{Y\emid X} + \muti{\Pi}{X\emid Y},
\end{align*}
where $(X,Y)\sim\mu$. Combining our 
\autoref{thm:klogk} with a result of \cite{KerenidisLLRX2015}
which relates information and communication costs of a protocol 
under suitable circumstances, one
can conclude that any randomized protocol for $\Ham^n_k$ has
information cost $\Omega(k\log k)$ as well, under the distribution
$\mu = (\mu_k+\mu_{k-2}+\mu_{k+2})/3$. However we note that instead 
of using \autoref{thm:main} black-box,
taking a closer look at the proof of \autoref{thm:blakley-dixon} 
and not performing the relaxation provided in \autoref{lem:negterms},
we get the following more directly.
\begin{theorem}
\label{thm:infocost}
Let $P$ be a protocol outputting 1 on pairs $(x,y)$ having
$\lone{x-y}=k$ with 
probability $1-\delta$ and outputting 0 on pairs 
$(x,y)$ having
$\lone{x-y}\in\set{k-2,k+2}$ with probability $1-\delta$. 
We have 
$\mathsf{IC}_{\mu_k}(P)=\Omega(k\log (k/\delta))$.
\end{theorem}

Let us finally mention another highly related problem, 
the so called the gap Hamming distance problem. 
In $\Ghd^n_{k}$, each of the players receive a bit string,
respectively $x,y\in\cube$, with the promise that either 
$\lone{x-y}\le k$ or $\lone{x-y}\ge k + \sqrt{k}$.
Their goal is to determine which is the case for any given input.
In \cite{ChakrabartiR2012}, an $\Omega(k)$ lower bound
for this problem was shown, which applies to protocols with any
number of rounds.
Here we conjecture an improvement to this bound
and argue that it would follow from a natural analogue of
\autoref{thm:main} for continuous time Markov chains, which
we discuss in \autoref{sec:ham:discussion}.
\begin{conjecture}
\label{conj:ghd}
For $k<\delta n$, we have $R_\delta(\Ghd^n_k) 
= \Omega(k \log(1/\delta))$.
\end{conjecture}

\section{Parity decision trees}
\label{sec:ham:pdt}

\def\PD{\mathrm{PD}}
\def\PS{\mathrm{PS}}
In the parity decision tree model, we are given a string 
$x\in\field_2^n$ and our goal is to determine whether 
$x$ satisfies a fixed predicate
$P\colon\field_2^n\to\binary$ by only making linear 
measurements of the form $\iprod{x}{y}$ for some 
$y\in \field_2^n$ we get to choose. Here, the inner product
is over $\field_2^n$, and therefore we get a single bit 
answer for every measurement we make.

Such measurements can be identified by binary decision
trees wherein each internal node is labeled by a
$y\in\field_2^n$ denoting the linear measurement
$\iprod{x}{y}$ we would make at that node and each leaf
is labeled by a YES or a NO denoting the final decision we 
arrive. Given such a tree and an $x$, the output of the 
decision tree is obtained by a root to leaf walk, where at 
each internal node $v$ with label $y_v$, 
we perform the measurement 
$\iprod{x}{y_v}$ and walk to the left child of $v$ if 
$\iprod{x}{y_v}=0$ and to its right child if 
$\iprod{x}{y_v}=1$. If a leaf node is reached, 
the label of the node is taken as the answer of 
the decision tree. Two quantities we are concerned with
are the depth and the size (i.e., the total number of nodes)
of the tree.

A $\delta$-error randomized decision tree is a distribution
$\nu$ over decision trees such that for any fixed $x$, 
the sampled decision tree outputs the correct answer with 
probability at least $1-\delta$, where the randomness is over 
the choice of the decision tree from $\nu$. 
The depth and the size of a randomized decision tree can be 
taken as the maximum over the decision trees in the support 
of $\nu$ (here, one can also take the average depth or size also;
our result on decision tree size actually lower bounds this 
potentially smaller quantity).

For a predicate $P\colon\field_2^n\to\binary$, let 
$\PD_\delta(P)$ be the minimum, over all randomized decision 
trees $T$ computing $P$ with probability $1-\delta$, of the 
depth of $T$.
Let $\PS_\delta(P)$ be the minimum, over all randomized decision 
trees $T$ computing $P$ with probability $1-\delta$, of the 
size of $T$. The following inequalities are immediate
\begin{align}
  R_\delta(P\circ\oplus)&\le 2\PD_\delta(P)\label{eq:rvspd},\\
  \log\PS_\delta(P)&\le \PD_\delta(P)\nonumber,
\end{align}
where $P\circ \oplus$ is the two player communication game 
in which the two players are given strings $x,y\in\field_2^n$ 
and are required to calculate $P(x+y)$. 
We remark that $\log \PS_\delta$ is incomparable to $R_\delta$ 
in general.

Here we study the predicate $H^n_k$ which equals 1 if 
and only if the Hamming weight of its input is precisely $k$. 
By \autoref{eq:rvspd} and a padding argument similar to the 
one we gave before the proof of \autoref{thm:klogk}, each lower 
bound for $\Ham^n_k$ listed in \autoref{table:hamvsdisj} applies 
to $\PD_\delta(H^n_k)$ as well. In \cite{BlaisK2012} another 
direct $\Omega(k)$ bound for $\PD_\delta(H^n_k)$ was shown. In 
\cite{BhrushundiCK2014}, showing an $\Omega(k\log k)$ lower bound 
to a variant of $\PD_\delta(H^n_k)$ to obtain tight bounds for 
$k$-linearity problem (see \autoref{sec:ham:proptest}) was suggested. 
Finally, our \autoref{thm:klogk} shows that 
$\PD_\delta(H^n_k)=\Omega(k\log(k/\delta))$, which is tight. 
Next we show the same bound holds even for $\log \PS_\delta(H^n_k)$.

\begingroup
\def\thetheorem{\ref{thm:paritysize}}
\begin{theorem}[restated]
For $k^2<\delta n$, 
$\log \PS_\delta(H^n_k)=\Omega(k\log(k/\delta))$.
\end{theorem}
\addtocounter{theorem}{-1}
\endgroup
\begin{proof}
\def\f{\field_2^n}
The proof is very similar to that of \autoref{thm:klogk}, so we only 
describe the differences.

Let $T$ be a $\delta$-error randomized parity decision tree computing
$H^n_k$. Form $A\colon\f\to[0,1]$ so that $A(x)$ is the probability 
$T$ outputs 1 on input $x\in\f$.
Define the polytope
\begin{align*}
\mathcal{P}\defeq\conv
\set{x\mapsto \indicate{}[Bx = c]\mid B\in\field_2^{n\times n}, c\in\f}
\end{align*}
whose vertices are indicator functions for affine subspaces of $\f$.
Given a randomized parity decision tree, 
for each fixing of the randomness,
the set of inputs that end up in
a particular leaf of it is an affine subspace in $\f$. Therefore if
$T$ has at most $s$ leaves, then $A$ is inside $s\mathcal{P}$.
It remains to demonstrate a hyperplane
with normal $H$ so that 
$\iprod{A}{H}>s\iprod{V}{H}$ for any vertex $V$ of the polytope 
$\mathcal{P}$ for $s=\exp \Omega(k\log(k/\delta))$.

Let $\mu_k$ be a distribution on $\f$ obtained as follows. 
Start with the $0$ vector, and flip a coordinate chosen 
uniformly at random with replacement $k$ times. 
Here, flipping a coordinate an even number of times leaves it 
as $0$.
Set $H= \mu_k - (\mu_{k-2}+\mu_{k+2})/(6\delta)$.

First observe that 
$\iprod{A}{\mu_k}>(1-\delta)(1-\binom{k}{2}/n)>1-3\delta/2$
and $\iprod{A}{\mu_{k+2}}+ \iprod{A}{\mu_{k-2}}<3\delta$ so
$\iprod{A}{H}\ge 1/3$ for $\delta\le 1/9$.
Next we would like to upper bound $\iprod{V}{H}$ for an indicator
function $V$ of an affine subspace $\setbuilder{x\in\f}{Bx=c}$.
The key observation is
\begin{align}
\iprod{V}{\mu_k} = \iprod{\indicate{c}}{S^k\indicate{0}}
\label{eq:chi}
\end{align}
where $S$ is a stochastic matrix describing the following transition:
For any $x\in\f$, sample a column $y$ of $B\in\field_2^{n\times n}$ 
uniformly at random and transition to $x+y$. 
Namely, the right hand side of \autoref{eq:chi} describes the following
probability.
We start with the 0 vector in $\f$ and in each time step sample a 
uniform random column $y$ of $B$ and add $y$ to the current state. 
We measure the probability of reaching $c\in\f$ at time step $k$. 
Having observed \autoref{eq:chi}, and that 
$\lnorm{\indicate{0}}{2} = \lnorm{\indicate{c}}{2} = 1$,
the rest of the proof is identical to that of \autoref{thm:klogk}: 
by \autoref{thm:main}, we either have 
\begin{align*}
\iprod{\indicate{c}}{S^{k+2}\indicate{0}}
\iprod{\indicate{c}}{S^{k-2}\indicate{0}}\ge
  \alpha_2 \iprod{\indicate{c}}{S^k\indicate{0}}^2
\end{align*}
or
\begin{align*}
\iprod{\indicate{c}}{S^{k+2}\indicate{0}}
\ge k^{\alpha_1} \iprod{\indicate{c}}{S^k\indicate{0}}^{1+2/k}.
\end{align*}
In either event, we conclude that $\iprod{V}{H}\le 
\nparen{\frac{6\delta}{k^{\alpha_1}}}^{k/2}$.  This completes the proof.
\end{proof}
Note in \autoref{thm:klogk}, we use \autoref{thm:main} with a 
simple and fixed $S$ 
(i.e., the standard random walk on the Hamming cube), 
but with complicated vectors $u,v$ that come from the particular 
communication protocol whose communication cost we would like 
to lower bound. By contrast, in \autoref{thm:paritysize} the 
vectors $u,v$ are simple point masses on states $0$ and $c$ but 
the matrix $S$ is a convolution random walk on the Hamming cube 
that comes from the particular decision tree whose size we lower 
bound.

\section{Property testing}
\label{sec:ham:proptest}
In the property testing model, given black box access to 
an otherwise unknown function $f\colon\field^n_2\to\field_2$, 
our goal is to tell apart whether $x\in P$ for some fixed set 
of functions $P$ or $\lone{f-g}\ge \eps2^n$ for any $g\in P$. 
Here, the black box queries are done by providing an input 
$x\in\field_2^n$ to the function and observing $f(x)$.

A function $f\colon\field_2^n\to\field_2$ is
called $k$-linear if $f$ is given by
\begin{align*}
f(x) = \sum_{i\in S}x_i
\end{align*} 
for some $S\subseteq[n]$ of size at most $k$.
By combining our communication complexity lower bound 
\autoref{thm:klogk} with the reduction technique developed 
in \cite{BlaisBM2012} or by combining
our parity decision tree lower bound \autoref{thm:paritysize} 
with a reduction given in \cite{BhrushundiCK2014}, one obtains 
the following.

\begingroup
\def\thecorollary{\ref{cor:propertytest}}
\begin{corollary}[restated]
Any $\delta$-error property testing algorithm for 
$k$-linearity with $\epsilon=1/2$ requires 
$\Omega(k\log (k/\delta))$ queries.
\end{corollary}
\addtocounter{theorem}{-1}
\endgroup
In fact through this, one obtains similar lower bounds to
property testing for $k$-juntas, $k$-term DNFs, size-$k$ formulas,
size-$k$ decision trees, $k$-sparse $\field_2$-polynomials;
see \cite{Blais2009, ChakrabortyGM2011}.

% !TeX root = thesis.tex
\section{The lower bound for $\Ham^n_k$}
\label{sec:ham:lowerbound}

\begingroup
\def\thetheorem{\ref{thm:klogk}}
\begin{theorem}[restated]
For $k^2<\delta n$ we have 
$R_{\delta}(\Ham^n_k) = \Omega(k\log (k/\delta))$.
The bound applies even to protocols that may
output an arbitrary answer when $\lone{x-y} \notin \set{k-2, k,k+2}$.
\end{theorem}
\addtocounter{theorem}{-1}
\endgroup
Before we proceed with proving \autoref{thm:klogk}, let us
first warm up by showing that \autoref{thm:blakley-dixon} 
implies an  $\Omega(k\log(1/\delta))$ lower bound on 
$R_\delta(\Ham^n_k)$.
To do so, let us review the so called {\em corruption bound}
method. Let  $f\colon\mathcal{X}\times\mathcal{Y}\to\binary$ 
be the function the players would like to compute with 
Alice having received $x\in \mathcal{X}$ and Bob $y\in \mathcal{Y}$. 
For a protocol $P$ for $f$, define the matrix 
$A_P\colon\mathcal{X}\times\mathcal{Y}\to[0,1]$ 
such that $A_P(x,y)$ is the probability
that the protocol outputs 1 on input $(x,y)$.
It is well known and not difficult to see that
if $P$ has communication cost $c$, then $A_P$ is the
average of matrices each of which is the sum of at most
$2^c$ rank 1 matrices $u v^\transpose$ with 
$u\in \binary^\mathcal{X}$ and $v\in\binary^\mathcal{Y}$.
Therefore to show the communication cost of a protocol
$P$ is more than $c$, it suffices to argue $A_P$ lies
outside $2^c$ times the polytope
\begin{align*}
\mathcal{T}\defeq 
\conv\set{uv^\transpose\mid 
u\in\binary^\mathcal{X}, v\in\binary^\mathcal{Y}},
\end{align*}
where $\conv$ denotes the convex hull. By convexity, 
$A_p$ lies outside of $2^c\mathcal{T}$ if and only if there 
is a hyperplane (with normal $H$) separating the two; 
namely that $\iprod{A_P}{H}>2^c\iprod{R}{H}$ for all 
vertices $R$ of the polytope $\mathcal{T}$.

Let $\mu_k\colon\cube\times\cube\to\realspos$ be the distribution
on pairs $(x,y)$ obtained as follows. Sample $x$ uniformly at random
and obtain $y$ by flipping $k$ coordinates of $x$ chosen uniformly 
at random and with replacement (here if a coordinate gets flipped 
twice it reverts back to its initial value).
\begingroup
\def\thetheorem{\ref{thm:klogdelta}}
\begin{theorem}[restated]
For $k^2<\delta n$ we have 
$R_{\delta}(\Ham^n_k) = \Omega(k\log (1/\delta))$.
The bound applies even to protocols that may
output an arbitrary answer when $\lone{x-y} \notin \set{k,k+2}$.
\end{theorem}
\addtocounter{theorem}{-1}
\endgroup
\begin{proof}
Suppose we have a randomized protocol for $\Ham^n_k$ with 
error probability $\delta$. Form the matrix $A$, where
$A(x,y)$ is the probability that the protocol reports
$\lone{x-y}\le k$ on input $(x,y)$.

Set $H = \mu_k - \mu_{k+2}/(3\delta)$. Let us first argue that
$\iprod{A}{H}\ge 1/3$. Note that when we sample $k$ elements 
from $[n]$ uniformly at random, by a union bound, the probability
that there is a collision is at most $\binom{k}{2}/n$, therefore 
$\mu_k$ chooses a pair $(x,y)$ at distance $k$ with probability 
at least $1-\binom{k}{2}/n$.
Hence,
$\iprod{A}{\mu_k} 
   \ge (1-\delta)(1-\binom{k}{2}/n)
     > 1-3\delta/2$,
where in the last step we used $k^2< \delta n$.
Similarly, we have $\iprod{A}{\mu_{k+2}} \le 
                    \delta + \binom{k+2}{2}/n \le 3\delta/2$,
thus it follows that $\iprod{A}{H}\ge 1/3$ for $\delta \le 1/9$.

Next we argue that $\iprod{R}{H}< (3\delta)^{k/2}$ for any 
$R=uv^\transpose$ with $u,v\in\cube$. If 
$\iprod{R}{\mu_k}<(3\delta)^{k/2}$, we are done as 
$\iprod{R}{\mu_{k+2}}\ge 0$ and is a negative term in 
$\iprod{R}{H}$. If $\iprod{R}{\mu_k}\ge (3\delta)^{k/2}$
on the other hand,
observing $\iprod{R}{\mu_k}=\iprod{v}{W^k u}/2^n$,
where $W$ is the normalized adjacency matrix of the Hamming cube,
we have by \autoref{thm:blakley-dixon}
\begin{align*}
\nparen{\frac{\lnorm{u}{2}\lnorm{v}{2}}{2^n}}^{2/k}
\iprod{v}{W^{k+2}u} \ge 
\iprod{v}{W^{k}u}^{1+2/k}.
\end{align*}
Note $\lnorm{u}{2}\lnorm{v}{2}\le 2^n$ since $u,v$ are 0-1 vectors,
therefore 
$\iprod{R}{\mu_{k+2}}\ge 3\delta\iprod{R}{\mu_{k}}$
and hence $\iprod{R}{H}\le 0$. In either case we have shown
that $\iprod{R}{H} < (3\delta)^{k/2}$. This implies an 
$\log((3\delta)^{-k/2} / 3) = \Omega(k\log(1/\delta))$ bits 
lower bound on $R_\delta(\Ham^n_k)$.
\end{proof}
Interestingly, \autoref{thm:klogk} cannot be proved by a direct 
application of the corruption method described above. 
If we assume that the protocol
is supposed to output 1 on inputs $\lone{x-y}\le k$, then there are 
vertices of the polytope $\mathcal{T}$ for which 
the $\Omega(k\log(1/\delta))$ bound of 
\autoref{thm:klogdelta} is tight. If we assume that the protocol
is supposed to output 1 on inputs $\lone{x-y}> k$ on the other
hand, no bound above $\Omega(k)$ can be obtained, as there are
vertices for which this is tight.
If we insist however that the protocol outputs 1 for $\lone{x-y}=k$
and 0 for $\lone{x-y}\in\set{k-2,k+2}$ then a protocol with cost
smaller than $O(k\log (k/\delta))$
would be in violation of the near log-convexity principle we 
established in \autoref{thm:main} as we argue next.
Of course, if we had a $\delta$-error randomized protocol 
$P$ for 
$\Ham^{n+2}_k$ outputting 1 when $\lone{x-y}\in\set{k-2,k}$ 
and 0 if $\lone{x-y}=k+2$ 
(but without any guarantees for other types of inputs), then 
given inputs $a,b\in\cube$ Alice and Bob can run $P$
(say, in parallel)
on instances $(00a, 00b)$ and $(00a, 11b)$ and declare 
$\lone{a-b}=k$ if $P$ returns 1 on $(00a, 00b)$ and 
0 on $(00a, 11b)$.
This would lead to a protocol with twice the error probability
and communication cost of $P$, deciding between $\lone{a-b}=k$,
$\lone{a-b}=k-2$ and $\lone{a-b}=k+2$. The table below shows that
$P$ outputting 1 on $(00a, 00b)$ and 0 on $(00a, 11b)$ implies
$\lone{a-b}=k$ or at least one invocation of $P$ erred.

\begin{center}
\def\arraystretch{1.2}
\begin{tabular}{rccc}
Input       & $k-2$ & $k$  & $k+2$ \\\hline
$(00a,00b)$ &   1   &  1   &  0  \\
$(00a,11b)$ &   1   &  0   &  ? 
\end{tabular}
\end{center}

\begin{proof}[Proof of \autoref{thm:klogk}]
Suppose we have a $\delta$-error randomized protocol that 
outputs 1 when $\lone{x-y} = k$ and 0 when 
$\lone{x-y} = k-2$ or $\lone{x-y} = k+2$.
 
Form the matrix $A$, where $A(x,y)$ is the probability 
that the protocol reports that $\lone{x-y}= k$ on 
input $(x,y)$. Let $\alpha_1,\alpha_2> 0$ be some reals so that
\autoref{thm:main} implies
$m_{t+2}\ge t^{\alpha_1}m^{1+2/t}$ or 
$m_{t+2}m_{t-2}\ge \alpha_2{m_t}^2$ for $m_t$ 
defined in statement of this theorem. 

Set $H = \mu_{k} - (\mu_{k-2} + \mu_{k+2})/(6\delta)$.
Let us argue first that $\iprod{A}{H}\ge 1/3$.
One can verify that 
$\iprod{A}{\mu_k}\ge 
  (1-\delta)(1-\binom{k}{2}/n)> 1-3\delta/2$
and $\iprod{A}{\mu_{k-2}} + \iprod{A}{\mu_{k-2}} < 3\delta$.
Hence $\iprod{A}{H}\ge 1/3$ for $\delta \le 1/9$.

We upper bound $\iprod{R}{H}$ for some rank-1
matrix $R=uv^\transpose$ with 0-1 values. Let $W$ be
the normalized adjacency matrix of the Hamming cube graph.
Observe that $\iprod{R}{\mu_k} = \iprod{v}{W^ku}/2^n$.
By \autoref{thm:main}, either 
$\iprod{R}{\mu_{k+2}}\iprod{R}{\mu_{k-2}}
\ge \alpha_2 \iprod{R}{\mu_k}^2$ or 
\begin{align*}
\nparen{\frac{\lnorm{u}{2}\lnorm{v}{2}}{2^n}}^{2/k}
\iprod{R}{\mu_{k+2}}\ge k^{\alpha_1}\iprod{R}{\mu_{k}}^{1+2/k}.
\end{align*}
In the former case,
\begin{align*}
\frac{\iprod{R}{\mu_{k+2}}+\iprod{R}{\mu_{k-2}}}{2}\ge 
\sqrt{\iprod{R}{\mu_{k+2}}\iprod{R}{\mu_{k-2}}}\ge
\sqrt{\alpha_2}\iprod{R}{\mu_k},
\end{align*}
which implies that $\iprod{R}{H}<0$ whenever 
$\delta<2\sqrt{\alpha_2}/6$ (recall $\alpha_2$ is a constant).
In the latter case we get $\iprod{R}{H}<0$ unless
$6\delta\iprod{R}{\mu_{k+2}} \le \iprod{R}{\mu_{k}}$, 
which implies, 
recalling $\lnorm{v}{2}\lnorm{u}{2}\le 2^n$,
that $k^{\alpha_1}\iprod{R}{\mu_k}^{2/k}<6\delta$.
From this we get
\begin{align*}
  \iprod{R}{H}\le \iprod{R}{\mu_k}\le
  \nparen{\frac{6\delta}{k^{\alpha_1}}}^{k/2},
\end{align*}
and hence $\iprod{R}{H} < \nparen{\frac{6\delta}{k^{\alpha_1}}}^{k/2}$ 
in every case and 
$R_\delta(\Ham^n_k)=\Omega(k\log(k/\delta))$
whenever $k^2<\delta n$.
\end{proof}

For a protocol $P$, denote by $\Pi=\Pi(x,y)$ the random variable 
entailing all the messages communicated between the players 
on input $(x,y)$.
So far we have considered the communication cost of a protocol
which is the maximum length of $\Pi$ over all inputs and the 
configurations of the random source (these together determine 
the value of $\Pi$). When a distribution $\mu$
on the inputs is available, we may speak of
a more refined notion of cost, {\em the internal information cost}, 
for a protocol $P$ which is defined as
\begin{align*}
\mathsf{IC}_\mu(P)\defeq \muti{\Pi}{Y\emid X} + \muti{\Pi}{X\emid Y},
\end{align*}
where $(X,Y)\sim\mu$. Combining our 
\autoref{thm:klogk} with a result of \cite{KerenidisLLRX2015}
which relates information and communication costs of a protocol 
under suitable circumstances, one
can conclude that any randomized protocol for $\Ham^n_k$ has
information cost $\Omega(k\log k)$ as well, under the distribution
$\mu = (\mu_k+\mu_{k-2}+\mu_{k+2})/3$. However we note that instead 
of using \autoref{thm:main} black-box,
taking a closer look at the proof of \autoref{thm:blakley-dixon} 
and not performing the relaxation provided in \autoref{lem:negterms},
we get the following more directly.
\begin{theorem}
\label{thm:infocost}
Let $P$ be a protocol outputting 1 on pairs $(x,y)$ having
$\lone{x-y}=k$ with 
probability $1-\delta$ and outputting 0 on pairs 
$(x,y)$ having
$\lone{x-y}\in\set{k-2,k+2}$ with probability $1-\delta$. 
We have 
$\mathsf{IC}_{\mu_k}(P)=\Omega(k\log (k/\delta))$.
\end{theorem}

Let us finally mention another highly related problem, 
the so called the gap Hamming distance problem. 
In $\Ghd^n_{k}$, each of the players receive a bit string,
respectively $x,y\in\cube$, with the promise that either 
$\lone{x-y}\le k$ or $\lone{x-y}\ge k + \sqrt{k}$.
Their goal is to determine which is the case for any given input.
In \cite{ChakrabartiR2012}, an $\Omega(k)$ lower bound
for this problem was shown, which applies to protocols with any
number of rounds.
Here we conjecture an improvement to this bound
and argue that it would follow from a natural analogue of
\autoref{thm:main} for continuous time Markov chains, which
we discuss in \autoref{sec:ham:discussion}.
\begin{conjecture}
\label{conj:ghd}
For $k<\delta n$, we have $R_\delta(\Ghd^n_k) 
= \Omega(k \log(1/\delta))$.
\end{conjecture}

\section{Parity decision trees}
\def\PD{\mathrm{PD}}
\def\PS{\mathrm{PS}}
In the parity decision tree model, we are given a string 
$x\in\field_2^n$ and our goal is to determine whether 
$x$ satisfies a fixed predicate
$P\colon\field_2^n\to\binary$ by only making linear 
measurements of the form $\iprod{x}{y}$ for some 
$y\in \field_2^n$ we get to choose. Here, the inner product
is over $\field_2^n$, and therefore we get a single bit 
answer for every measurement we make.

Such measurements can be identified by binary decision
trees wherein each internal node is labeled by a
$y\in\field_2^n$ denoting the linear measurement
$\iprod{x}{y}$ we would make at that node and each leaf
is labeled by a YES or a NO denoting the final decision we 
arrive. Given such a tree and an $x$, the output of the 
decision tree is obtained by a root to leaf walk, where at 
each internal node $v$ with label $y_v$, 
we perform the measurement 
$\iprod{x}{y_v}$ and walk to the left child of $v$ if 
$\iprod{x}{y_v}=0$ and to its right child if 
$\iprod{x}{y_v}=1$. If a leaf node is reached, 
the label of the node is taken as the answer of 
the decision tree. Two quantities we are concerned with
are the depth and the size (i.e., the total number of nodes)
of the tree.

A $\delta$-error randomized decision tree is a distribution
$\nu$ over decision trees such that for any fixed $x$, 
the sampled decision tree outputs the correct answer with 
probability at least $1-\delta$, where the randomness is over 
the choice of the decision tree from $\nu$. 
The depth and the size of a randomized decision tree can be 
taken as the maximum over the decision trees in the support 
of $\nu$ (here, one can also take the average depth or size also;
our result on decision tree size actually lower bounds this 
potentially smaller quantity).

For a predicate $P\colon\field_2^n\to\binary$, let 
$\PD_\delta(P)$ be the minimum, over all randomized decision 
trees $T$ computing $P$ with probability $1-\delta$, of the 
depth of $T$.
Let $\PS_\delta(P)$ be the minimum, over all randomized decision 
trees $T$ computing $P$ with probability $1-\delta$, of the 
size of $T$. The following inequalities are immediate
\begin{align}
  R_\delta(P\circ\oplus)&\le 2\PD_\delta(P)\label{eq:rvspd},\\
  \log\PS_\delta(P)&\le \PD_\delta(P)\nonumber,
\end{align}
where $P\circ \oplus$ is the two player communication game 
in which the two players are given strings $x,y\in\field_2^n$ 
and are required to calculate $P(x+y)$. 
We remark that $\log \PS_\delta$ is incomparable to $R_\delta$ 
in general.

Here we study the predicate $H^n_k$ which equals 1 if 
and only if the Hamming weight of its input is precisely $k$. 
By \autoref{eq:rvspd} and a padding argument similar to the 
one we gave before the proof of \autoref{thm:klogk}, each lower 
bound for $\Ham^n_k$ listed in \autoref{table:hamvsdisj} applies 
to $\PD_\delta(H^n_k)$ as well. In \cite{BlaisK2012} another 
direct $\Omega(k)$ bound for $\PD_\delta(H^n_k)$ was shown. In 
\cite{BhrushundiCK2014}, showing an $\Omega(k\log k)$ lower bound 
to a variant of $\PD_\delta(H^n_k)$ to obtain tight bounds for 
$k$-linearity problem (see \autoref{sec:proptest}) was suggested. 
Finally, our \autoref{thm:klogk} shows that 
$\PD_\delta(H^n_k)=\Omega(k\log(k/\delta))$, which is tight. 
Next we show the same bound holds even for $\log \PS_\delta(H^n_k)$.

\begingroup
\def\thetheorem{\ref{thm:paritysize}}
\begin{theorem}[restated]
For $k^2<\delta n$, 
$\log \PS_\delta(H^n_k)=\Omega(k\log(k/\delta))$.
\end{theorem}
\addtocounter{theorem}{-1}
\endgroup
\begin{proof}
\def\f{\field_2^n}
The proof is very similar to that of \autoref{thm:klogk}, so we only 
describe the differences.

Let $T$ be a $\delta$-error randomized parity decision tree computing
$H^n_k$. Form $A\colon\f\to[0,1]$ so that $A(x)$ is the probability 
$T$ outputs 1 on input $x\in\f$.
Define the polytope
\begin{align*}
\mathcal{P}\defeq\conv
\set{x\mapsto \indicate{}[Bx = c]\mid B\in\field_2^{n\times n}, c\in\f}
\end{align*}
whose vertices are indicator functions for affine subspaces of $\f$.
Given a randomized parity decision tree, 
for each fixing of the randomness,
the set of inputs that end up in
a particular leaf of it is an affine subspace in $\f$. Therefore if
$T$ has at most $s$ leaves, then $A$ is inside $s\mathcal{P}$.
It remains to demonstrate a hyperplane
with normal $H$ so that 
$\iprod{A}{H}>s\iprod{V}{H}$ for any vertex $V$ of the polytope 
$\mathcal{P}$ for $s=\exp \Omega(k\log(k/\delta))$.

Let $\mu_k$ be a distribution on $\f$ obtained as follows. 
Start with the $0$ vector, and flip a coordinate chosen 
uniformly at random with replacement $k$ times. 
Here, flipping a coordinate an even number of times leaves it 
as $0$.
Set $H= \mu_k - (\mu_{k-2}+\mu_{k+2})/(6\delta)$.

First observe that 
$\iprod{A}{\mu_k}>(1-\delta)(1-\binom{k}{2}/n)>1-3\delta/2$
and $\iprod{A}{\mu_{k+2}}+ \iprod{A}{\mu_{k-2}}<3\delta$ so
$\iprod{A}{H}\ge 1/3$ for $\delta\le 1/9$.
Next we would like to upper bound $\iprod{V}{H}$ for an indicator
function $V$ of an affine subspace $\setbuilder{x\in\f}{Bx=c}$.
The key observation is
\begin{align}
\iprod{V}{\mu_k} = \iprod{\indicate{c}}{S^k\indicate{0}}
\label{eq:chi}
\end{align}
where $S$ is a stochastic matrix describing the following transition:
For any $x\in\f$, sample a column $y$ of $B\in\field_2^{n\times n}$ 
uniformly at random and transition to $x+y$. 
Namely, the right hand side of \autoref{eq:chi} describes the following
probability.
We start with the 0 vector in $\f$ and in each time step sample a 
uniform random column $y$ of $B$ and add $y$ to the current state. 
We measure the probability of reaching $c\in\f$ at time step $k$. 
Having observed \autoref{eq:chi}, and that 
$\lnorm{\indicate{0}}{2} = \lnorm{\indicate{c}}{2} = 1$,
the rest of the proof is identical to that of \autoref{thm:klogk}: 
by \autoref{thm:main}, we either have 
\begin{align*}
\iprod{\indicate{c}}{S^{k+2}\indicate{0}}
\iprod{\indicate{c}}{S^{k-2}\indicate{0}}\ge
  \alpha_2 \iprod{\indicate{c}}{S^k\indicate{0}}^2
\end{align*}
or
\begin{align*}
\iprod{\indicate{c}}{S^{k+2}\indicate{0}}
\ge k^{\alpha_1} \iprod{\indicate{c}}{S^k\indicate{0}}^{1+2/k}.
\end{align*}
In either event, we conclude that $\iprod{V}{H}\le 
\nparen{\frac{6\delta}{k^{\alpha_1}}}^{k/2}$.  This completes the proof.
\end{proof}
Note in \autoref{thm:klogk}, we use \autoref{thm:main} with a 
simple and fixed $S$ 
(i.e., the standard random walk on the Hamming cube), 
but with complicated vectors $u,v$ that come from the particular 
communication protocol whose communication cost we would like 
to lower bound. By contrast, in \autoref{thm:paritysize} the 
vectors $u,v$ are simple point masses on states $0$ and $c$ but 
the matrix $S$ is a convolution random walk on the Hamming cube 
that comes from the particular decision tree whose size we lower 
bound.

\section{Property testing}
\label{sec:ham:proptest}
In the property testing model, given black box access to 
an otherwise unknown function $f\colon\field^n_2\to\field_2$, 
our goal is to tell apart whether $x\in P$ for some fixed set 
of functions $P$ or $\lone{f-g}\ge \eps2^n$ for any $g\in P$. 
Here, the black box queries are done by providing an input 
$x\in\field_2^n$ to the function and observing $f(x)$.

A function $f\colon\field_2^n\to\field_2$ is
called $k$-linear if $f$ is given by
\begin{align*}
f(x) = \sum_{i\in S}x_i
\end{align*} 
for some $S\subseteq[n]$ of size at most $k$.
By combining our communication complexity lower bound 
\autoref{thm:klogk} with the reduction technique developed 
in \cite{BlaisBM2012} or by combining
our parity decision tree lower bound \autoref{thm:paritysize} 
with a reduction given in \cite{BhrushundiCK2014}, one obtains 
the following.

\begingroup
\def\thecorollary{\ref{cor:propertytest}}
\begin{corollary}[restated]
Any $\delta$-error property testing algorithm for 
$k$-linearity with $\epsilon=1/2$ requires 
$\Omega(k\log (k/\delta))$ queries.
\end{corollary}
\addtocounter{theorem}{-1}
\endgroup
In fact through this, one obtains similar lower bounds to
property testing for $k$-juntas, $k$-term DNFs, size-$k$ formulas,
size-$k$ decision trees, $k$-sparse $\field_2$-polynomials;
see \cite{Blais2009, ChakrabortyGM2011}.

% !TeX root = thesis.tex
\section{Discussion}
\label{sec:ham:discussion}
We showed that for a symmetric matrix 
$S\colon\Omega\times\Omega\to\realspos$ and 
unit vectors $u,v\colon\Omega\to\realspos$, defining
$m_t = \iprod{v}{S^tu}$ for $t=0,1,\ldots$, we have 
\begin{align}
m_{t+2}    &\ge m_t^{1+2/t}, \text{ and}\label{eq:dbd66}\\
m_{t+2}    &\ge m_t^{1+2/t}\cdot \min\set{t^{1-\eps}, 
\ceil{\delta\frac{m_t^{1-2/t}}{m_{t-2}}}} \label{eq:dconv}
\end{align}
and argued that \autoref{eq:dconv} and \eqref{eq:dbd66}, 
in this order, are best viewed as gradual weakenings of
the log-convexity of $\set{m_t}_{t=0}^\infty$.
We conjecture that a similar principle holds
true for continuous time Markov chains as well.

Call a function $f\colon\realspos\to[0,1]$,
whose logarithm is continuously twice differentiable 
(i.e., $\log f\in C^2(\realspos)$), {\em nearly-log-convex} 
if $x^2 (\log f)''(x)\ge 2 \log f(x)$ for $x\in\realspos$. 
Note that $\log f \le 0$, therefore this is a weakening
of the usual log-convexity definition, which requires 
$(\log f)''\ge 0$.

\begin{conjecture}
\label{conj:continuouslogconv}
Let $S\colon\Omega\times\Omega\to\realspos$
be a symmetric substochastic matrix and $u,v\colon\Omega\to\realspos$
be unit vectors. The function
\begin{align*}
t\mapsto \iprod{v}{e^{t(S-I)}u}
\end{align*}
is nearly-log-convex.
\end{conjecture}
By an argument similar to the proof of \autoref{thm:klogk}, one can 
show the following.
\begin{theorem}
\autoref{conj:continuouslogconv} implies \autoref{conj:ghd}.
\end{theorem}

\section{Chapter notes}
\label{ham:notes}
The results of this chapter have been published in FOCS 2018 in \cite{Saglam2018}.

\bibliographystyle{plain}
\bibliography{references/main}

\end{document}
