% !TeX root = thesis.tex
\abstract{%
Suppose two parties, traditionally called Alice and Bob, are
given respectively the inputs $x\in\mathcal{X}$ and
$y\in\mathcal{Y}$ to a function
$f\colon\mathcal{X}\times\mathcal{Y}\to\mathcal{Z}$ and are
required to compute $f(x,y)$.
Since each party only has one part of the input, they can
compute $f(x,y)$ only if some communication takes place between
them. The communication complexity of a given function is the
minimum amount of communication (in bits) needed to evaluate it
on any input with high probability.

We study the communication complexity of two related problems,
the $k$-Hamming distance and $k$-disjointness and give tight
bounds to both of these problems: The $r$-round communication
complexity of the $k$-disjointness problem is
$\Theta(k\log^{(r)}k)$, whereas a tight $\Omega(k\log k)$ bound
holds for the $k$-Hamming distance problem for any number of
rounds.

The lower bound direction of our first result is obtained by
proving a {\em super-sum} result on computing the OR of $n$
equality problems, which is the first of its kind.
Using our second bound, we settle the complexity of various
property testing problems such as $k$-linearity, which was open
since 2002 or earlier. Our lower bounds are obtained via
information theoretic arguments and along the way we resolve a
question conjectured by Erdős and Simonovits in 1982, which
incidentally was studied even earlier by Blakley and Dixon in
1966.
}
