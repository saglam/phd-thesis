% !TeX root = thesis.tex
\section{Lower bound for $\Disj^n_k$}
\label{sec:disj:lowerbound}

%Let us start with an informal overview of the proof.

%Recall that by $\mu$ we denoted the uniform 
%distribution on $[t]^n$.

%\subsection{Inspecting the protocol for the large instance}
%\label{sec:inspect}
In this section we prove our main lower bound result for the 
exists-equal problem, which implies a corresponding lower bound
for the $k$-disjointness problem.
\begin{theorem}
\label{thm:main}
For any $r\leq\log^*n$, an $r$-round probabilistic protocol for
$\EE_n$ with error probability at most $1/3$ sends at least one
message of size $\Omega(n\log^{(r)}n)$.
\end{theorem}

Note that the $r=1$ round case of this theorem was proved as
\autoref{thm:singleround} in \autoref{sec:elementary}. The other
extreme, which immediately follows from \autoref{thm:main}, is
the following.

\begin{corollary}
Any randomized protocol for $\EE_n$ with maximum message size
$O(n)$ and error $1/3$ has at least $\log^* n - O(1)$ rounds.
\end{corollary}

\autoref{thm:main} is a direct consequence of the corresponding
statement on deterministic protocols with small distributional
error on uniform distribution; see \autoref{thm:main2} at the
end of this section. Indeed, we can decrease the error of a
randomized protocol below any constant $\epsilon>0$ for the
price of increasing the message length by a constant factor,
then we can fix the coins of this low error protocol in a way
that makes the resulting deterministic protocol $Q$ err in at
most $\epsilon$ fraction of the possible inputs. Applying
\autoref{thm:main2} to the protocol $Q$ proves
\autoref{thm:main}.

In the rest of this section we use a round-elimination argument
to prove \autoref{thm:main2}, that is, we will use $Q$ to solve
smaller instances of the exists-equal problem in a way that the
first message is always the same, and hence can be eliminated.

Suppose Alice sends the first message of $c$ bits in protocol $Q$. 
By \autoref{lem:determine-s}, there exists a $S\subset [t]^n$ of
size $\mu(S)=2^{-c-1}$ such that the first message of Alice is
fixed when $x\in S$ and we have $\Pr_{y\sim
\mu}[Q(x,y)\neq\EE(x,y)]\leq 2\epsilon$ for all $x\in S$. Fix
such a set $S$ and let $k\defeq t/2^{\frac{5(c+1)}{4n} + 1}$ and
$M \defeq nk/(20t)$. By \autoref{thm:isoperimetry}, there exists
a $T\subset S$ of size $k^{n/5}$ and $I\subset[n]$ of size $n/5$
such that defining
\begin{align*}
N_x=\{y\in T\mid\Match(x_I,y_I)\ge M\}
\end{align*}
we have $\Pr_{x\sim\mu}[N_x=\emptyset] \le 5^{-M}$ and
$\E_{x\sim \mu}[\log|N_x|]\geq (n/5-M)\log k - n\log k /5^M$.
Let us fix such sets $T$ and $I$. Note also that
\autoref{thm:isoperimetry} guarantees that $T$ is a strict
subset of $S$. Designate an arbitrary element of $S\setminus T$
as $x'_e$.

\subsection{Embedding the smaller problem}
\label{sec:embed}
Let $n'\defeq M/10$ and $t'\defeq4n'$. Suppose Alice and Bob
are given an instance $(u,v)$ of the $\EE_{n'}$ problem,
where $u,v\in[t']^{n'}$. To compute $\EE(u,v)$, 
through a random process, Alice and Bob will map $(u,v)$ to
random vectors $(X', Y)$, where $X'$ and $Y$ are supported on 
$\in[t]^n$, and then run the protocol on $\EE(X',Y)$.
The players embed a smaller instance $u,v\in[t']^{n'}$ of the
exists-equal problem in $\EE_n$ concentrating on the coordinates
$I$ determined above. We set $n'\defeq M/10$ and $t'\defeq4n'$.
Optimally, the same embedding should guarantee low error
probability for all pairs of inputs, but for technical reasons
we need to know the number of coordinate agreements
$\Match(u,v)$ for the input pairs $(u,v)$ in the smaller problem
having $\EE_{n'}(u,v)=1$. Let $R\ge1$ be this number, so we are
interested in inputs $u,v\in[t']^{n'}$ with $\Match(u,v)=0$ or
$R$. We need this extra parameter so that we can eliminate a
non-constant number of rounds and still keep the error bound a
constant. For results on constant round protocols one can
concentrate on the $R=1$ case.

In order to solve the exist-equal problem with parameters $t'$
and $n'$ Alice and Bob use the shared random source to turn their
input $u,v\in[t']^{n'}$ into longer random vectors
$X',Y\in[t]^n$, respectively, and apply the protocol $Q$ above
to solve this exists-equal problem for these larger inputs. Here
we informally list the main requirements on the process
generating $X'$ and $Y$. We require these properties for the
random vectors $X',Y\in[t]^n$ generated from a fixed pair
$u,v\in[t']^{n'}$ satisfying $\Match(u,v)=0$ or $R$.

\begin{enumerate}[(P1)]
\item $\EE(X',Y)=\EE(u,v)$ with large probability,
%\item if $\Match(u,v)=r \geq 1$ and $R\leq r$, then 
%$\Pr[\EE(X', Y)=1] \ge 0.80$;
\label{prop:2}
\item $\supp(X')=T\cup \{x'_e\}$ and
\label{prop:3}
\item $Y\emid X'$ is distributed almost uniformly
\label{prop:4}
\end{enumerate}

Combining these properties with the fact that $\Pr_{y\sim
\mu}[Q(x,y)\neq\EE(x,y)]\leq 2\epsilon$ for each $x\in S$, we
will argue that for the considered pairs of inputs $Q(X',Y)$
equals $\EE(u,v)$ with large probability, thus the combined
protocol solves the small exists-equal instance with small
error, at least for input pairs with $\Match(u,v)=0$ or $R$.
Furthermore, by \propref{prop:3} the first message of Alice will
be fixed and hence does not need to be sent, making the combined
protocol one round shorter.

The random variables $X'$ and $Y$ are constructed as follows.
Let $m\defeq 2n/(MR)$ be an integer. Each player repeats his or
her input ($u$ and $v$, respectively) $m$ times, obtaining a
vector of size $n/(5R)$. Then using the shared randomness, the
players pick $n/(5R)$ uniform random maps $m_i\colon[t']\to[t]$
independently and apply $m_i$ to $i$\/th coordinate.
Furthermore, the players pick a uniform random 1-1 mapping
$\pi\colon[n/(5R)]\to I$ and use it to embed the coordinates of the
vectors they constructed among the coordinates of the vectors
$X$ and $Y$ of length $n$. The remaining $n-n/(5R)$ coordinates
of $X$ is picked uniformly at random by Alice and similarly, the
remaining $n-n/(5R)$ coordinates of $Y$ is picked uniformly at
random by Bob. Note that the marginal distribution of both $X$
and $Y$ are uniform on $[t]^n$. If $\Match(u,v)=0$ the vectors
$X$ and $Y$ are independent, while if $\Match(u,v)=R$, then $Y$
can be obtained by selecting a random subset of $I$ of
cardinality $mR$, copying the corresponding coordinates of $X$
and filling the rest of $Y$ uniformly at random.

This completes the description of the random process for Bob.
However Alice generates one more random variable $X'$ as
follows. Recall that $N_x=\{z\in T\mid\Match(z_I,x_I)\ge M\}$.
The random variable $X'$ is obtained by drawing $x\sim X$ first
and then choosing a uniform random element of $N_x$. In the
(unlikely) case that $N_x=\emptyset$, Alice chooses $X'=x'_e$.

Note that $X'$ either equals $x'_e$ or takes values from $T$,
hence \propref{prop:3} holds. In the next lemma we quantify and
prove \propref{prop:2} as well.

\begin{lemma}
\label{lem:error}
Assume $n\ge3$, $M\ge2$ and $u,v\in[t']^{n'}$. We have
\begin{enumerate}[(i)]
\item 
if $\Match(u,v)=0$ then $\Pr[\EE(X',Y)=0] > 0.77$;
\item 
if $\Match(u,v)=R$, then
$\Pr[\EE(X', Y)=1] \ge 0.80$.
\end{enumerate}
\end{lemma}

\begin{proof}
For the first claim, note that when $\Match(u,v) = 0$, the
random variables $X$ and $Y$ are independent and uniformly
distributed. We construct $X'$ based on $X$, so its value is
also independent of $Y$. Hence $\Pr[\EE(X',Y)=0]=(1-1/t)^n$.
This quantity goes to $e^{-1/4}$ since $t=4n$ and is larger than
$0.77$ when $n\geq 3$. This establishes the first claim.

For the second claim let $J = \{i\in I\mid X_i=Y_i\}$ and
$K=\{i\in I\mid X'_i = X_i\}$. By construction,
$|J|=\Match(X_{I},Y_{I})\ge mR$ and $|K|=\Match(X'_{I}, X_I)
\geq M$ unless $N_X=\emptyset$. By our construction, each
$J\subset I$ of the same size is equally likely by symmetry,
even when we condition on a fix value of $X$ and $X'$. Thus we
have $\E[|J\cap K|\emid N_X\ne\emptyset]\ge mRM/|I|=10$ and
$\Pr[J\cap K=\emptyset\emid N_X\ne\emptyset]<e^{-10}$. Note that
$X$ is distributed uniformly over $[t]^n$, therefore by
Theorem~\ref{thm:isoperimetry}(i) the probability that
$N_X=\emptyset$ is at most $5^{-M}$. Note that
$\Match(X',Y)\ge|J\cap K|$ and thus
$\Pr[\EE(X',Y)=0]\le\Pr[J\cap K=\emptyset]\le\Pr[J\cap
K=\emptyset\emid N_X\ne\emptyset] +\Pr[N_X=\emptyset]\le
e^{-10}+5^{-M}$. This completes the proof.
\end{proof}

We quantify the correlation of $X'$ and $Y$ stated in
\propref{prop:4} by their mutual information. This mutual
information argument is postponed to the next subsection; here
we show how such a bound to the mutual information implies that
the error introduced by $Q$ is small.

\begin{lemma}
\label{lem:kl-err}
Let $\gamma = \Pr[Q(X',Y)\neq \EE(X',Y)]$. 
If $\gamma \ge 2\eps$, then we have $\kldivb{\gamma}{2\eps}\le \muti{X'}{Y}$.
\end{lemma}
\begin{proof}
For all fixings $x\in\supp(X')$ and a distribution $\nu$
on $[t]^n$, define 
\begin{align*}
e_{x}(\nu) \defeq \Pr_{y\sim \nu}[Q(x, y)\neq\EE(x, y)].
\end{align*}
By the definition of mutual information and the conditional divergence, 
%We prove the contrapositive of the
%statement of the lemma, that is assuming $\Pr_{y\sim Y|X'=x'}[Q(x',y) 
%\neq \EE(x',y)]>\gamma$ we prove $\Ent(Y\emid X'=x')< n\log t - 
%\BD(\gamma \dmid 2\epsilon)$:
\begin{align*}
\muti{X'}{Y}
    &=   \kldiv{Y\emid X'}{Y}\\
    &=   \E_{x\sim X'}\kldiv{Y\emid X'=x}{Y}\\
    &\ge \E_{x\sim X'}
         \kldivb{e_x\nparen{\dist(Y\emid X'=x)}}
                         {e_x\nparen{\mu}}\\
    &\ge \kldivb{\E_{x\sim X'}e_x\nparen{\dist(Y\emid X'=x)}}
                         {\E_{x\sim X'}e_x\nparen{\mu}}\\
    &= \kldivb{\gamma}{\E_{x\sim X'}e_x\nparen{\mu}}\\
    &\ge \kldivb{\gamma}{2\eps}
\end{align*}
where the first inequality is the data processing inequality,
the second inequality follows from the convexity of $\kldiv{\cdot}{\cdot}$
and the last inequality follows from 
the guarantee $e_x(\mu)\le 2\eps$ 
for all $x$ that is provided by \autoref{lem:determine-s}
and the assumption of the present lemma that $\gamma\ge 2\eps$.
\end{proof}

\subsection{Establishing the low correlation property}

We quantify \propref{prop:4} using the mutual information. 
If $\Match(u,v)=R$ our process generates $X$
and $Y$ with the expected number $\E[\Match(X_I,Y_I)]$ of
matches only slightly more than the minimum $mR$. We lose most
of these matches with $Y$ when we replace $X$ by $X'$ and only
an expected constant number remains. A constant number of forced
matches with $X'$ within $I$ restricts the number of possible
vectors $Y$ but it only decreases the entropy by $O(1)$. The
calculations in this subsection make this intuitive argument
precise.

Recall that $X$ and $Y$ are correlated due to the random process
with which Alice and Bob generate them and $X'$ is obtained from
$X$. To understand $\muti{X'}{Y}$
\begin{lemma}
\label{lem:x-uniform}
For any $u,v\in[t']^{n'}$ it holds that
$\muti{X}{X'}\le M\log k  + n\log k / 5^M$.
\end{lemma}
\begin{proof}
We have
\begin{align*}
\muti{X}{X'} = \ent{X'} - \ent{X'\emid X}
\end{align*}
and 
$\ent{X'}\le \log |\supp(X')|
    = \log \nparen{|T|+1}\le \frac{n}{5}\log k + 1$.

Observe that $\ent{X'\emid X} = \E_{x\sim
\mu}[\log|N_x|]$, where $\log 0$ is now taken to be $0$. From
\autoref{thm:isoperimetry}(ii) we get $\ent{X'\emid X}\geq
\frac{n}{5}\log k - M\log k - n\log k/5^M$.
Plugging in, the $\frac{n}{5}\log k$ terms cancel and we get
the statement of the lemma.
\end{proof}


\begin{lemma}
\label{lem:y-uniform}
Let $X',Y$ be as constructed above. The following hold.
\begin{enumerate}[(i)]
\item If $\Match(u,v)=0$ we have
$\muti{X'}{Y} = 0$
\item If $M>100\log n$ and $\Match(u,v)=R$ we have
$\muti{X'}{Y} = O(1).$
\end{enumerate}
\end{lemma}
\begin{proof}
Part (i) holds as $Y$ is independent of $X'$
whenever $\EE(u,v)=0$ by construction.

For part (ii) recall that if $\Match(u,v)=R$ one can construct
$X$ and $Y$ by uniformly selecting a size $mR$ set $L\subseteq
I$ and selecting $X$ and $Y$ uniformly among all pairs
satisfying $X_L=Y_L$. Recall that $L$ is the set of coordinates
the $mR$ matches between $u^m$ and $v^m$ were mapped. These are
the ``intentional matches'' between $X_I$ and $Y_I$. Note that
there may be also ``unintended matches'' between $X_I$ and
$Y_I$, but not too many: their expected number is
$(n/5-mR)/t<1/20$. As given any fixed $L$, the marginal
distribution of both $X$ and $Y$ are still uniform, so in
particular $X$ is independent of $L$ and so is $X'$ constructed
from $X$. Let us expand $\muti{X'L}{Y}$ using the chain rule in
two different ways obtaining
\begin{align}
\muti{X'L}{Y}
    &= \muti{X'}{Y} + \muti{L}{Y\emid X'}\nonumber\\
    &= \muti{L}{Y} + \muti{X'}{Y\emid L}\label{teq:mutexpand}.
\end{align}
Since the first term of \eqref{teq:mutexpand} is zero 
by independence of $Y$ and $L$, we conclude
\begin{align}
\muti{X'}{Y}
    &= \muti{X'}{Y\emid L} - \muti{L}{Y\emid X'}\nonumber\\
    &= \muti{X'}{Y\emid L} - \muti{L}{X'Y}\label{teq:twoterms},
\end{align}
where the second inequality follows again by the chain rule 
and the fact that $X'$ and $L$ are independent.
Let us understand the terms of \eqref{teq:twoterms} one by one.
First we  expand the first term by the chain rule, obtaining
\begin{align*}
\muti{X'}{Y\emid L} = \muti{X'}{Y_L\emid L} + \muti{X'}{Y_{[n]\setminus L}\emid LY_L}
\end{align*}
however since $Y_{[n]\setminus L}$ is uniformly distributed for any fixed
$L$, $X'$ and $Y_L$, the second term on the right hand side is zero. 
By construction we have $X_L=Y_L$, thus
\begin{align*}
\muti{Y_L}{X'\emid L} 
    &= \muti{X_L}{X'\emid L}\\
    &\le \frac{mR}{n/5}\muti{X_I}{X'}\\
    &\le 10\log k + \frac{MR}{5^{M-1}}\log k,
\end{align*}
where the first inequality follows by \autoref{lem:ent-subset}
as $L$ is a uniform and independent of $X$ and $X'$ and the
second inequality follows from \autoref{lem:x-uniform} that we
will prove shortly and the formula defining $m$.

Here, when condition on $L$, the correlation of $X'$ and $Y$
is roughly $10\log k$ bits, which is significantly more than the
constant bound we seek. Next we will see that all but a constant bits
of this correlation comes from having observed what $L$ is.
The next term, $\ent{L}$ is easy to compute as $L$ is a uniform
subset of $I$ of size $mR$:
\begin{align*}
\ent{L}=\log{n/5\choose mR}
\end{align*}

It remains to bound the term $\ent{L\emid Y, X'}$. Let
$Z=\{i\mid i\in I \text{ and } X'_i=Y_i\}$. Note that $Z$ can be
derived from $X',Y$ (as $I$ is fixed) hence 
$\ent{L\emid Y,X'}\leq \ent{L\emid Z}$. 
Further, let $C=|Z\setminus L|$. We obtain
\begin{align*}
\ent{L\emid Y,X'}&\le\ent{L\emid Z}\leq \ent{L\emid Z, C} + \ent{C}\\
&< \E_{Z,C}\left[\log{n/5-|Z|+C \choose mR - |Z|+C}\right]
+ \E_{Z,C}\left[\log {|Z| \choose C}\right]+2
\end{align*}
where we used $\ent{C}<2$. Note that for any fixed $x'\in T$ and
$x\in \supp(X\emid X'=x')$, we have $$\E[|Z|-C\emid X=x,
X'=x']=\Match(x_I,x_I') mR /(n/5) \geq 10$$ as $\Match(x_I,
x_I')\geq M$ by definition.
Hence we have $$\log{n/5\choose mR}-\log{n/5-|Z|+|C|\choose
mR-|Z|+|C|}\ge10\log\frac n{5m}-O(1),$$ $$\E_{Z,C}\left[\log
{|Z| \choose C}\right] \le \E[|Z|] < 20.$$ Summing the estimates
above for the various parts of $\ent{Y\emid X'}$ the statement
of the lemma follows.
\end{proof}

It remains to prove the following simple lemma that ``reverses''
the conditional entropy bound in \autoref{thm:isoperimetry}(ii):

\subsection{The round elimination lemma}

Let $\nu_n$ be the uniform distribution on $[t]^n\times [t]^n$,
where we set $t=4n$. The following lemma gives the base case of
the round elimination argument.
\begin{lemma}\label{lem:terminal} 
Any 0-round deterministic protocol for $\EE_n$ has at least 0.22
distributional error on $\nu_n$, when $n\geq 1$.
\end{lemma}
\begin{proof}
The output of the protocol is decided by a single player, say
Bob. For any given input $y\in[t]^n$ we have $3/4
\leq\Pr_{x\sim\mu}[\EE(x,y)=0] < e^{-1/4} < 0.78$. Therefore the
distributional error is at least $0.22$ for any given $y$
regardless of the output Bob chooses, thus the overall error is
also at least $0.22$.
\end{proof}
Now we give our full round elimination lemma.
\begin{lemma}\label{lem:roundel} 
Let $r>0, c ,n$ be an integers such that $c < \frac{4}{5}n\log n$.
There is a constant $0<\epsilon_0<1/200$ such that if there is
an $r$-round deterministic protocol with $c$-bit messages for
$\EE_n$ that has $\epsilon_0$ error on $\nu_n$, then there is an
$(r-1)$-round deterministic protocol with $O(c)$-bit messages
for $\EE_{n'}$ that has $\epsilon_0$ error on $\nu_{n'}$, where
$n' = \Omega(n/2^\frac{5c}{4n})$.
\end{lemma}
\begin{proof}
We start with an intuitive description of our reduction. Let us
be given the deterministic protocol $Q$ for $\EE_n$ that errs on
an $\epsilon_0$ fraction of the inputs. To solve an instance
$(u,v)$ of the smaller $\EE_{n'}$ problem the players perform
the embedding procedure described in previous subsection $k_0$
times independently for each parameter $R\in[R_0]$. Here $k_0$
and $R_0$ are constants we set later. They perform the protocol
$Q$ in parallel for each of the $k_0R_0$ pairs of inputs they
generated. Then they take the majority of the $k_0$ outputs for
a fixed parameter $R$. We show that this result gives the
correct value of $\EE(u,v)$ with large probability provided that
$\Match(u,v)=0$ or $R$. Finally they take the OR of these
results for the $R_0$ possible values of $R$. By the union bound
this gives the correct value $\EE(u,v)$ with large probability
provided $\Match(u,v)\le R_0$. Fixing the random choices of the
reduction we obtain a deterministic protocol. The probability of
error for the uniform random input can only grow by the small
probability that $\Match(u,v)>R_0$ and we make sure it remains
below $\epsilon_0$. The rest of the proof makes this argument
precise.

For random variables $X'$ and $Y$ constructed in
\autoref{sec:embed}, \autoref{lem:y-uniform} guarantees that
$\ent{Y\emid X'}\ge n\log t - \alpha_0$ for some constant
$\alpha_0$, as long as $M>100\log n$ and $\Match(u,v)=R$. Let
$\epsilon_0$ be a constant such that $\kldivb{1/10}{2\epsilon_0}
> 200(\alpha_0 + 1)$. Note that such $\epsilon_0$ can be found
as $\kldivb{1/10}{\epsilon}$ tends to infinity as $\epsilon$
goes to 0. We can bound $\Pr_{(x,y)\sim\nu_m}[\Match(x,y) \ge l]
\le 1/(4^l l!)$ for all $m\ge1$. We set $R_0$ such that
$\Pr_{(x,y)\sim\nu_m}[\Match(x,y) \ge R_0 ] \le \epsilon_0 / 2$
for all $m\ge1$.

Let $Q$ be a deterministic protocol for $\EE_n$ that sends $c <
(n\log n)/2$ in each round and that has $\epsilon_0$ error on
$\nu_n$. Let $S$ be as constructed in \autoref{lem:determine-s}
and let $M$ be as defined in \autoref{thm:isoperimetry}. We have
$M=\frac{n}{40}2^{\frac{-5(c+1)}{4n}}$ as $t=4n$ and
$\mu(S)=2^{-(c+1)}$ by \autoref{lem:determine-s}. Note that by
our choice of $c$, we have $M>100\log n$, hence the hypotheses
of \autoref{lem:y-uniform} are satisfied.

Let $n' = M/10 = \frac{n}{400}2^{\frac{-5(c+1)}{4n}}$. Now we
give a randomized protocol $Q'$ for $\EE_{n'}$. Suppose the
players are given an instance of $\EE_{n'}$, namely the vectors
$(u,v)\in[4n']^{n'}\times[4n']^{n'}$. Let $k_0 = 10\log (R_0 +
1/\epsilon_0)$. For $R\in[R_0]$ and $k\in [k_0]$, the players
construct the vectors $X'_{R,k}$ and $Y_{R,k}$ as described in
\autoref{sec:embed} with parameter $R$ and with fresh randomness
for each of the $R_0k_0$ procedures. The players run $R_0 k_0$
instances of protocol $Q$ in parallel, on inputs $X'_{R,k},
Y_{R,k}$ for $R\in[R_0]$ and $k\in[k_0]$. Note that the first
message of the first player, Alice, is fixed for all instances
of $Q$ by \propref{prop:3} and \autoref{lem:determine-s}.
Therefore, the second player, Bob, can start the protocol
assuming Alice has sent the fixed first message. After the
protocols finish, for each $R\in[R_0]$, the last player who
received a message computes $b_R$ as the majority of
$Q(X_{R,k}',Y_{R,k})$ for $k\in [k_0]$. Finally, this player
outputs $0$ if $b_R=0$ for all $R\in[R_0]$ and outputs $1$
otherwise.

Suppose now that $\EE(u,v) = 0$. By \autoref{lem:error}(i), we
have $\Pr[\EE(X'_{R,k},Y_{R,k}) = 0] \ge 0.77$ for each $R$ and
$k$. Recall that that $Y_{R,k}$ is distributed uniformly for
each $R$ and $k$ and since $\EE(u,v)=0$, it is independent of
$X'_{R,k}$. Therefore, by $X'_{R,k}\in S$ (\propref{prop:3}) and
the fact that $\Pr_{y\sim \mu}[Q(x,y)\neq\EE(x,y)]\leq
2\epsilon_0$ for all $x\in S$ as per \autoref{lem:determine-s},
we obtain $\Pr[Q(X'_{R,k},Y_{R,k}) = 0] \ge 0.77 - 2\epsilon_0 >
0.76$. By the Chernoff bound we have $\Pr[b_R = 1] <
\epsilon_0/(2R_0)$, and by the union bound $\Pr[Q'\hbox{ outputs
}0]\ge1-\epsilon_0 /2$.

Let us now consider the case $\Match(u,v) = R$ for some
$R\in[R_0]$. Fix any $k\in[k_o]$ and set $X'=X'_{R,k}$,
$Y=Y_{R,k}$. By \autoref{lem:error}(ii), $\Pr[\EE(X',Y) = 1]\ge
0.80$. By \autoref{lem:y-uniform}, $\muti{X'}{Y}\leq \alpha_0$ 
and that $Y$ is distributed uniformly at random.
By \autoref{lem:kl-err} and our choice of $\epsilon_0$, we
have $\Pr[\EE(X', Y)\neq Q(X', Y)] < 1/10$. Furthermore, by
\autoref{lem:error}(ii), $\Pr[\EE(u,v) \neq \EE(X', Y)]< 0.20$
hence with probability at least $0.70$ we have $\EE(u,v) = Q(X',
Y)$. This happens independently for all the values of
$k\in[k_0]$, so by the Chernoff bound and our choice of $k_0$,
we have $\Pr[Q'\hbox{ outputs }0]\le\Pr[b_R = 0] < \epsilon_0 /
2$.

Finally, $\Pr_{(u,v)\sim \nu_{n'}}[\Match(u,v) \ge R_0] \le
\epsilon_0 /2$ by our choice of $R_0$. Note that the protocol
$Q'$ uses a shared random bit string, say $W$, in the
construction of the vectors $X'_{R,k}$ and $Y_{R,k}$. Hence,
overall, we have
\begin{align*}
  \Pr_{W, (u,v)\sim\nu_{n'}}[\EE(u,v) = Q'(u,v)]
      \ge 1 - \eps_0
\end{align*}
Since we measure the error of the protocol under a
distribution, we can fix $W$ to a value without increasing the
error under the aforementioned distribution by the so called
easy direction of Yao's lemma. Namely, there exists a $w\in
\supp(W)$ such that
\begin{align*}
  \Pr_{(u,v)\sim\nu_{n'}}[\EE(u,v)  = Q'(u,v)\emid W=w] 
      \ge 1 - \eps_0
\end{align*}
Fix such $w$. Observe that $Q'$ is a $(r-1)$-round protocol for
$\EE_{n'}$ where
$n'=\frac{n}{400}2^\frac{-5(c+1)}{4n}=\Omega(n/2^\frac{5c}{4n})$
and it sends at most $R_0k_0c = O(c)$ bits in each message.
Furthermore, $Q'$ is deterministic and has at most $\eps_0$
error on $\nu_{n'}$ as desired.
\end{proof}

\begin{theorem}
\label{thm:main2}
There exists a constant $\epsilon_0$ such that for any $r\leq\log^*n$,
an $r$-round deterministic protocol for $\EE_n$ which has
$\epsilon_0$ error on $\nu_n$ sends at least one message of size 
$\Omega(n\log^{(r)}n)$.
\end{theorem}
\begin{proof}
Suppose we have an $r$-round protocol with $c$-bit messages for
$\EE_n$ that has $\epsilon_0$ error on $\nu_n$, where $c=\gamma
n\log^{(r)}n$ for some $\gamma<4/5-o(1)$. By
\autoref{lem:roundel}, this protocol can be converted to an
$r-1$ round protocol with $\alpha c$-bit messages for $\EE_{n'}$
that has $\epsilon_0$-error on $\nu_{n'}$, where $n'=\beta
n/2^{5c/4n}$ for some $\alpha, \beta >0$. We only need to verify
that $\alpha c \leq \gamma n'\log^{(r-1)}n'$. We have
\begin{align*}
\gamma n'\log^{(r-1)} n' &= \gamma\beta n/2^{5c/4n}\log^{(r-1)}
(\beta n/2^{5c/4n})\\
&= \gamma\beta n/2^{\frac{5\gamma}{4}\log^{(r)}n}\log^{(r-1)}
(\beta n/2^{5c/4n})\\
&\geq \gamma\beta n \left(\log^{(r-1)}n\right)^{1-\frac{5\gamma}{4}-o(1)}\\
&\geq \gamma\alpha n\log^{(r)}n
\end{align*}
for $\gamma< 4/5 - o(1)$ and large enough $n$. Therefore, by
iteratively applying \autoref{lem:roundel} we obtain a $0$-round
protocol for $\EE_{\bar n}$ that makes $\epsilon_0$ error on
$\nu_{\bar n}$ for some $\bar n$ satisfying $\gamma {\bar n}^2 =
\gamma \bar n \log^{(0)} \bar n\geq c \alpha^r$. Therefore $\bar
n \geq 1$ and since $\epsilon_0< 0.22$, the protocol we obtain
contradicts \autoref{lem:terminal}, showing that the protocol we
started with cannot exists.
\end{proof}
\begin{remark}
We note that in the proof of \autoref{thm:main}, to show that a
protocol with small communication does not exist, we start with
the given protocol and apply the round elimination lemma (i.e.,
\autoref{lem:roundel}) $r$ times to obtain a $0$-round protocol
with small error probability, which is shown to be impossible by
\autoref{lem:terminal}. Alternatively, one can apply the round
elimination $r-1$ times to obtain a $1$-round protocol with
$o(n\log n)$ communication for $\EE_{n}$, which is ruled out by
\autoref{thm:singleround}.
\end{remark}
