% !TeX root = thesis.tex
\section{Discussion}
\label{sec:ham:discussion}
We showed that for a symmetric matrix 
$S\colon\Omega\times\Omega\to\realspos$ and 
unit vectors $u,v\colon\Omega\to\realspos$, defining
$m_t = \iprod{v}{S^tu}$ for $t=0,1,\ldots$, we have 
\begin{align}
m_{t+2}    &\ge m_t^{1+2/t}, \text{ and}\label{eq:dbd66}\\
m_{t+2}    &\ge m_t^{1+2/t}\cdot \min\set{t^{1-\eps}, 
\ceil{\delta\frac{m_t^{1-2/t}}{m_{t-2}}}} \label{eq:dconv}
\end{align}
and argued that \autoref{eq:dconv} and \eqref{eq:dbd66}, 
in this order, are best viewed as gradual weakenings of
the log-convexity of $\set{m_t}_{t=0}^\infty$.
We conjecture that a similar principle holds
true for continuous time Markov chains as well.

Call a function $f\colon\realspos\to[0,1]$,
whose logarithm is continuously twice differentiable 
(i.e., $\log f\in C^2(\realspos)$), {\em nearly-log-convex} 
if $x^2 (\log f)''(x)\ge 2 \log f(x)$ for $x\in\realspos$. 
Note that $\log f \le 0$, therefore this is a weakening
of the usual log-convexity definition, which requires 
$(\log f)''\ge 0$.

\begin{conjecture}
\label{conj:continuouslogconv}
Let $S\colon\Omega\times\Omega\to\realspos$
be a symmetric substochastic matrix and $u,v\colon\Omega\to\realspos$
be unit vectors. The function
\begin{align*}
t\mapsto \iprod{v}{e^{t(S-I)}u}
\end{align*}
is nearly-log-convex.
\end{conjecture}
By an argument similar to the proof of \autoref{thm:klogk}, one can 
show the following.
\begin{theorem}
\autoref{conj:continuouslogconv} implies \autoref{conj:ghd}.
\end{theorem}

\section{Chapter notes}
\label{ham:notes}
The results of this chapter have been published in FOCS 2018 in \cite{Saglam2018}.