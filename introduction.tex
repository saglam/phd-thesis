% !TeX root = thesis.tex
\chapter{Introduction}
\label{sec:introduction}

Communication complexity is a versatile abstraction of communication requirements of computational tasks, leaving aside the compute power required and focusing on bandwidth, latency and everything else communication related.


What is communication complexity

Who introduced it

Why is it useful

Useful on its own; an interactive generalization of Shannon's theory of communication.



In a two player communication problem the players, 
named Alice and Bob, receive separate inputs, 
respectively $x\in \mathcal{X}$ and $y\in \mathcal{Y}$, 
and they communicate in order to compute the value 
$f(x,y)$ of a function
$f\colon \mathcal{X}\times \mathcal{Y}\to\binary$ 
(known to both players). 
In an $r$-round protocol, the players can take at most 
$r$ turns alternately sending each other a message 
(that is, a bit string) and the last player to receive 
a message declares the output of the protocol.
A protocol can be {\em deterministic} or {\em randomized}; 
in the latter case the players can base their actions on 
a common random source and we measure the 
{\em error probability}: the maximum over inputs 
$(x,y)\in \mathcal{X}\times\mathcal{Y}$, of the probability 
that the output of the protocol differs from $f(x,y)$. 
The {\em communication cost} of a protocol is the maximum, 
over the inputs and the random string, of the total number 
of bits sent between the players.
For a function 
$f\colon \mathcal{X}\times\mathcal{Y}\to\binary$, 
an integer $r$ and $\delta\in[0,1]$, we denote by 
$R^r_{\delta}(f)$ the minimum over all protocols
for $f$ having $r$-rounds and error probability at most 
$\delta$, of the communication cost incurred. We define 
$R_{\delta}(f)$ similarly, but we take the maximum over 
$\delta$-error protocols with no restriction on the number 
of rounds it uses.

In the $k$-Hamming distance problem, denoted $\Ham^n_k$,
the players receive length-$n$ bit strings, respectively 
$x,y\in\cube$, and are required determine if 
$\lone{x-y}< k$ or not.
There is a well known one-round communication protocol
which accomplishes this with error probability $\delta$ 
by communicating $O(k\log\nparen{k/\delta})$ bits.

\begin{theorem}
[e.g., Huang, Shi, Zhang and Zhu \cite{HuangSZZ2006}]
\label{thm:ub}
It holds that $$R^1_\delta(\Ham^n_k) = 
O(\min\set{k\log\nparen{k/\delta}, k\log (n/k)}).$$
\end{theorem}

Highly related to the $\Ham^n_k$ is the $k$-disjointness 
problem $\Disj^n_k$, wherein the players each receive a 
$k$-subset of $[n]$ and their goal is to determine if their 
sets intersect.
Notice that $\Disj^n_k$ can be seen as a promise version of
$\Ham^n_{2k-2}$ where each player is guaranteed to have a string
with Hamming weight $k$: the sets are disjoint if and only if the
Hamming distance between the characteristic vectors of the sets
is more than $2k-2$. Therefore any upper bound for the $\Ham^n_k$ 
carries over to $\Disj^n_k$ and any lower bound for $\Disj^n_k$
carries over to $\Ham^n_k$.
Around 1993, 
Håstad and Wigderson \cite{HastadW2007} showed that
there is a more efficient protocol for $\Disj^n_k$ than that implied 
by \autoref{thm:ub}, which communicates only $O(k)$ bits, but over
$O(\log k)$ rounds.

On the lower bounds side, the result of \cite{KalyanasundaramS1992}
implies that $\Omega(k)$ bits is needed for these problems even
if one uses arbitrarily large number of round protocols.
In \cite{BuhrmanGMW2012} it was shown that
any 1-round protocol for $\Disj^n_k$ needs to communicate
at least $\Omega(k \log k)$ bits when $k^2<n$
(this result was proven later in \cite{DasguptaKS12} also).
In Theorem~3.2 of \cite{Saglam2011}, an $\Omega(k\log(1/\delta))$
bound for $1$-round complexity of $\Disj^n_k$ was shown even 
when Bob receives just one element (i.e., the indexing problem) 
for $k<\delta n$ and a 
slightly more general result was shown in \cite{JayramW2011}.
Finally Sağlam and Tardos in \cite{SaglamT2013}, 
which we review in sections \ref{sec:disj-upperbound} and 
\ref{sec:disj-lowerbound} of this document, the communication complexity
of $\Disj^n_k$ was settled:
\begin{align*}
R^r_{1/3}(\Disj^n_k)=\Theta(k\log^{(r)}k)
\end{align*}
for $1\le r \le \log^*k$ and $k^2<n$.
Our upper bound solves the disjointness problem with 
error probability at most $1/\exp k + 1/\exp^{(r)}(c\log^{(r)}k)$ 
for any $c>1$ by communicating
$O(k \log^{(r)} k)$ bits over $r$ rounds. In fact bulk of the bits
is sent in the first round and the rest of the rounds
amount to an $O(k)$ bits of communication. Taking $r=\log^* k$,
this leads to an $O(k)$ bits protocol with error probability 
that is exponentially small in $k$.
Their lower bound shows that at least one
message of size 
$\Omega(k\log^{(r)} k)$ bits needs to be sent by any 
$r$-round protocol, even if it has error probability $1/3$.
Prior to this work, this lower bound provided the strongest lower
bound for $\Ham^n_k$ also, along with the incomparable bound of 
$\Omega(k\log (1/\delta))$ due to \cite{BlaisBG2014}
which holds for any number of rounds, which we discuss shortly.

\input{hamvsdisj}

To summarize the above results, the 1-round communication 
complexity of both $\Disj^n_k$ and $\Ham^n_k$ is 
$\Theta(k\log(k/\delta))$ by 
\cite{BuhrmanGMW2012, Saglam2011, JayramW2011} and 
\cite{HuangSZZ2006}. We know that $\Disj^n_k$ can be solved 
much more efficiently if one is allowed multiple rounds:
firstly the $\log k$ factor can be removed \cite{HastadW2007} 
and secondly the error probability can be brought down to 
$\exp(-k)$, by using no more than $\log^* k$ rounds 
\cite{SaglamT2013}. It is an interesting question whether 
similar efficiency improvements can be obtained for $\Ham^n_k$ 
also, by using multiple rounds.
The first separation of $\Disj^n_k$ and $\Ham^n_k$ was proven
in \cite{BlaisBG2014}, which shows that $\Omega(k\log(1/\delta))$
lower bound holds for any protocol solving $\Ham^n_k$. 
Therefore in $\Ham^n_k$, we get no improvements in error probability
by interactive communication. It remained an open 
question whether {\em any} improvement can be made 
at all to the 1-round protocol by communicating interactively.
In \autoref{sec:ham-lowerbound} we answer this question negatively: 
\begin{theorem}
\label{thm:klogk}
Any two party $\delta$-error randomized protocol
solving the $k$-Hamming distance problem 
over length-$n$ strings communicates 
at least $\Omega(k\log (k/\delta))$ 
bits for $k^2\le \delta n$.
\end{theorem}